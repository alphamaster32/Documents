\documentclass{beamer}

\usepackage{wrapfig}
\usepackage{algpseudocode}
\usepackage{algorithm}
\usepackage[export]{adjustbox}
\usepackage{svg}

\usetheme{Boadilla}
\usecolortheme{dolphin}
\setbeamertemplate{navigation symbols}{}
\setbeamertemplate{sections/subsections in toc}[sections numbered]

\setbeameroption{hide notes}
% \setbeameroption{show only notes}
% \setbeameroption{show notes on second screen=right}

\title[Adaptive Embedded Trace]
{Adaptive Embedded Trace: Assessment of Ideas}
\author[]{Hossein Afkar}
\institute{DRTS Lab}
\date{\today}

\begin{document}

\frame{\titlepage}

\begin{frame}
    \frametitle{Outline}
    \tableofcontents[hideallsubsections]
\end{frame}

\AtBeginSection[]
{
    \begin{frame}{Outline}
        \tableofcontents[currentsection]
    \end{frame}
}

\section{Preface}
\begin{frame}
    \frametitle{Preface}
    \begin{itemize}
        \item Embedded Trace in Cyber-Physical Systems.
        \item Exploring Adaptability.
        \item eBPF: What we already have in kernel space via software.
        \item Performance Observability Tools in Linux Kernel.
    \end{itemize}
\end{frame}

\section{Embedded Trace in Cyber-Physical Systems}
\begin{frame}
    \frametitle{Introduction}
    \begin{itemize}
        \item Advanced embedded system technology is one of the key
            driving forces behind the rapid growth
            of Cyber-Physical System (CPS) applications.
        \item Advanced architectures create challenges for testing and
            verification.
            \begin{itemize}
                \item Observability.
                \item Non-intrusiveness
            \end{itemize}
        \item A promising technology to address the observability issue is
            Embedded Trace (ET).
            \begin{itemize}
                \item Opens the window to the systems behaviour while
                    executing software.
                    \begin{itemize}
                        \item Revealing memory access patterns
                        \item Execution control
                        \item Data flow
                        \item Protocol sequences
                    \end{itemize}
            \end{itemize}
    \end{itemize}
\end{frame}

\begin{frame}
    \frametitle{Noted Excerpts From The Paper}
    \begin{itemize}
        \item Data-Flow messages are large and can consume trace bandwidth.
        \item ETM evolution:
            \begin{itemize}
                \item ETMv1 and ETMv2 report the pipeline state in a cycle by
                    cycle basis.
                \item ETMv3 abolishes pipeline state and introduces P-Header
                    packets that represent the state of the instructions.
                \item ETMv3 to ETMv1 only output the information of the flow of
                    execution when it can not be inferred from the execution
                    image.
                \item Observeability tradeoffs can be solved using ITM.
                \item ETMv4 reports the cancellation or completion of
                    speculative code execution.
            \end{itemize}
    \end{itemize}
\end{frame}
\section{Adaptability}
\begin{frame}
    \frametitle{Exploring What It Means to Adapt}
    \begin{itemize}
        \item Adaptability is defined in relation to a feature.
        \item Fault detection and various online error detection methods come
            to mind when mentioning adaptability.
        \item In order to be adaptable we should define what we need to adapt
            for
            \begin{itemize}
                \item Adapt to ?
            \end{itemize}
    \end{itemize}
\end{frame}
\begin{frame}
    \frametitle{Online Error Detection Through Trace Infrastructure in
    ARM Microprocessors}
    \begin{itemize}
        \item ASD
    \end{itemize}
\end{frame}
\section{eBPF: Extended Berekley Packet Filter}
\begin{frame}
    \frametitle{Extended Berekley Packet Filter}
    \begin{itemize}
        \item Extended berkekley packet filter is a tracing solution present
            in the linux kernel.
        \item BPF is an emulated machine that aids us to run controlled
            programs in kernel space.
    \end{itemize}
\end{frame}
\section{Memory Observability and Bandwidth Control}
\begin{frame}
    \frametitle{ASD}
\end{frame}

\begin{frame}
  \centering \Large
  \emph{Thank You For Your Attention}
\end{frame}

\end{document}
