\documentclass{beamer}

\usepackage{wrapfig}
\usepackage{algpseudocode}
\usepackage{algorithm}

\usetheme{Boadilla}
\usecolortheme{dolphin}
\setbeamertemplate{navigation symbols}{}
\setbeamertemplate{sections/subsections in toc}[sections numbered]

\title[Terabit Networking]
{Towrads fast Ethernet and terabit speeds}
\author[]{Hossein Afkar}
\institute{University of Tehran}
\date{\today}

\begin{document}

\frame{\titlepage}

\section{Introduction}
\begin{frame}
    \frametitle{Introduction}
    Too many critical flaws in the Linux networking stack has push us
    through rethinking the networking stacks that
    includes the packet processing pipeline, sync or async I/O, and user or
    kernel level processing of the packets.
    In this project we try to gather information and propose methods
    for increasing the linux or any other operating system networking speed but
    the main emphasis will be on linux.
\end{frame}

\begin{frame}
    \frametitle{Emphasis}
    In project we put our emphasis on these parts regarding the networking
    stacks.
    \begin{enumerate}
        \item Frequently cited flaws in packet processing pipeline and Linux
            networking stacks.
        \item Synchronous and Asynchronous IO and new syscalls like 
            \textit{io\_uring}.
        \item How Linux networking stack operates with regards to the
            scheduler and Timers and Interrupts.
        \item Hardware capabilities required to achieve better performance.
    \end{enumerate}
\end{frame}

\begin{frame}
    \frametitle{Project Incentive}
    The goal is to propose architectures and ways to imporve kernel networking
    speeds that can be implemented in the linux kernel or any other kenrel.
\end{frame}

\begin{frame}
  \centering \Large
  \emph{Thank You For Your Attention}
\end{frame}

\end{document}
