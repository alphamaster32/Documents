% LaTeX Template for short student reports.
% Citations should be in bibtex format and go in references.bib
\documentclass[a4paper, 12pt]{article}
\usepackage[top=2cm, bottom=3cm, left = 2cm, right = 2cm]{geometry} 
\geometry{a4paper} 
\usepackage[utf8]{inputenc}
\usepackage{textcomp}
\usepackage{graphicx} 
\usepackage{amsmath,amssymb}  
\usepackage{bm}  
\usepackage{memhfixc} 
\usepackage{fancyhdr}
\usepackage{float}
\usepackage{booktabs}
\usepackage{xepersian}
\settextfont[Scale=1.]{HM FNazli}
\setlatintextfont[Scale=.9]{Noto Sans}
\pagestyle{fancy}

\title{\lr{x86-TSO: A Rigorous and Usable Programmer’s Model for
x86 Multiprocessors}}
\author{حسین افکار}
%\date{}

\begin{document}
\maketitle
این مقاله در مورد
\lr{Memory Ordering}
بحث خواهد کرد.
در این مقاله یه مدل طبقه‌بندی حافظه به نام
\lr{Total Store Order}
معرفی خواهد شد. \\
نویسندگان این مقاله معتقدند که مدل فعلی پردازنده‌های
\lr{x86}
برای برنامه‌نویسان قابل فهم نمی‌باشد.
این مقاله با معرفی طبقه‌بندی حافظه‌هایی که در پردازنده‌های
\lr{x86}
موجود هستند شروع می‌کند. این پردازنده‌ها عموما از مدل‌های
\lr{Strongly Ordered}
استفاده می‌کنند که فهم آن‌ها سخت است و ممکن است برای برنامه‌نویس ابهام ایجاد کند.
برای همین این مقاله ایده
\lr{x86-TSO}
را ارائه می‌دهد و نشان می‌دهد که این مدل حافظه از بقیه قوی‌تر است و راحت‌تر قابل فهم است.
در ادامه نویسندگان این مقاله مدل این حافظه را برای صحت‌سنجی فورمال ارائه می‌دهند.
و همچنین به ما کمک می‌کنند که صحت‌سنجی را انجام دهیم.
ایده این
\lr{Memory Ordering}
توسط یک کلاک گلوبال که توسط همه هسته‌ها استفاده می‌شود، پیاده‌سازی شده است.
به عنوان مثال می‌شود از
\lr{TSC}
موجود در پردازنده برای این کار استفاده کرد. \\
با این وجود این مدل حافظه نیز از مشکلات مبرا نیست. به دلیل دادن قید‌های قوی برای
مرتب کردن دسترسی‌های به حافظه، می‌تواند باعث کاهش کارآیی شود.
با این وجود این مدل پذیرفته شده است و در زبان
\lr{C++}
استاندارد به عنوان مدل حافظه وجود دارد. \\
نقاط ضعف و قوت این مقاله عبارت‌اند از:
\begin{itemize}
    \item این مقاله یک مدل قوی و قابل صحت‌سنجی و قابل‌ استفاده برای برنامه‌نویسان مطرح کرده است
    \item در این مقاله ویژگی‌های مثبت این مدل در مقایسه با بقیه مدل‌ها ذکر شده است
    \item مدل صحت‌سنجی فورمال برای این مدل حافظه‌ای ذکر شده است که کار را برای صحت‌سنجی نرم‌افزار ساده می‌کند
    \item این مقاله نکات خوبی برای ارائه یک مدل حافظه‌ای جدید برای پردازنده‌های 
    \lr{x86}
    ارائه می‌دهد
    \item این مقاله توضیحاتی در مورد سنجش کارآیی این مدل در برابر بقیه مدل‌ها ارائه نمی‌دهد
    \item این‌که این مدل حافظه‌ای در چه مواردی ممکن است بد عمل کند نیز آورده نشده است
    \item این مقاله امکان ریلکس کردن برخی از نوشتن‌ها و خواندن‌ها را در برابر بقیه پردازنده‌ها نیز بررسی نکرده است
\end{itemize}
در آخر این مقاله مدل بسیار خوبی برای حافظه ارائه داده بود که در پردازنده‌های فعلی اینتل بخش‌هایی از این مدل نیز
استفاده می‌شود و در کل مقاله تأثیرگذاری در حوزه خودش می‌باشد.
ولی بهتر بود در مورد تأثیرات کارآیی این مدل در زمانی که نیازی به مرتب‌کردن اورد دسترسی به حافظه در برنامه وجود نداشته باشد
نیز صحبت‌هایی می‌شد. \\
بررسی مدل حافظه‌ای فعلی اینتل و
\lr{AMD}
با یافته‌های این مقاله می‌تواند مطالب جالبی را در دستاورد های این مقاله به ما ارائه دهد.
% \bibliography{references}  % need to put bibtex references in references.bib 
\end{document}
