\documentclass[a4paper, 11pt]{article}
\usepackage[top=2cm, bottom=3cm, left = 2cm, right = 2cm]{geometry} 
\geometry{a4paper} 
\usepackage{textcomp}
\usepackage{graphicx} 
\usepackage{amsmath,amssymb}  
\usepackage{bm}  
\usepackage{memhfixc} 
\usepackage{fancyhdr}
\usepackage{enumerate}
\usepackage{tikz}
\usepackage{float}
\usepackage{booktabs}
\usepackage{listings}
\usepackage[framed]{ntheorem}
\pagestyle{fancy}
\setlength{\headheight}{14pt}
\addtolength{\topmargin}{-2pt}
\theoremstyle{nonumberplain}
\newtheorem{definition}{Definition}

\lstset{basicstyle=\ttfamily,
  showstringspaces=false,
  commentstyle=\color{red},
  keywordstyle=\color{blue}
}

\title{HW1: Shell Scripting}
\author{Hossein Afkar}
%\date{}

\begin{document}
\maketitle
% \tableofcontents

\section{del.sh}
The first described script performs a remove of range of lines specified
in in the command inputs.
For inputs the \textit{getopts} was used and after it two args for the line
ranges is also used to isolate the lines required to show. \\
\begin{lstlisting}[language=bash]
while getopts ":i:" o; do
    case "${o}" in
        i)
            FILE=${OPTARG}
            ;;
        *)
            usage
            exit 1
            ;;
    esac
done
# Aftermath of the `getopts`
# Use shift to remove the parsed operands
shift $(($OPTIND - 1))
RANGE_1=$1
RANGE_2=$2
\end{lstlisting}
Then each line from a file is read and if the line number is between the range
it will be printed otherwise it will be not.
Range was set to be including.
\begin{lstlisting}[language=bash]
[hossein@TiD shell_advt]$ ./del.sh -i file 1 2
line1
line2
[hossein@TiD shell_advt]$
\end{lstlisting}

\section{kill.sh}
This script will kill the process by sendig a \textit{SIGTERM} to the process
by using its process id, if the process exceeds the cpu percentage and usage
time proviede to the script.
\begin{lstlisting}[language=bash]
{
    # Extra read to discard the first line
    read
    while IFS= read -r line
    do
        CPU_USAGE=$(echo $line | cut -d' ' -f3)
        PID=$(echo $line | cut -d' ' -f1)
        SECOND=$(echo $line | cut -d' ' -f2)
        if [[ $(echo "$CPU_USAGE > $CPU" | bc) -ne 0 ]]; then
            kill -9 $PID
        fi
        if [[ $(echo "$SECOND > $TIME" | bc) -ne 0 ]]; then
            kill -9 $PID
        fi
    done 
}< <(ps -eo pid,etimes,%cpu)

\end{lstlisting}
Every line of the \textit{ps} command will be read and if the cpu usage or
running time exceeds the times specified to the script it will kill the process
by sending a \textit{SIGTERM} to the program by utilizing the \textit{kill}
command.
This command is very destructive and does not have an expressable output.
\begin{lstlisting}[language=bash]
root@test:~# ./kill.sh
kill.sh -c cpu -t time
root@test:~# ./kill.sh -c 5 -t 100
Error: websocket: close 1006 (abnormal closure): unexpected EOF
[hossein@TiD ~]$
\end{lstlisting}
As you can see the ssh daemon of the remote machine has been killed by running
this command.

\section{mynet1.sh}
In this script we should extract the ip address for all the active links in
the system.
This is achieved by reading the output of the \textit{ip addr} line by line
and using \textit{awk} to extract the lines following \textit{inet}
Also grep could be used for this script which outputs whole ip addresses
without the ip block range included.

\begin{lstlisting}[language=bash]
[hossein@TiD shell_advt]$ ./mynet1.sh
127.0.0.1
192.168.1.8
192.168.1.9
10.128.152.1
[hossein@TiD shell_advt]$
\end{lstlisting}

\section{mynet2.sh}
This script changes the ip and dns server.
Changing the ip address is straight forward and the \textit{ip} command can
be used for changing the ip address but the dns server requires root access.
Changing the dns on modern systems that use systemd as their PID 0 is handled
with \textit{systemd-resolved} and the old \textit{resolv.conf} is not respected
anymore. Therefore this script checks for the root access and resovlectl being
present then proceeds with using the commands required to change the ip address
and dns server.
This command has no apparent output and due to the destructive nature of this
command it was not run on the local machine and rather in an lxc virtual
machine.

\begin{lstlisting}[language=bash]
root@test:~# ./mynet2.sh
root@test:~# ip addr
1: lo: <LOOPBACK,UP,LOWER_UP> mtu 65536 qdisc noqueue state UNKNOWN group default qlen 1000
    link/loopback 00:00:00:00:00:00 brd 00:00:00:00:00:00
    inet 127.0.0.1/8 scope host lo
       valid_lft forever preferred_lft forever
    inet6 ::1/128 scope host
       valid_lft forever preferred_lft forever
2: enp5s0: <BROADCAST,MULTICAST,UP,LOWER_UP> mtu 1500 qdisc mq state UP group default qlen 1000
    link/ether 00:16:3e:a3:b7:d3 brd ff:ff:ff:ff:ff:ff
    inet 192.168.15.0/24 scope global enp5s0
       valid_lft forever preferred_lft forever
root@test:~# resolvectl
Global
       Protocols: -LLMNR -mDNS -DNSOverTLS DNSSEC=no/unsupported
resolv.conf mode: stub

Link 2 (enp5s0)
    Current Scopes: DNS
         Protocols: +DefaultRoute +LLMNR -mDNS -DNSOverTLS DNSSEC=no/unsupported
Current DNS Server: 8.8.8.8
       DNS Servers: 8.8.8.8 10.128.152.1
        DNS Domain: lxd
\end{lstlisting}

\section{mynet3.sh}
This command increments ip address by 1 every five seconds
The script logic is presented as follow
\begin{lstlisting}[language=bash]
while sleep 5; do
    IP_ADDR=$(awk -F\. '{ print $1"."$2"."$3"."$4+1 }' <<< $IP_ADDR)
    ip addr flush $LINK$
    ip addr add "$IP_ADDR/24" dev $LINK
done
\end{lstlisting}
every five seconds the ip is incremented by 1. Note that this will not start
incrementing the ip address beyond 255 and it will fail but parsing the whole
ip address is deemed unneccessary but nevertheless it could be done using a
more complex \textit{awk} script.
This command has no explicit output but is verified that it works correctly in
a virtual machine as it is also disruptive to the workflow.

\section{mycopy}
This command tries to emulate \textit{cp} and \textit{cat} by looking at
the number of arguments passed into it.
A switch statement is called on the number of arguments and the program
logic decides which behaviour to emulate.
The whole program is consisted of read and write syscalls and for demonstration
the \textit{cat} behaviour of the program is presented as follow.
\begin{lstlisting}[language=c]
fd_from = open(argv[1], O_RDONLY);
if (fd_from < 0) {
    return EBADF;
    goto fd_err;
}
while(nread = read(fd_from, buf, sizeof(buf)), nread > 0) {
    write(STDOUT_FILENO, buf, nread);
} 
\end{lstlisting}
output of the command is also presented as follow.
\begin{lstlisting}[language=bash]
[hossein@TiD shell_advt]$ ./mycopy
Hello
Hello
World
World
^C
[hossein@TiD shell_advt]$ ./mycopy file
Hello
World
!
[hossein@TiD shell_advt]$ ./mycopy file file_dup
[hossein@TiD shell_advt]$ ./mycopy file_dup
Hello
World
!
\end{lstlisting}

\section{mylist}
This program should output the names of the files present at the specified
directory. For this we utilized the \textit{dirent} header to get the
directory entries that are present.

\begin{lstlisting}[language=c]
for (size_t i = 1; i < argc; i++) {
    d = opendir(argv[i]);
    if (d) {
        while ((dir = readdir(d)) != NULL) {
            printf("%s\n", dir->d_name);
        }
        closedir(d);
    }
}
\end{lstlisting}
Sample output is presented as follow:
\begin{lstlisting}[language=bash]
[hossein@TiD shell_advt]$ ./mylist
.
..
kill.sh
file_dup
mylist.c
mycopy
Makefile
mynet2.sh
mynet3.sh
myshell.sh
mycopy.c
mylist
del.sh
file
mynet1.sh
[hossein@TiD shell_advt]$
\end{lstlisting}
Note that the file discriptors for present directory and last directory
are also present. They could be ommited by using an if directive but it was
deemed unneccessary.

\section{myshell.sh}
In this script previous c programs must be used to craft a directory entry
counter.
\begin{lstlisting}[language=bash]
# Output to the shell
./mylist $DIR > file

# Count the number of enteries
./mycopy file | wc -l

\end{lstlisting}
This script will output mylist to the file and mycopy will read the file
and will pipe it to the \textit{wc} to count the lines present that will
represent the number of files.
\begin{lstlisting}[language=bash]
[hossein@TiD shell_advt]$ ./myshell.sh .
15
[hossein@TiD shell_advt]$ ./myshell.sh
Incorrect number of arguments
myshell.sh -i DIR
[hossein@TiD shell_advt]$ ls -a | wc -l
15
[hossein@TiD shell_advt]$
\end{lstlisting}

% \bibliographystyle{abbrv}
% \bibliography{references}  % need to put bibtex references in references.bib 
\end{document}
