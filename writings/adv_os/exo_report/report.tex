% LaTeX Template for short student reports.
% Citations should be in bibtex format and go in references.bib
\documentclass[a4paper, 12pt]{article}
\usepackage[top=2cm, bottom=3cm, left = 2cm, right = 2cm]{geometry} 
\geometry{a4paper} 
\usepackage[utf8]{inputenc}
\usepackage{textcomp}
\usepackage{graphicx} 
\usepackage{amsmath,amssymb}  
\usepackage{bm}  
\usepackage{memhfixc} 
\usepackage{fancyhdr}
\usepackage{float}
\usepackage{booktabs}
\usepackage{xepersian}
\settextfont[Scale=1.]{HM FNazli}
\setlatintextfont[Scale=.9]{Noto Sans}
\pagestyle{fancy}

\title{\lr{Exokernel: An Operating System Architecture for
Application-Level Resource Management}}
\author{حسین افکار}
%\date{}

\begin{document}
\maketitle
این مقاله یک معماری برای یک سیستم‌عامل معرفی می‌کند که در آن برنامه‌ها اجازه دارند مدیریت سخت‌افزار
را به صورت مستقیم انجام دهند. \\
معماری این مقاله از یک سری هسته‌های سبک تشکیل شده است که کمترین مقدار سرویس مورد‌نیاز
را به برنامه ارائه می‌دهند.
در این مدل هسته‌های کوچک کنار هم قرار می‌گیرند و اجازه مدیریت منابع مستقیم را به برنامه‌ها می‌دهند.
در ادامه در مورد مزایا و معایب این مدل لیست می‌شوند.
\begin{itemize}
    \item در این مدل کارآیی افزایش پیدا می‌کند به دلیل اینکه برنامه‌ها با توجه به نیاز خود
    می‌تواند منابع را مدیریت کرده که می‌تواند اورهد سیستم را کاهش ده.
    \item در این مدل سیستم‌عامل منابع با دقت بیشتری کنترل می‌شوند که باعث می‌شود هدررفت 
    منابع کمتر شود.
    \item این مدل هسته‌ها می‌توانند به گونه‌ای طراحی شوند که بر روی سخت‌افزار‌های مختلف اجرا
    شوند.
    \item برنامه‌ها می‌توانند به گونه‌ای طراحی شوند که با استفاده از کنترل بیشتر روی منابع
    از تمامی منابع در دسترس استفاده بهینه انجام دهند.
    \item پیچیدگی ارتباط میان اجزا می‌تواند باعث شود که دیزاین پیچیده شده
    و حتی شاید نتوان ارتباط میان اجرا را به صورت مؤثر برقرار کرد.
    \item ممکن است برای شماری از معماری‌های ساده پیاده‌سازی ارتباطات و دسترسی‌های
    مورد نیاز ممکن نباشد.
    \item ممکن است این دسترسی‌های مستقیم به سخت‌افزار و منابع باعث شود که در سیستم
    ایراد امنیتی ایجاد شود.
    \item چون معماری این سیستم‌عامل منحصر به فرد است مجبور هستیم که برنامه‌های مختلف را
    دوباره بسازیم
\end{itemize}

یکی از موارد اصلی که این طراحی برای ما ایجاد می‌کند این است که باید برنامه‌ها به فرمی نوشته شوند
که بتوانند از کنترل منابع استفاده مؤثر انجام دهند در غیر این صورت طراحی این سیستم‌عامل
و ارتباطات میان اجزای آن باعث ایجاد سربار در سیستم شده و پیچیدگی را افزایش می‌دهد.
همچنین دسترسی مستقیم به سخت‌افزار امنیت سیستم را کاهش می‌دهد که برای یک سری از کاربرد
ها لازم است.
به طور کلی تغییر معماری سیستم‌عامل از حالت کلاسیک به حالت مدرن نیاز به در نظر گرفتن موارد زیادی
دارد که گذار از مدل کلاسیک به مدل مدرن را بسیار سخت می‌کند.
همچنین نوشتن لایه‌ها میانی برای این‌گونه سیستم‌عامل‌ها مانند کتابخانه‌های
زبان
\lr{C}
و نرم‌افزار‌های مختلف سیستمی هزینه‌ زیادی را به ما تحمیل می‌کند.
به همین دلیل استفاده کامل از مدل به نظر بیهوده می‌آید و بهتر است فقط مواردی از ایده‌ها مانند
ریزکرنل‌ها را از این سیستم برداریم و سعی کنیم سیستم را طوری طراحی کنیم که با نرم‌افزار‌های فعلی سازگار باشد
که در این صورت یکی از فاکتور‌های اصلی طراحی این سیستم که افزایش کارآیی و بهبود استفاده از سخت‌افزار می‌باشد
را از دست می‌دهیم که انجام این کار را کم فایده می‌کند.

% \bibliography{references}  % need to put bibtex references in references.bib 
\end{document}
