% LaTeX Template for short student reports.
% Citations should be in bibtex format and go in references.bib
\documentclass[a4paper, 12pt]{article}
\usepackage[top=2cm, bottom=3cm, left = 2cm, right = 2cm]{geometry} 
\geometry{a4paper} 
\usepackage[utf8]{inputenc}
\usepackage{textcomp}
\usepackage{graphicx} 
\usepackage{amsmath,amssymb}  
\usepackage{bm}  
\usepackage{memhfixc} 
\usepackage{fancyhdr}
\usepackage{float}
\usepackage{booktabs}
\usepackage{xepersian}
\settextfont[Scale=1.]{HM FNazli}
\setlatintextfont[Scale=.9]{Noto Sans}
\pagestyle{fancy}

\title{\lr{Xen and the Art of Virtualization}}
\author{حسین افکار}
%\date{}

\begin{document}
\maketitle
این مقاله سیستم مجازی سازی
\lr{Xen}
را معرفی می‌کند.
در آن زمان هدف این بوده است که سرباز مجازی‌سازی کم شده همچنین با پیشرفت‌های سخت‌افزاری انجام
شده بتوانیم تعداد زیادی ماشین مجازی را بر روی یک سیستم اجرا کنیم.
یکی از ویژگی‌های مهم این روش این است که برای اجرا نیاز به ادیت کردن سورس سیستم‌عامل دارد
که به این معنی است که سیستم‌عامل‌ها به طور پیش‌فرض نمی‌توانند توسط این سیستم اجرا شوند.
در این معماری مانیتور ماشین مجازی از هسته سیستم‌عامل جدا می شود که این
مانیتور در مجازی‌سازی‌های معمول یک رابط برای برقراری ارتباط بین هسته سیستم‌عامل و
سخت‌افزار‌های معمول را فراهم می‌کند. \\
روشی که ابزار
\lr{Xen}
معرفی می‌کند به نام
\lr{Paravirtualization}
معروف است که به این معنی است که هسته سیستم‌عامل باید از وجود مجازی‌ساز آگاه باشد.
همچنین در این روش سیستم کال‌ها به هایپر کال‌ها تغییر کاربری می‌دهند به این گونه
سربار تغییر متن سنگین بین سطح کاربر به سیستم‌عامل و مجازی‌ساز از بین می‌رود. \\
همچنین این مقاله مفهوم
\lr{Domain}
را مطرح می‌کند که در هر
\lr{Domain}
یک سیستم‌عامل همراه با برنامه‌های سطح کاربر خود اجرا می‌شود.
هایپروایزور تعیین می‌کند که چقدر پردازنده و حافظه به دامنه اختصاص پیدا کند که این کار به مجازی‌ساز کمک می‌کند
تا اجرای سیستم‌ها را از یکدیگر از هم مستقل کند.
به طور کلی روش این مقاله بر بهتر کردن کارآیی متمرکز شده است که اثر آن تعریف دو
مفهوم بالا است که در مجازی‌ساز‌های صنعتی و آکادمی نیز استفاده‌های زیادی دارند. \\
برای سنجش نتایج نیاز بود که این مقاله سیستم‌عامل‌های موجود را برای این کار شخصی سازی کند.
که برای لینوکس سورس دست‌خوش تغییر قرار گرفت و برای ویندوز
\lr{XP}
درایور‌ها تغییراتی پیدا کرده تا بتوانند با مانیتور مجازی‌ساز ارتباط برقرار کنند. \\
ایرادی که به این مقاله وارد است این است که بنچ‌مارک‌های تعبیه شده در این مقاله خیلی عمومی
می‌باشند و برای همین تعمیم دادن آن به نتایج واقعی کمی سخت می‌شود
به طور کلی این مقاله دست‌آورد‌های خوبی داشته است که شامل موارد زیر می‌شود:
\begin{itemize}
    \item تعریف مفهوم
    \lr{Paravirtualization}
    که باعث شد هسته سیستم‌عامل با مجازی‌ساز به طور مستقیم در ارتباط باشد.
    \item تعریف مفهوم دامنه که باعث دست‌یابی به استقلال و امنیت بین ماشین‌های مجازی شد
    \item پیاده‌سازی مفاهیم موجود به عنوان هایپروایزور لول یک که در حال حاضر نیز بعد از ۲۰ سال
    در حال استفاده است.
    \item نشان دادن این موضوع که چگونه این مدل از طراحی به طرز بسیار خوبی کارآیی را بهبود می‌بخشد
    \item این مجازی‌ساز و مفاهیم مطرح شده در آن الهام‌بخش بسیازی از مجازی‌ساز های صنعتی
    و متن‌باز دیگر بوده است.
\end{itemize}
% \bibliography{references}  % need to put bibtex references in references.bib 
\end{document}
