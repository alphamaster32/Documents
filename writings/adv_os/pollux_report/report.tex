% LaTeX Template for short student reports.
% Citations should be in bibtex format and go in references.bib
\documentclass[a4paper, 12pt]{article}
\usepackage[top=2cm, bottom=3cm, left = 2cm, right = 2cm]{geometry} 
\geometry{a4paper} 
\usepackage[utf8]{inputenc}
\usepackage{textcomp}
\usepackage{graphicx} 
\usepackage{amsmath,amssymb}  
\usepackage{bm}  
\usepackage{memhfixc} 
\usepackage{fancyhdr}
\usepackage{float}
\usepackage{booktabs}
\usepackage{xepersian}
\settextfont[Scale=1.]{HM FNazli}
\setlatintextfont[Scale=.9]{Noto Sans}
\pagestyle{fancy}

\title{\lr{Pollux: Co-adaptive Cluster Scheduling
for Goodput-Optimized Deep Learning paper}}
\author{حسین افکار}
%\date{}

\begin{document}
\maketitle
این مقاله یک روش زمان‌بندی برای یک
\lr{cluster}
یادگیری عمیق ارائه می‌دهد. هدف این ارائه این است که مقدار کار مؤثر در واحد زمان را بیشینه
کند. نویسندگان این مقاله معتقدند که زمان‌بند‌های فعلی برای سیستم‌های یادگیری عمیق مناسب نیستند.
این سیستم‌ها به دلیل پیچیدگی داده‌ها و سربار ارتباطی نیازمند زمان‌بند ویژه‌ای هستند که در این مقاله
به آن پرداخته می‌شود. \\
این مقاله طراحی و پیاده‌سازی زمان‌بندی به نام
\lr{Pollux}
را شرح می‌دهد.
این زمان‌بند نیاز‌های ارتباطی و محاسباتی تسک‌های سیستم‌های یادگیری عمیق را نیز در نظر می‌گیرد.
زمان‌بند
\lr{Pollux}
یک روش جدید را برای زمان‌بندی انتخای می‌کند که در این روش جدید منابع به صورت پویا بر اساس
مدل تسک‌ها و شرایط سیستم اختصاص می‌یابند.
در این مقاله الگوریتم‌ها و معماری این زمان‌بند توضیح داده شده‌است. \\
در ادامه این زمان‌بند بر روی تسک‌های یادگیری عمیق تست شده و نشان داده می‌شود
که زمان‌بند پیشنهادی از زمان‌بند‌های فعلی بر اساس معیار مقدار کار مؤثر در واحد زمان
بهینه‌تر عمل می‌کند. در این مقاله یک آنالیز دقیق بر روی کارآیی این زمان‌بند انجام شده است. \\
این مقاله یک
\lr{contribution}
ارزشمند به حوزه زمان‌بندی
\lr{cluster}
ها در حوزه یادگیری عمیق انجام داده‌است.
ولی با این وجود محدودیت‌ها و مفروضاتی نیز وجود دارند که باید به آن‌ها توجه شوند.
برای مثال سنجش نتایج توسط یک مدل
\lr{workload}
و تنظیمات سیستمی از پیش تعیین شده انجام شده است.
همچنین این مقاله به صورت پیش‌فرض در نظر می‌گیرد که در تسک‌های یادگیری عمیق همواره
ارتباط بین تسک‌ها
\lr{bottleneck}
است که به صورت پیش‌فرض درست نیست.
پیاده‌سازی این زمان‌بند نیازمند این است که سیستم‌های فعلی دست‌خوش تغییرات زیادی شوند که ممکن است
باعث شود این زمان‌بند در سیستم‌های موجود استفاده نشود.
همچنین در ادامه بررسی این‌که اثر این زمان‌بند بر روی
\lr{determinism}
این سیستم چگونه است نیز می‌تواند انجام شود. \\
در آخر این مقاله که در کنفرانس معروف
\lr{usenix}
چاپ شده است یک سیستم با طراحی خوب را در اختیار ما می‌گذارد که ما را از کمبود‌های
سیستم‌های زمان‌بند فعلی آگاه می‌کند.
سنجش نتایج این مقاله به خوبی انجام شده است که البته باید پیش‌فرض‌ها را در تحلیل این نتایج
در نظر بگیریم.
در ادامه نقاط ضعف و قوت این مقاله به صورت زیر ارزیابی می‌شوند
\begin{itemize}
    \item مقاله دارای هدف خوب و راه‌حل خوب برای سیستم‌های یادگیری عمیق است
    \item روش تنظیم منابع به صورت پویا نوآوری دارد
    \item سنجش نتایج مقاله به خوبی انجام شده است
    \item سنجش مقاله بر اساس پیش‌فرض‌هایی مانند بزرگ بودن سربار ارتباطی انجام شده است
    \item این مقاله بین تفاوت متریک پیشنهادی که بر اساس کار مفید بر اساس واحد زمان است
    و دیگر متریک‌ها صحبتی نمی‌کند
\end{itemize}


% \bibliography{references}  % need to put bibtex references in references.bib 
\end{document}
