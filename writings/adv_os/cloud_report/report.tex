% LaTeX Template for short student reports.
% Citations should be in bibtex format and go in references.bib
\documentclass[a4paper, 12pt]{article}
\usepackage[top=2cm, bottom=3cm, left = 2cm, right = 2cm]{geometry} 
\geometry{a4paper} 
\usepackage[utf8]{inputenc}
\usepackage{textcomp}
\usepackage{graphicx} 
\usepackage{amsmath,amssymb}  
\usepackage{bm}  
\usepackage{memhfixc} 
\usepackage{fancyhdr}
\usepackage{float}
\usepackage{booktabs}
\usepackage{xepersian}
\settextfont[Scale=1.]{HM FNazli}
\setlatintextfont[Scale=.9]{Noto Sans}
\pagestyle{fancy}

\title{\lr{Above the Clouds: A Berkeley View Of Cloud  Computing}}
\author{حسین افکار}
%\date{}

\begin{document}
\maketitle
این مقاله به پیدایش تکنولوژی
\lr{Cloud Computing}
می‌پردازد. کلاد یعنی دادن یک کاربرد کامپیوتری به ازای یک سرویس. به عنوان مثال دادن
امکانات سخت‌افزاری به عنوان سرویس به عنوان
\lr{IaaS: Infrastructure as a Service}
بین سرویس‌دهنده‌های کلاد رواج دارد.
کلاد خوبی‌هایی دارد که از آن‌ها می‌توان به توهم منابع بی‌نهایت، پرداخت فقط برای منابع استفاده شده،
و نبودن نیاز به قرارداد‌های تجاری با منابع زیاد اشاره کرد. \\
مدلی که آمازون در سرویس‌های
\lr{EC2}
خود ارائه می‌دهد دسترسی‌های سطح سخت‌افزار بالایی به ما می‌دهد ولی در این کنار نمی‌تواند از ما
در برابر خطا ها حفاظت کند و عمل
\lr{Scalability}
را انجام نمی‌دهد. در این کنار
\lr{AppEngine}
شرکت
\lr{Google}
دسترسی‌های سطح بالا فقط برای اجرا کردن برنامه‌های وب را به ما می‌دهد ولی در این کنار
\lr{Abstraction}
های سخت‌افزار را حذف کرده و به ما تحمل‌پذیری خطا و اسکیل شدن را می‌دهد. \\
مشکلاتی که در کلاد مطرح هستند در این مقاله به سه دسته پذیرش این تکنولوژی، رشد این تکنولوژی، و سیاست‌های
شرکت‌ها و پذیرش آن‌ها تقسیم می‌شود. \\
در آخر این مقاله مطرح می‌کند که سیاست‌های فعلی شرکت‌ها و شرایط فعلی بازار که فرصت‌های برنامه‌های جدید را فراهم کرده باعث شده است
که کلاد خیلی پیشرفت کند و فرصت استفاده از تکنولوژی‌های جدید را برای ما فراهم کند. \\
در بخش‌های آتی این مقاله مواردی نیز در مورد مدل اقتصادی این سیستم مطرح می‌شود.
این بخش عباراتی مانند
\lr{CAPEX}
و‍‍
\lr{OPEX}
مطرح می‌کند که به ما می‌فهماند که بحث این قسمت در مورد مدل اقتصادی کلاد است. \\
یکی از مفاهیمی که در این بخش مطرح می‌شود مفهوم
\lr{Elasticity}
است که ارتباط مستقیمی با مدل
\lr{Pay as You Go}
دارد. تعریف
\lr{Elasticity}
توانایی اضافه کردن و یا کم کردن منابع به صورت
\lr{Fine Grain}
می‌باشد. این قابلیت به کاربران کمک می‌کند که در زمان پیش‌بینی منابع به دلیل ذات
نوسانی نیاز به منابع، تخصیص دادن منابع را به شکل بهتری انجام بدهند و
همچنین مفهوم
\lr{Elasticity}
به ما کمک می‌کند که در بازه‌های زمانی کوتاه منابع را به سیستم اضافه و کم کنیم. \\
در آخر این مقاله به این سوال پاسخ می‌دهد که آیا باید برنامه‌های خود را به کلاد منتقل کنیم یا خیر.
به طور کلی این سوال در بخش بستگی دارد دسته‌بندی می‌شود اما این مقاله مثال‌هایی را برای سنجش مطرح
کرده است. به طور کلی قیمت‌های استفاده از کلاد به طور صوری بیشتر هستند
ولی به خاطر وجود مفهوم
\lr{Elasticity}
ارزش هر دلار مصرف شده در کلاد از دیتا‌سنتر‌‌های عادی بیشتر می‌باشد. \\
در بخش آخر چالش‌ها و فرصت‌های
\lr{Cloud Computing}
مطرح شده است که این بخش چالش‌هایی را برای استفاده از این سیستم‌ها مطرح می‌کند.
چالش‌های مطرح شده از قبیل
\lr{Data Lock In}
و
\lr{Performance Unpredictability}
به نظر نویسنده این گزارش مهم‌ترین چالش‌های این صنعت می‌باشند همانطوری که در ابتدای این مقاله
از طرف
\lr{Richard Stallman}
مورد اول آن مطرح شده است.
مورد دوم به خاطر اینکه سازمان کلاد و مدل آن در ارائه‌دهنده‌های تجاری
غیر قابل دسترس است باعث می‌شود که دید مناسبی برای بهینه‌سازی برنامه‌ها و یافتن ایراد‌ها نداشته باشیم. \\
در آخر این مقاله در معرفی کردن تکنولوژی کلاد در سال ۲۰۰۹ موفق بوده و چالش‌ها و فرصت‌ها
اصلی این تکنولوژی را معرفی کرده است.
هدف این مقاله یک بررسی تکنیکال نبوده است و صرفاً سعی کرده است یک تعریف برای
\lr{Cloud Computing}
در آن سال‌ها ارائه دهد و مشخصات و چالش‌ها و فرصت‌های آن را بررسی کنند.
در آخر این مقاله به تولید‌کنندگان نرم‌افزار پیشنهاد می‌کند تا نرم‌افزار‌های خود را برای
نسل جدید سیستم‌ها که بر پایه کلاد هستند طراحی کنند که این توصیه نشان می‌دهد
که در ۱۴ سال پیش چه دید مناسبی را داشته‌اند.
% \bibliography{references}  % need to put bibtex references in references.bib 
\end{document}
