% LaTeX Template for short student reports.
% Citations should be in bibtex format and go in references.bib
\documentclass[a4paper, 12pt]{article}
\usepackage[top=2cm, bottom=3cm, left = 2cm, right = 2cm]{geometry} 
\geometry{a4paper} 
\usepackage[utf8]{inputenc}
\usepackage{textcomp}
\usepackage{graphicx} 
\usepackage{amsmath,amssymb}  
\usepackage{bm}  
\usepackage{memhfixc} 
\usepackage{fancyhdr}
\usepackage{float}
\usepackage{booktabs}
\usepackage{xepersian}
\settextfont[Scale=1.]{HM FNazli}
\setlatintextfont[Scale=.9]{Noto Sans}
\pagestyle{fancy}

\title{\lr{Live Migration Of Virtual Machines}}
\author{حسین افکار}
%\date{}

\begin{document}
\maketitle
این مقاله در مورد تکنیک
\lr{Migration}
بحث می‌کند.
وجود امکان جابه‌جایی ماشین‌های مجازی بین هاست‌های فیزیکی به ما کمک می‌کند تا بتوانیم
ضریب اطمینان و تحمل خرابی سیستم را افزایش دهیم.
این مقاله سعی می‌کند تا طراحی‌ای ارائه دهد که بتواند ماشین‌های مجازی روشن را بین هاست‌های فیزیکی
انتقال دهد.
همچنین ایده
\lr{Live Migration}
به ما کمک می‌کند که در سطح سیستم بتوانیم از ایده
\lr{Load Balancing}
استفاده کنیم.
پیاده‌سازی این مقاله بر روی مانیتور هایپروایزور
\lr{Xen}
انجام شده است. \\
ایده اصلی این مقاله‌ این است که یک طراحی برای انتقال حافظه مجازی ماشین مجازی ارائه دهد.
این کار مسائل متعددی دارد که باید به آن‌ها توجه شود که در مقاله قید شده‌اند.
چالش اصلی این مقاله این است که چگونه حافظه را به هاست دیگر انتقال دهد. \\
همچنین یک بخش اصلی این مقاله در مورد این است که مسائلی مانند دیوایس‌ها چگونه باید انتقال یابند.
نکات ضعف و قوت این مقاله به فرم زیر هستند
\begin{itemize}
    \item این مقاله‌ راه‌حلی برای یک مشکل بسیار مهم در این حوزه ارائه کرده است که برای
    اولین بار در این زمینه مطرح شده بود.
    \item روشی که این مقاله معرفی کرده در بسیاری از هایپروایزور‌های معروف از جمله
    \lr{Zen}
    و
    \lr{ESXi}
    استفاده می‌شود.
    \item روشی که این مقاله معرفی می‌کند توجه ویژه‌ای به حافظه و مسائل جزئی می‌کند
    که پیاده‌سازی را ساده می‌کند.
    \item انتقال ماشین‌های مجازی طبق داده‌های مقاله سرباری را در حین انتقال به ماشین‌مجازی وارد می‌کند.
    \item هنگام انتقال داده‌ها بین هاست‌ها ممکن است امنیت داده‌ها کاهش یابد.
    \item انتقال ماشین‌مجازی برای کاربرد‌های بی‌درنگ یا ریل‌تایم نمی‌تواند استفاده شود.
\end{itemize}
طبق موارد مطرح شده در بخش نتایج این مقاله، انتقال ماشین مجازی با خاموشی ۶۰ میلی‌ثانیه ممکن بوده است.
همچنین انتقال ماشین‌‌مجازی ملزم به انتقال دیسک است که این مقاله فقط به سربار انتقال مموری‌ مجازی پرداخته است
که همه موضوع نیست و ممکن است انتقال دیسک زمان بسیاری را صرف کند و مسائل جزئی زیادی داشته باشد
که این مقاله تصمیم گرفته است از آن صرف‌نظر کند و در نظر بگیرد که دیسک در آرایه‌های
کپی یا فایل‌سیستم‌های توزیع شده قرار دارد. البته این مقاله در بخش کار‌های آتی به این موضوع اشاره کرده است. \\
در پایان این مقاله روش‌های بسیار خوبی را برای مواجهه با مشکل انتقال بین‌ هاست‌های مختلف مطرح کرده است
که در زمان‌ خود روش ‌بسیار انقلابی و کاربردی محسوب می‌شود و راه‌کار‌های ارائه شده تاکنون نیز در
هایپروایزور‌ها استفاده می‌شوند.
% \bibliography{references}  % need to put bibtex references in references.bib 
\end{document}
