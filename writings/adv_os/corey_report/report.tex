% LaTeX Template for short student reports.
% Citations should be in bibtex format and go in references.bib
\documentclass[a4paper, 12pt]{article}
\usepackage[top=2cm, bottom=3cm, left = 2cm, right = 2cm]{geometry} 
\geometry{a4paper} 
\usepackage[utf8]{inputenc}
\usepackage{textcomp}
\usepackage{graphicx} 
\usepackage{amsmath,amssymb}  
\usepackage{bm}  
\usepackage{memhfixc} 
\usepackage{fancyhdr}
\usepackage{float}
\usepackage{booktabs}
\usepackage{xepersian}
\settextfont[Scale=1.]{HM FNazli}
\setlatintextfont[Scale=.9]{Noto Sans}
\pagestyle{fancy}

\title{\lr{Corey: An Operating System for Many Cores}}
\author{حسین افکار}
%\date{}

\begin{document}
\maketitle
این مقاله در مورد طراحی و ساخت یک سیستم‌عامل بحث می‌کند که این سیستم‌عامل
برای سیستم‌های چند هسته‌ای بهینه شده است.
ایده اصلی این مقاله این است که یک سیستم‌عامل برای پردازنده‌های چند هسته‌ای نیازی ندارد که
خود را با برنامه‌های قدیمی هماهنگ کند و می‌توان برای برنامه‌های جدید رابطی را طراحی کرد
که کارآیی را افزایش می‌دهد.
همچنین این سیستم‌عامل کنترل به اشتراک گذاری را به نرم‌افزار سطح کاربر واگذار می‌کند
که این موضوع باعث می‌شود سیستم‌عامل برای برنامه‌هایی که به اشتراک گذاری نیازی ندارند،
تبدیل به
\lr{bottleneck}
نشود. \\
روشی که این مقاله اتخاذ کرده است بر پایه این است که یک طراحی و پیاده‌سازی برای این سیستم‌عامل
به نام
\lr{Corey}
انجام دهد و همچنین کارآیی آن را در سناریو‌های مختلف بسنجد و مؤثر بودن آن‌ را بررسی کند.
همچنین در آخر این مقاله مشکلات طراحی را مطرح کرده که نقشه کار را برای کار‌های آتی مشخص می‌کند \\
نتایج نشان می‌دهد که
\lr{Corey}
از سیستم‌عامل‌های فعلی در دو بعد کارآیی و
\lr{Scalabiliy}
بهتر عمل می‌کند.
نتایج بعد‌های مختلف این سیستم‌عامل را بررسی می‌کنند که اولین نتیجه در مورد مدیریت حافظه است.
در ادامه نتایج انتزاع این سیستم‌عامل از هسته‌ها توسط یک میکروبنچ‌مارک
\lr{TCP}
بررسی می‌گردند.
میکروبنچ‌مارک بعدی در مورد مکانیزم اشتراکی است که این سیستم‌عامل پیاده‌سازی کرده است.
در آخر دو برنامه
\lr{MapReduce}
و
\lr{Webd}
بررسی شده‌اند که نشان می‌دهند این سیستم‌عامل در مقایسه با لینوکس می‌تواند بهبود کارآیی 
۲۰ درصدی
در تعداد هسته‌های بالا را داشته باشد.

در آخر نتایج این مقاله نشان می‌دهد که سیستم‌عامل
\lr{Corey}
یک سیستم‌عامل با طراحی و پیاده‌سازی مناسب است که برای سیستم‌های چند هسته‌ای
بهینه شده است و همچنین کارآیی را نسبت به سیستم‌عامل‌های دیگر بهبود می‌بخشد.
در آخر موارد زیر برای کاستی‌های این طراحی مطرح می‌شود.
\begin{itemize}
    \item این طراحی فقط برای سیستم‌های چند هسته‌ای ارائه شده است
    که این موضوع باعث می‌شود که بهبود‌های مدیریت حافظه برای سیستم‌های تک هسته‌ای مطرح نشود.
    \item استفاده از این سیستم‌عامل به این نیاز دارد که از کتاب‌خانه‌ها و فراخوانی‌های سطح سیستم
    مخصوصی استفاده بشود که استفاده از نرم‌افزار‌های فعلی را سخت می‌کند.
    \item این طراحی نمی‌تواند با سیستم‌عامل‌های عمومی رقابت کند
    هر چند که از نتایج این طراحی می‌توان در بهبود طراحی‌های فعلی استفاده کرد.
\end{itemize}
به طور کلی این مقاله راه‌حل‌ها و چالش‌های سیستم‌های چند هسته‌ای را مطرح می‌کند که این موضوع باعث می‌شود
که این سیستم‌ها بهتر شناخته شوند.
همچنین با استفاده از نتایج بدست‌ آورده شده در این مقاله نتیجه می‌گیریم که برای استفاده
بهینه از منابع سیستم‌های چند هسته‌ای بهتر از رابط نرم‌افزاری جدیدی برای سیستم طراحی کنیم
تا نرم‌افزار‌های جدید بتوانند از تعداد هسته‌های متعدد سیستم استفاده بهینه را داشته باشند.

% \bibliography{references}  % need to put bibtex references in references.bib 
\end{document}
