% LaTeX Template for short student reports.
% Citations should be in bibtex format and go in references.bib
\documentclass[a4paper, 11pt]{article}
\usepackage[top=2cm, bottom=3cm, left = 2cm, right = 2cm]{geometry} 
\geometry{a4paper} 
\usepackage{textcomp}
\usepackage{graphicx} 
\usepackage{amsmath,amssymb}  
\usepackage{bm}  
\usepackage{memhfixc} 
\usepackage{fancyhdr}
\usepackage{float}
\usepackage{booktabs}
\usepackage{xepersian}
\settextfont[Scale=1.]{HM FNazli}
\setlatintextfont[Scale=.9]{Noto Sans}
\pagestyle{fancy}

\title{\lr{Improving IPC by Kernel Design}}
\author{حسین افکار}
%\date{}

\begin{document}
\maketitle
در این مقاله به کارآیی روش‌های ارتباط میان پروسه‌ای پرداخته شده است
که طبق گفته این مقاله می‌توان از ۱۰۰ میکرو‌ثانیه با طراحی پیشنهادی به
۵ میکروثانیه رسید \\
در این مقاله در مورد میکروکرنل
\lr{L3}
بحث می‌شود که یک ماشینی است بر پایه تسک کار می‌کند که در اینجا هر تسک یک
ترد است.
ارتباط میان پردازه‌ای در این کرنل بسیار ساده است به طوری که
میان تسک‌ها از تبادل پیام استفاده می‌شود که به صورت یک رشته
و/یا شی درون مموری عمل می‌کند.
برای اینکه یک طراحی بهتر ارائه شود این مقاله سعی کرده است که به این مشکل از یک زاویه دیگری
نگاه کند
که محور اصلی این طراحی جدید این است که کارآیی ارتباط میان پردازه‌ای مهم‌ترین بخش است
و همچنین ارسال پیغام میان پردازه‌ها از سمت یوزر انجام شود.
که این بخش بسیار مهم باعث می‌شود که هر چیزی قربانی شود تا کارآیی بهینه بماند.
در ادامه با استفاده از تکنیک‌های زیر این موضوع تحقق پیدا کرده است
این تکنیک‌ها به سه دسته رابطی، الگوریتمی، و معماری تقسیم می‌شوند.
\begin{itemize}
    \item اضافه کردن فراخوانی‌های سیستمی جدید
    \item تقارن بافر‌های دریافت و ارسال
    \item کپی کردن بافر فقط یک بار با استفاده از استک کرنل
    \item اضافه کردن کنترل بلاک در حافظه مجازی
    \item بهینه کردن سازمان رشته‌ها
    \item بیرون انداختن بلاک کنترل ترد از صف موقع \lr{umap} کردن
    \item بهینه‌سازی حسابداری تایم اوت
    \item زمان‌بندی به صورت \lr{Lazy}
    \item سوویچ کردن پروسه به صورت مستقیم
    \item ارسال پیغام‌های کوتاه از طریق رجیستر
    \item کاهش تعداد میس در کش
    \item کاهش تعداد میس در \lr{TLB}
    \item استفاده بهینه از سگمنت رجیستر
    \item استفاده بهینه از رجیستر‌های عمومی
    \item خودداری از پرش و چک کردن پرش
    \item کم کردن سربار سوییچ پروسه
\end{itemize}
طراحی که توسط این مقاله پیشنهاد شده است باعث شده است که درگیری سطح یوزر و کرنل
کم شود که باعث می‌شود کارآیی افزایش یابد
ولی پیاده‌سازی این طراحی نیاز‌مند این است که این طراحی نرم‌افزار‌های فعلی
و طراحی سیستم‌عامل به صورت کلی عوض شود.
و همچنین ممکن است مکانیزم فعلی برای تمامی نرم‌افزار‌ها بهینه نباشد و الزامات لازم
مورد نظر استاندارد‌های مختلف را به ما ارائه نکند. \\
در کل این مقاله روند افزایش
\lr{Performance}
را در یک سیستم به صورت خیلی کامل بررسی کرده بود و به ما نشان داد که برای افزایش
کارآیی یک سیستم نیازی نیست که از دادن تغییرات حتی در
سطوح پایین طراحی سیستم بترسیم و همچنین نشان داد که شجاعت در انجام تغییرات می‌تواند
کارآیی را به شکل خیلی زیادی بهبود بخشد.
% \bibliography{references}  % need to put bibtex references in references.bib 
\end{document}
