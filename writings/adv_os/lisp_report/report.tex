% LaTeX Template for short student reports.
% Citations should be in bibtex format and go in references.bib
\documentclass[a4paper, 12pt]{article}
\usepackage[top=2cm, bottom=3cm, left = 2cm, right = 2cm]{geometry} 
\geometry{a4paper} 
\usepackage[utf8]{inputenc}
\usepackage{textcomp}
\usepackage{graphicx} 
\usepackage{amsmath,amssymb}  
\usepackage{bm}  
\usepackage{memhfixc} 
\usepackage{fancyhdr}
\usepackage{float}
\usepackage{booktabs}
\usepackage{xepersian}
\settextfont[Scale=1.]{HM FNazli}
\setlatintextfont[Scale=.9]{Noto Sans}
\pagestyle{fancy}

\title{\lr{Lisp: Good News, Bad News, How to Win Big}}
\author{حسین افکار}
%\date{}

\begin{document}
\maketitle
در این مقاله به موفقیت‌ها شکست‌ها و روند رو به جلوی می‌پردازیم. در صنعت
\lr{Lisp}
نتوانسته است به خوبی عمل کند که این مقاله نقات قوت و نقات ضعف عملکری این زبان را بیان می‌کند.
\lr{Lisp}
در بخش استانداردسازی خوب عمل کرده و توانسته است استاندارد‌های متعددی را در بر بگیرد که
\lr{Common Lisp}
در این زمینه پیش‌تاز می‌باشد. همچنین در بخش کارآیی زبان بسیار خوب عمل شده و ابزار‌های بسیار
کاربردی برای این زبان نوشته شده‌است که بسیاری از ابزار‌های معمول زبان‌ها ریشه در محیط
\lr{Lisp}
دارند.
در این مقاله به یکی از اشتباه‌های این زبان اشاره شده است که بحث
\lr{The Right Thing}
است که باعث شده بسیاری از انرژی طراحان این زبان صرف موضوعاتی بشود که به پیشرفت کارآیی
این زبان کمکی نکند.
این مقاله نسبت به ایده
\lr{Worse is Better}
بیان مثبت‌تری دارد و دلیل پیشرفت
\lr{Unix}
و
\lr{C}
را این عنوان می‌کند که این تکنولوژی‌ها خود را درگیر مسائلی نکردند که باعث شود سیستم پیچیده
تحویل داده شود.
برای همین دو سناریو مطرح شد که به شرح زیر می‌باشند

\begin{itemize}
    \item \lr{big complex system scenario}
    \item \lr{diamond-like jewel scenario}
\end{itemize}

که سناریو اول بیان می‌کند که ابتدا باید طرح سیستم به شکل درست طراحی شود و سپس
پیاده‌سازی باید طراحی شود و در آخر پیاده‌سازی شود.
به خاطر اینکه این سیستم درست است ۱۰۰ درصد کارآیی مورد نظر را دارد ولی همیشه بیست درصد
پایانی این سیستم سخت خواهد بود و هشتاد درصد توان طراحان را می‌گیرد

سناریو دوم بیان می‌کند که طراحی سیستم درست بسیار طولانی است ولی در ابتدا بسیار کوچک
می‌باشد. ولی پیاده‌سازی سریع این سیستم بیشتر از توان هر طراح می‌باشد.
درسی که از این سیستم می‌توان گرفت این است که بهتر است سراغ چیز درست رفت و پنجاه درصد
آن را ساخت و در دسترس گذاشت تا مانند یونیکس گسترش پیدا کند. \\

مورد مهم دیگر این است که محیط‌های دیگر برنامه‌نویسی خیلی بهتر شده‌اند و به ابزار‌هایی که
\lr{Lisp}
در گذشته به آن‌ها دسترسی داشته است دست پیدا کرده‌اند
برای همین این مقاله مفهوم ویروس را مطرح کرده و می‌گوید بهتر است کار‌هایی که انجام می‌شوند
در سطح بیولوژیکی ویروس منتشر شوند و تبدیل به یک موجود پیچیده را به مراحل بعدی واگذار کنند.

همچنین در ادامه به این پرداخته شد که چگونه
\lr{Lisp}
می‌تواند موفقیت کسب کند.
یکی از نقات قوت
\lr{Lisp}
این بود که محیط قدرت‌مندی برای برنامه‌نویسی داشت که باید برای بقیه محیط‌ها ساخته و تقویت شود
موضوع اصلی این است که این زبان مرکز دنیا نیست و باید به گونه‌ای کار شود که بتواند با بقیه دنیای
سایبری تعامل کند.
در ادامه در مورد این صحبت شد که زبان بعدی به چه شکل می‌تواند باشد که اهمیت چندانی برای
خوانندگانی که با این زبان کار نمی‌کنند ندارد. \\

مهم‌ترین نتیجه‌ای که در حین این مقاله گرفته شد این بود که ساخت چیز صحیح و اصرار بر روی اینکه
تمامی موارد مربوط به سیستم به درستی مدل شوند ممکن است باعث شود که یک سیستم شکست بخورد
و با وجود اینکه شروع خوبی داشته‌است بازی را به ویروس‌های موجود در محیط ببازد.
برای همین موقع طراحی باید اصول
\lr{worse is better}
رعایت شوند و به سادگی پیاده‌سازی توجه ویژه‌ای شود.

% \bibliography{references}  % need to put bibtex references in references.bib 
\end{document}
