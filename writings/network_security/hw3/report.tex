\documentclass[a4paper, 11pt]{article}
\usepackage[top=2cm, bottom=3cm, left = 2cm, right = 2cm]{geometry} 
\geometry{a4paper} 
\usepackage{textcomp}
\usepackage{graphicx} 
\usepackage{amsmath,amssymb}  
\usepackage{bm}  
\usepackage{memhfixc} 
\usepackage{fancyhdr}
\usepackage{enumerate}
\usepackage{float}
\usepackage{booktabs}
\usepackage{listings}
\usepackage{mathtools}
\pagestyle{fancy}
\setlength{\headheight}{14pt}
\addtolength{\topmargin}{-2pt}

\title{Advanced Network Security: Assignment \#3}
\author{Hossein Afkar \\ 810101102}
%\date{}

\begin{document}
\maketitle
% \tableofcontents
\section{Q1}
\subsection{a}
By referring to section 13.3 of Newman's book we can find out how
this equation was derived. \\
Suppose that we have $m$ as the number of edges. Given the fact that $p_k$ is
the total fraction of vertices in the network with the degree k the total number
of such vertices is $np_{k}$. Also, stubs are half-edges in the configuration model
so according to Newman's book the probability of our edge ending at any
particular vertex of degree $k$ is $\frac{k}{2m-1}$ which we can assume as
$\frac{k}{2m}$. Therefore the probability of an edge attaching to
any vertices with degree $k$ is:
\begin{equation}
    \frac{k}{2m} \times np_k = \frac{kp_k}{\langle k \rangle^{\prime}}
\end{equation}
Where $\langle k \rangle$ is the average degree over the whole network.
There is proof that $2m = n \langle k \rangle$ in section 6 of the book. \\
By multiplying this probability by $k$ and summing it over all of the vertices
we have:
\begin{equation}
    \text{average degree of a neighbor} =
    \sum_{k}k\frac{kp_k}{\langle k \rangle} =
    \frac{\langle k^2 \rangle}{\langle k \rangle}
\end{equation}
We can calculate the probability distribution of the excess degree from the
above equation. The probability $q_k$ of having excess degree $k$ is the
probability of having total degree $k+1$ and putting $k \rightarrow k + 1$.
So we get
\begin{equation}
    q_k = \frac{(k+1)p_{k+1}}{\langle k \rangle}
\end{equation}
According to the book, the denominator is still $\langle k \rangle$ because
the equation is still normalized at infinity and converges to 1.
\subsection{b}
A degree-based approximation is often better than a fully-mixed model
when analyzing networks because it takes into account the heterogeneity
of node connections. In a fully-mixed model,
all nodes are assumed to have the same average degree,
which oversimplifies the real-world scenario
where nodes can have varying degrees of connectivity.
By using a degree-based approximation,
we can capture the heterogeneity in node
connectivity, which is crucial for understanding the
the behavior of complex networks such as social networks, biological
networks, and the World Wide Web. This approach allows us to better model
the real-world structure of networks and make more accurate
predictions about their behavior and dynamics.
\subsection{c}
One problem with the degree-based approximation in malware propagation,
especially when compared to the fully mixed model, is that it may not capture
the localized clustering and community structures present in real-world
networks. So this makes this model susceptible to overestimation when one
vertice has several neighbors.
\section{Q2}
\subsection{a}
This question looks like the SIR model.
According to page 55 of the malware slides the differential equations for
The SIR model is presented as follows:
\begin{equation}
    \frac{ds}{dt} = -\beta sx
\end{equation}
\begin{equation}
    \frac{dx}{dt} = \beta sx - \gamma x
\end{equation}
\begin{equation}
    \frac{dr}{dt} = -\gamma x
\end{equation}
According to the question $\beta / n$ is 0.00002. we know that the $S = s / n$.
And also we know that the recovery rate is $\gamma = 8 / 100$.
so we can rewrite those equations easily. For this ODE we have:
\begin{equation}
    \frac{ds}{dt} = -0.00002Sx
\end{equation}
\begin{equation}
    \frac{dx}{dt} = 0.00002Sx - 0.08x
\end{equation}
\begin{equation}
    \frac{dr}{dt} = 0.08x
\end{equation}
Therefore we obtained the ODEs required for the SIR model and we also do not
need to solve them.
\subsection{b}
So we need to calculate the $S$ or susceptibles at our arrival time.
from equation number 7 we can get the number of $S$ if only we know how many
infected nodes are there. There were 100 infected nodes at the time of our
arrival and 8 recovered and 30 new infections happened. This means that the
$x$ is now $100 - 8 + 30 = 122$.
We know that 30 nodes are infected that means at the time of our arrival
the steepness of the curve for $S$ is $-30$ which means that its derivation
over time is $-30$. Now we have all the ingredients of the equation number 7
that we can solve as a linear equation.
\begin{equation}
    \begin{aligned}
        -30 &= -0.00002 \times S \times 122 \\
        S &= \frac{-30}{122 \times -0.00002} \\
        &= 12295.081 \approx 12296
    \end{aligned}
\end{equation}
The approximation is pessimistic because we are not sure of the continued
steepness of the curve so we can not be sure that this value will be seen.
\subsection{c}
According to Newman's book part 17.3, a person will stay infected for a mean
value of $\frac{1}{\gamma}$.
The probability of recovery is $\gamma dt$ so the probability of staying
infected according to Newman's book is:
\begin{equation}
    \lim_{dt \rightarrow 0} (1 - \gamma dt)^{\frac{t}{dt}} = e^{-\gamma t}
\end{equation}
We can substitute the $e^{-\gamma t}$ with $(1 - \gamma)$ if the $t$ is too
small.
So we get the probability as $(1- \gamma)$ in any time segment.
for the second time segment we have $\gamma(1- \gamma)$ as the probability of
staying infected. In the second time segment, we have $\gamma(1-\gamma)^2$
This equation can be salvaged as:
\begin{equation}
    \gamma(1-\gamma)^{t-1}
\end{equation}
The mean value according to the book on page 633 is $\frac{1}{\gamma}$ which
is derived from the equation above. So for the answer, we have
$\frac{1}{0.08} = 12.5$ time steps.
\subsection{d}
Climax happens when the $X$ value is at its highest which is equation number 8.
the steepness at the climax is zero and the equation is of the second degree so
we should have either a minimum or a maximum.
\begin{equation}
    \begin{aligned}
        0 &= 0.00002 Sx - 0.08x \\
        S &= \frac{0.08x}{0.00002x} \\
          &= \frac{0.08}{0.00002} \\
          &= 4000
    \end{aligned}
\end{equation}
Because of the threshold phenomenon and the value of $\frac{\beta}{\gamma}$
The climax value must be the maximum.
\section{Q3}
Using the SVM lecture of professor Winston we must maximize $\frac{2}{|w|}$ and
minimize $\frac{1}{2}|w|^2$ which was derived from
the street walk example. (For our sanity most of the steps are excluded) \\
Also, the simplified equation governing the Lagrange one is
$y_i(\bar{x_i}\bar{w} + b) - 1 = 0$.
To solve the equation we need to change our working space from
cartiesan to lagrangian.
\begin{equation}
    L = \frac{1}{2}|w|^2 - \sum\alpha[y_i(\bar{w}\bar{x_i} + b - 1]
\end{equation}
By derivating this equation by $w$ we get
\begin{equation}
    \frac{\partial L}{\partial \bar{w}} = \bar{w} - \sum_i \alpha_i y_i x_i = 0
\end{equation}
\begin{equation}
    \bar{w} = \sum_i \alpha_i y_i x_i
\end{equation}
Another equation goes for the partial derivation of $b$.
\begin{equation}
    \frac{\partial L}{\partial b} = -\sum \alpha_i y_i = 0
\end{equation}
\begin{equation}
    \sum \alpha_i y_i = 0
\end{equation}
By replacing the corresponding variables in the first Lagrange equation we get
the last equation which is:
\begin{equation}
    L = \sum \alpha_i + \frac{1}{2} \sum_i \sum_j a_i a_j y_i y_j x_i \cdot x_j
\end{equation}
Now we can set to find the relative alphas for the SVM. Then we can find the
$\bar{w}$
\begin{equation}
    x_{-} =  (0, 0)
\end{equation}
\begin{equation}
    x_{+} = (2, 0)
\end{equation}
\begin{equation}
    y_{-} = -1
\end{equation}
\begin{equation}
    y_{+} = +1
\end{equation}
\begin{equation}
    \sum \alpha_i y_i = 0 \rightarrow \alpha_1 - \alpha_2 = 0 \rightarrow \alpha_1 = \alpha_2
\end{equation}
Then we will use equation 19 which was the last derivation of the lecture
to find the alphas
\begin{multline}
    L = 2 \alpha + \frac{1}{2}[((-1) \alpha^2 (0, 0) \cdot (0, 0)) + ((-1) \alpha^2 (0, 0) \cdot (2, 0)) + ((-1) \alpha^2 (2, 0) \cdot (0, 0)) ((-1) \alpha^2 (2, 0) \cdot (2, 0))] \\
    = 2 \alpha + \frac{1}{2}[(-1) \alpha^2 \times [0 + 0 + 0 + 4]] \\
    = 2 \alpha + -2 \alpha^2 \\
    = 2 [\alpha - \alpha^2] \\
\end{multline}
Its derivation by $\alpha$ must be zero as professor Winstons stated
(Back to 1801) because we are trying to maximize the vector values.
\begin{equation}
    \begin{aligned}
        \frac{\partial L}{\partial \alpha} &= 0 \\
                                           &= 2 [1 - 2 \alpha] = 0 \\
                                           &\rightarrow \alpha = 0.5
    \end{aligned}
\end{equation}
Now we can substitute the $\alpha$ value into equation 16 and get the $\bar{w}$.
\begin{equation}
    \begin{aligned}
        \bar{w} &= \sum_i \alpha_i y_i x_i \\
                &= (0.5 \times y_{-} x_{-}) + (0.5 \times y_{+} x_{+}) \\
                &= (0.5 \times -1 (0, 0)) + (0.5 \times 1 \times (2, 0)) \\
                &= (0, 0) + (1, 0)  \\
                &= (1, 0)
    \end{aligned}
\end{equation}
So the $\bar{w}$ is found to be (1, 0) which is the maximized street dividing
the negative and positive sidewalk that is our negative and positive $x$
values.
\section{Q4}
\subsection{a}
Crowds work by making each node seem equally likely to be the initiator of
the message. As we said each node joins the network by starting a Jondo
(from "John Doe"), which is a small process that will forward and receive
requests from other users. When the Jondo is started all nodes in the network
are informed of the new node's entrance, and will begin to select him as a
forwarder. To send a message a node chooses randomly
(with uniform probability) from all nodes in the network and forwards
the message to them which is like flipping a coin. (Extracted from Wikipedia) \\
We can formulate probability for each forward like below:
\begin{equation}
    (1-p_f)+ (1-p_f)pf+ (1-p_f)p_f^2+ ..... +(1-p_f)p_f^{n}
\end{equation}
This looks like a geometric distribution in which we can find the formulas in
the wikipedia.
\begin{equation}
    \frac{1}{1-pf}
\end{equation}
\subsection{b}
This also follows the same equation as the previous subsection. We only need
to factor in the probability that a node is infected ($\frac{n-c}{n}$).
\begin{equation}
    (1-p_f) + p_f\frac{n-c}{n}(1-p_f) + p_f^2 \frac{n-c}{n}(1-p_f) + ... + p_f^{n} \frac{n-c}{n} (1-p_f)
\end{equation}
We don't need to write the closed form of this equation which means the sum
may be sufficient.
\begin{equation}
    \sum_{i=0}^{n} p_f^i \frac{n-c}{n} (1-pf) = (1-pf) \times \sum_{i=0}^{n} p_f^i \frac{n-c}{n}
\end{equation}
\end{document}
