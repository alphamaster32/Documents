\documentclass[a4paper, 11pt]{article}
\usepackage[top=2cm, bottom=3cm, left = 2cm, right = 2cm]{geometry} 
\geometry{a4paper} 
\usepackage{textcomp}
\usepackage{graphicx} 
\usepackage{amsmath,amssymb}  
\usepackage{bm}  
\usepackage{memhfixc} 
\usepackage{fancyhdr}
\usepackage{enumerate}
\usepackage{float}
\usepackage{booktabs}
\usepackage{listings}
\pagestyle{fancy}
\setlength{\headheight}{14pt}
\addtolength{\topmargin}{-2pt}

\title{Advanced Network Security: Assignment \#2}
\author{Hossein Afkar \\ 810101102}
%\date{}

\begin{document}
\maketitle
% \tableofcontents
\section{Q1}
Because of definition of $\equiv$ we have $a \equiv b (mod) c$: that
$\forall d \in Z$ we have $a = dc + b$
For $H(m) \equiv xr + sk (mod p - 1)$ we can also find a d that:
\begin{equation}
    H(m) = xr + sk + (p-1)d
\end{equation}
Fermat's little theorem that we had from previous assignment proves that
for $p$ and $g \stackrel{p}{\equiv} 0$ we have
$g^{p-1} \stackrel{p}{\equiv} 1$.
Using that we can formally prove the signature scheme. From the signature
generation in ElFamal we can derive that:
\begin{equation}
    g^{H(m)} \equiv g^{xr + sk + (p-1)d}
\end{equation}
Fermat's little theorem can be used here to eliminate $(g^{p-1})^d$
\begin{equation}
    \equiv (g^{x})^{r} (g^{k})^s (g^{p-1})^{d}
\end{equation}
\begin{equation}
    \equiv (g^{x})^{r} (g^{k})^s 1^d
\end{equation}
\begin{equation}
    \equiv (g^{x})^{r} (g^{k})^s
\end{equation}
\begin{equation}
    \equiv y^{r} r^s
\end{equation}
\section{Q2}
For RSA we have to choose two prime numbers $p$ and $q$
and then compute the $n = pq$. After picking an encryption key $e \in
Z_{n^\prime}$ we have message as $m: c = m^e$ and decryption as $c: m = c^d$.
\\
Our thesis is:
\begin{equation}
    (m^e)^d \stackrel{n}{\equiv} m, \text{ for all } m \in Z_{n}
\end{equation}
Chinese remainder thereom states that if $gcd(p,q)=1$.
\begin{equation}
    x = y \pmod p \land x = y \pmod q \Rightarrow x = y \pmod{pq}
\end{equation}
So if can prove that the following two statements are correct we have proved
the theorem:
\begin{equation}
    (m^e)^d = m \mod p
\end{equation}
\begin{equation}
    (m^e)^d = m \mod q
\end{equation}
Because $gcd(m,n) \neq 1$, one of $gcd(m,n) = p$ or $gcd(m,n) = q$ must be
true. we start with $gcd(m,n) = p$. \\
So if $gcd(m,n) = p$ then we have $m = kp$ which measns that $m \mod p = 0$.
\begin{equation}
    (m^e)^d = ((kp)^e)^d
\end{equation}
So equation (9) is proved to be correct.
For equation (10) we have the following as stated in fermat's little theorem:
\begin{equation}
    m^{q-1} = 1 \mod q
\end{equation}
So using this we can continue with (10).
\begin{equation}
    \begin{aligned}
        (m^{e})^d &= m^{ed} \\
                  &= m^{ed - 1}m\\
                  &= m^{h(p-1)(q-1)}m\\
                  &= (m^{q-1})^{h(p-1)}m\\
                  &= 1^{h(p-1)}m = m \bmod{q}.
    \end{aligned}
\end{equation}
Which prooves that the second equation and therefore our whole theorem is
correct and RSA works when $gcd(m,n) \neq 1$.
\section{Q3}
\subsection{a}
False-accept happens when the user does not have the required privilages but
is granted access to the resource. \\
Flase-reject happens when the user does have access to the required privilages
but is denied access to the resource.
\subsection{b}
So we need the probability of the false-reject to happen at least 5 times.
\begin{equation}
    p(\text{false-reject } \times 5) = (0.1) ^ 5
\end{equation}
\subsection{c}
So if we get access the first time the probability is 0.005 or 0.5\%.
For the second time we need to be rejected first and get access in the second
try.
\begin{equation}
    p(\text{false-accept in second try}) = (1 - 0.005) \times 0.005 = 0.004975
\end{equation}
So the probability decreases for the second time. For the third time we have.
\begin{equation}
    p(\text{false-accept in third try}) = (1 - 0.005) \times (1 - 0.005)
    \times 0.005 = 0.004950
\end{equation}
Lets go straight into the fifth try.
\begin{equation}
    p(\text{false-accept in fifth try}) = (1 - 0.005)^4 \times 0.005 = 0.00490074
\end{equation}
So all of these probabilities can happen which means we can sum them up.
Sum of all these probabilities is higher than the $(0.0490074 \times 5)$ because
all of them are around the same value just some 0.0001 factors short of 0.005.
\\
So we can conclude that the probability is equal to 0.02450.
Which is around 2\% which is very high for this kind of application.
Either the admins must find another way to decrease the false-positive rates
which are very dangerous or resort to the simple password authentication or
RFID tag method that decreases the chances of false-positive compared to the
biometric ones.
\end{document}
