\documentclass[a4paper, 11pt]{article}
\usepackage[top=2cm, bottom=3cm, left = 2cm, right = 2cm]{geometry} 
\geometry{a4paper} 
\usepackage{textcomp}
\usepackage{graphicx} 
\usepackage{amsmath,amssymb}  
\usepackage{bm}  
\usepackage{memhfixc} 
\usepackage{fancyhdr}
\usepackage{enumerate}
\usepackage{float}
\usepackage{booktabs}
\usepackage{listings}
\pagestyle{fancy}
\setlength{\headheight}{14pt}
\addtolength{\topmargin}{-2pt}

\title{Advanced Network Security: Assignment \#1}
\author{Hossein Afkar \\ 810101102}
%\date{}

\begin{document}
\maketitle
% \tableofcontents
\section{Q1}
By assuming $H(x)$ as a collision resistant function, the following
propositions are also assumed:
\begin{itemize}
    \item It is computationaly infeasable to find two objects that map to the
        same hash result.
\end{itemize}
For every hash value $h = H(x)$, $x$ is a preimage of $h$.
$H$ function takes $b$ number of bytes and output $n$ bytes of hash and we
also know that $b > n$. So each hash value corresponds to $2^{b-n}$ preimages.
For $H$ to be collison resistant this value should be large enough to make it
computationaly infeasable. So the conclusion of the clause presented in the
question is false and it is not required for this statement to be true in
order for a hash function to be collision resistant.
\section{Q2}
Let $C_i$ be the probability of an event in which the \textit{i}th entry
collides with the previous ones. Also let $q$ be the number of the events
happening in the system (Number of calculated hashes) and $N$ be the number of
bins (Number of all the hashes $2^n$). There are two ways to approach this
question. First:
\begin{equation}
    C(N,q) = Pr[C_1 \vee C_2 \vee ... \vee C_q] \le Pr[C_1] + Pr[C_2] + ... + 
    Pr[C_q] \le \frac{0}{N} + \frac{1}{N} + ... + \frac{q-1}{N}
\end{equation}
\begin{equation}
    C(N,Q) \le \frac{q(q-1)}{2N}
\end{equation}
We need this probabilty to be 1.
\begin{equation}
    1 \le \frac{q(q-1)}{2N} \implies 1 \le \frac{q^2 - q}{2N} \implies 2N \le
    q^2 - q \le q^2 \implies 2N \le q^2 \implies \sqrt{2N} \le q
\end{equation}
This implies that for every q that is larger than $2 \times 2^n$ this will
definitely happen.
Secondly we can walk backwards and calculate the probability for this event
not happening.
\begin{equation}
    1 - C(N,q) = Pr[D_q] = Pr[D_q|D_{q-1}] \cdot Pr[D_{q-1}] = \sum_{i=1}^{q-1}Pr[D_{i+1}|D_i] = \sum_{i=1}^{q-1}(1-\frac{i}{N})
\end{equation}
Using the assumption $x = 1 + e^x$ we can deduce that:
\begin{equation}
    \sum_{i=1}^{q-1}e^{\frac{-i}{N}} = e^{\frac{-1}{N} - \frac{-2}{N} ...
    \frac{q-1}{N}} = e^{\frac{-q(q-1)}{2N}}
\end{equation}
Then we will get:
\begin{equation}
    C(N,q) = 1 - e^{\frac{-q(q-1)}{2N}} = \frac{q(q-1)}{2N}
\end{equation}
Which is similar to what we got from the first approach.
\section{Q3}
According to the reference provided to us by the assignment description we
have:
\begin{equation}
    \tag{Inclusion-Exclusion Principle}
    {n \choose 1} \times (n-1)! - {n \choose 2}
    \times (n-2)! + {n \choose 3} \times (n-3)! - ... {n \choose n}
    \times (n - n)!
\end{equation}
We can expand this equation into:
\begin{equation}
    \frac{n!}{1!(n-1)!} \times (n-1)! - \frac{n!}{2!(n-2)!} \times (n-2)! +
    ... - \frac{n!}{n!(n-n)!} \times (n-n)!
\end{equation}
This turn into:
\begin{equation}
    n! \times (1 - \frac{1}{2!} + \frac{1}{3!} - ...)
\end{equation}
Number of all permutations of a set that are equal in length is $n!$.
Therefore for the possibility we have:
\begin{equation}
    1 - \frac{1}{2!} + \frac{1}{3!} - \frac{1}{4!} + ... \implies
    0.6321205588285578
\end{equation}
Using a simple program in 10000 iterations proves that this series converges to
approximately 0.63212.
This implies that the possibility is larger than 60\% which is what this
question wanted us to prove.
\section{Q4}
\subsection{a}
There are $2^{n}!$ distinct mappings in our block cipher. There are $t$
key-text mappings that are already present and we need to find a key that
maps all correctly. There are $(2^{n} - t)!$ mappings to the $2^{n}$ plaintexts.
We can state the following: 
\begin{equation}
    \frac{(2^n - t)!}{2^{n}!}
\end{equation}
So the probability of this not happening is:
\begin{equation}
    1 - \frac{(2^n - t)!}{2^{n}!}
\end{equation}
\subsection{b}
We need to take into account the probability of the same thing happening twice.
This means that our mapping space has $N - t^\prime$ objects left.
\begin{equation}
    1 - (\frac{(2^n - t)!}{2^{n}!} \times \frac{2^n - t - t^\prime}{(2^{n} -t)!})
\end{equation}
\begin{equation}
    1 - \frac{(2^n - t - t^\prime)!}{2^{n}!}
\end{equation}
According to stalling there is a different answer:
\begin{equation}
    Pr(t^\prime \text{ additional fixed points}) = {N - t \choose t^\prime} \times
    Pr(\text{no fixed points in } N - t - t^\prime)
\end{equation}
\begin{equation}
    \frac{1}{(t^\prime)!} \times \sum_{k = 0}^{N - t - t^\prime}\frac{-1^{k}}{k!} \implies 1 - \frac{1}{(N - t)!}
\end{equation}
This brings us back to the same equation but with less precision which is fine
in large numbers that pass the birthday boundary.
\section{Q5}
\subsection{a}
Function presented to us as $\phi(n)$ is the Euler's totient function.
By definition $\phi(p) = p-1$ if $p$ is prime. Consider the set of all
integers that are less than $pq$:
\begin{equation}
    \phi(pq) = {1, ..., (pq-1)}
\end{equation}
The integers in this set are relatively prime to $pq$. We know that $p$ and
$q$ are prime numbers therefore we should quanitify the following:
\begin{equation}
    \{p, 2p, ..., (q-1)p\} , \{q, 2q, ... (p-1)q\}
\end{equation}
Then we can derive:
\begin{equation}
    \phi(n) = \phi(pq) = pq - 1 - [(q-1)+(p-1)] = pq - (p+q) + 1 = (p-1)(q-1)
\end{equation}
\subsection{b}
Using dirichlet's proof of fermat's little theorem we assume that $a$ is
positive and not divisible by $p$:
\begin{equation}
    a,2a,3a, ..., (p-1)a
\end{equation}
modulo of these series is:
\begin{equation}
    1, 2, 3, ..., p-1
\end{equation}
If we multiply the numbers together the result must be identical modulo of p:
\begin{equation}
    a \times 2a \times 3a \times ... \times (p-1)a \stackrel{q}{\equiv} 1
    \times 2 \times 3 \times ... \times (p-1)
\end{equation}
\begin{equation}
    a^{p-1}(p-1)! \stackrel{q}{\equiv} (p-1)!
\end{equation}
After removing $(p-1)!$ we have:
\begin{equation}
    a^{p-1} \stackrel{q}{\equiv} 1
\end{equation}
\subsection{c}
First lets formulate the $n$ as $n = (p-1)q + r$ and $r$ as $ r = n - (p-1)q
\implies n \mod (p-1)$
Using the fermat's little theorem we have
\begin{equation}
    a^{p-1} \stackrel{p}{\equiv} 1
\end{equation}
Using the exponential rule in modular arithmetic we can assume:
\begin{equation}
    a^{(p-1) \times q} \stackrel{p}{\equiv} 1^{q}
\end{equation}
Now using the scaling rule in modular arithmetic we have:
\begin{equation}
    a^{(p-1) \times q} \times a^{r} \stackrel{p}{\equiv} 1^{q} \times a^{r}
\end{equation}
\begin{equation}
    a^{(p-1) \times q + r} \stackrel{p}{\equiv} a^{r}
\end{equation}
\begin{equation}
    a^{n} \stackrel{p}{\equiv} a^{n \mod (p-1)}
\end{equation}
\begin{equation}
    a^{n} \stackrel{p}{\equiv} a^{n \mod \phi(p)}
\end{equation}
Therefore using the fermat's little theorem we reached the conclusion required
for the assignment. Using this theory we can use smaller $n$ values. This is
particulary helpful when we need to find prime numbers because smaller prime
numbers are easier to find rather than large ones.
\end{document}
