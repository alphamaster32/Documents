\documentclass[a4paper, 11pt]{article}
\usepackage[top=2cm, bottom=3cm, left = 2cm, right = 2cm]{geometry} 
\geometry{a4paper} 
\usepackage{textcomp}
\usepackage{graphicx} 
\usepackage{amsmath,amssymb}  
\usepackage{bm}  
\usepackage{memhfixc} 
\usepackage{fancyhdr}
\usepackage{enumerate}
\usepackage{float}
\usepackage{booktabs}
\usepackage{listings}
\pagestyle{fancy}
\setlength{\headheight}{14pt}
\addtolength{\topmargin}{-2pt}

\title{Advanced Network Security: Assignment \#1}
\author{Hossein Afkar}
%\date{}

\begin{document}
\maketitle
% \tableofcontents
\section{Q1}
By assuming $H(x)$ as a collision resistant function, the following
propositions are also assumed:
\begin{itemize}
    \item It is computationaly infeasable to find two objects that map to the
        same hash result.
\end{itemize}
For every hash value $h = H(x)$, $x$ is a preimage of $h$.
$H$ function takes $b$ number of bytes and output $n$ bytes of hash and we
also know that $b > n$. So each hash value corresponds to $2^{b-n}$ preimages.
For $H$ to be collison resistant this value should be large enough to make it
computationaly infeasable. So the conclusion of the clause presented in the
question is false and it is not required for this statement to be true in
order for a hash function to be collision resistant.
\section{Q2}
Let $C_i$ be the probability of an event in which the \textit{i}th entry
collides with the previous ones. Also let $q$ be the number of the events
happening in the system (Number of calculated hashes) and $N$ be the number of
bins (Number of all the hashes $2^n$). There are two ways to approach this
question. First:
\begin{equation}
    C(N,q) = Pr[C_1 \vee C_2 \vee ... \vee C_q] \le Pr[C_1] + Pr[C_2] + ... + 
    Pr[C_q] \le \frac{0}{N} + \frac{1}{N} + ... + \frac{q-1}{N}
\end{equation}
\begin{equation}
    C(N,Q) \le \frac{q(q-1)}{2N}
\end{equation}
We need this probabilty to be 1.
\begin{equation}
    1 \le \frac{q(q-1)}{2N} \implies 1 \le \frac{q^2 - q}{2N} \implies 2N \le
    q^2 - q \le q^2 \implies 2N \le q^2 \implies \sqrt{2N} \le q
\end{equation}
This implies that for every q that is larger than $2 \times 2^n$ this will
definitely happen.
Secondly we can walk backwards and calculate the probability for this event
not happening.
\begin{equation}
    1 - C(N,q) = Pr[D_q] = Pr[D_q|D_{q-1}] \cdot Pr[D_{q-1}] = \sum_{i=1}^{q-1}Pr[D_{i+1}|D_i] = \sum_{i=1}^{q-1}(1-\frac{i}{N})
\end{equation}
Using the assumption $x = 1 + e^x$ we can deduce that:
\begin{equation}
    \sum_{i=1}^{q-1}e^{\frac{-i}{N}} = e^{\frac{-1}{N} - \frac{-2}{N} ...
    \frac{q-1}{N}} = e^{\frac{-q(q-1)}{2N}}
\end{equation}
Then we will get:
\begin{equation}
    C(N,q) = 1 - e^{\frac{-q(q-1)}{2N}} = \frac{q(q-1)}{2N}
\end{equation}
Which is similar to what we got from the first approach.
\section{Q3}
\end{document}
