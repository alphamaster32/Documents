\documentclass[a4paper, 11pt]{article}
\usepackage[top=2cm, bottom=3cm, left = 2cm, right = 2cm]{geometry} 
\geometry{a4paper} 
\usepackage{textcomp}
\usepackage{graphicx} 
\usepackage{amsmath,amssymb}  
\usepackage{bm}  
\usepackage{memhfixc} 
\usepackage{fancyhdr}
\usepackage{enumerate}
\usepackage{float}
\usepackage{booktabs}
\usepackage{listings}
\pagestyle{fancy}
\setlength{\headheight}{14pt}
\addtolength{\topmargin}{-2pt}

\title{Advanced Network Security: Research Assignment \#1}
\author{Hossein Afkar \\ 810101102}
%\date{}

\begin{document}
\maketitle
% \tableofcontents
\section{NIKE and NIKE+ Methods}
NIKE and NIKE+ was constructed for smart power grids. The goal is to 
secure the meters in the power grids o satisfy the confidentiality, integrity
and availability constraints. In smart power grids the main focus is on
security and processing overhead. \\
According to the \cite{boyd1993limitation} and \cite{nessett1990critique}
there are various examples that a BAN logic proved statement might not
be totally attack free but in this research assignment we will try to
see how that affects our BAN logic implementation of this protocol.
NIKE+ is a protocol constructed by 5 steps.
\begin{enumerate}
    \item Meter chooses a random number $a$ and computes $A = a + r_{M}$.
        Then it sends it with its $ID_M$ to the AHE.
    \item After AHE got the message $T^{\prime}_M = AP$ is computed using $A$
        and $P$ which is choosed during installation. Then $T_x = b + r_{AHE}$
        and $T_{AHE} = T_xP$ is computed and via these inputs the key
        $K_{AHE \rightarrow M} = T_x(T^{\prime}_M + H_2(ID_M, y_{AHE})P_{pub})$
        is constructed then this key is hashed using
        $H_1(0, K_{AHE \rightarrow M})$ as $M_1$ and sent with $T_{AHE}$ and $ID_{AHE}$
        to the meter.
    \item In step 3 the meter computes the key
        $K_{M \rightarrow AHE} = (S_M + a)T_{AHE}$ and
        $M^{\prime}_1 = H_1(0, K_{M \rightarrow AHE})$ If $M^{\prime}_1 = M_1$
        then it authenticates AHE and sets the $K = H_1(ID_M || ID_{AHE} || K_{M \rightarrow AHE})$
        as the key.
    \item The meter computes $M_2 = H_1(1, K_{M \rightarrow AHE})$ and sends it
        back to AHE.
    \item AHE computes $M^{\prime}_2 = H_1(1, K_{AHE \rightarrow M})$
        if $M^{\prime}_2 = M_2$ then it authenticates the meter and sets 
        $K = H_1(ID_M || ID_{AHE}, K_{AHE \rightarrow M})$ as the session key.
\end{enumerate}
This operations basically removes the meter from the processing side of
the authentication equation
and therefore reduces the number of calculations needed by the meter which 
makes it more suitable for embedded systems which are using small and 
low power processors.
\section{Task 1: BAN Logic}
For starters lets rewrite the assumptions using a more literal BAN logic
representation.
\begin{enumerate}
    \item M has publickey $R_M$ ($r_M$ is private)
    \item AHE has publickey $R_{AHE}$ ($r_{AHE}$ is private)
    \item TA has publickey $P_{pub}$ ($x$ is private)
    \item M and AHE share secret key $y_{AHE}$ and $y_m$
    \item M has jurisdiction over $a$
    \item AHE has jurisdiction over $b$
    \item M believes $a$ is fresh
    \item AHE believes $b$ is fresh
    \item M believes that AHE has jurisdiction over M and AHE sharing key $K$
    \item AHE believes that M has jurisdiction over M and AHE sharing key $K$
\end{enumerate}
Security Goals to be proven:
\begin{enumerate}
    \item M believes that AHE and M are sharing key $K$
    \item AHE believes that AHE and M are sharing key $K$
    \item M believes that AHE believes that M and AHE are sharing the key $K$
    \item AHE believes that M believes that M and AHE are sharing the key $K$
\end{enumerate}
In the step 2 of the protocol we see that key gets calculated and
sent back to Meter for communication which is $k_{AHE \rightarrow M}$.
We should prove that M believes that $k_{AHE \rightarrow M}$ and that key is
fresh to continue.
\begin{itemize}
    \item M produced A and has jurisdiction over $a$.
    \item M believes $a$ is fresh.
    \item M has publickey $R_M$.
    \item We can conclude that M sees $K_{AHE \rightarrow M}$.
    \item If M sees $K_{AHE \rightarrow M}$ and M has the public key then
        M believes that AHE believes that $K_{AHE \rightarrow M}$.
    \item Then we need to believe that the $K_{AHE \rightarrow M}$ is
        fresh therefore M must believe that b is fresh. because
        b is encrypted using Public key share by AHE and M so also
        B is fresh because B is $b + r_{AHE}$.
    \item Now that M knows B and b is fresh and it knows that AHE also
        believes b and B we have: M believes that B.
    \item We know that B constructs $K_{AHE \rightarrow M}$ in the step 2 of
        the protocol so we have: M believes that $K_{AHE \rightarrow M}$
\end{itemize}
Now we need to know if M believes the $T_{AHE}$ sent by the AHE.
We can use forward chaining to achieve this.
\begin{itemize}
    \item $T_{AHE} = (b + r_{AHE}) P$
    \item If M believes $K_{AHE \rightarrow M}$ then it also believes
        $(b + r_{AHE})P$ because the components are the same. So M believes
        $T_{AHE}$.
    \item If M believes $T_{AHE}$ and M beliveves a and Also M knows its own
        $S_{m}$ that is the private component of the meter shown in paper
        as $S_i = y_i + r_i$. Then M believes all the components that construct
        the key $K_{AHE \rightarrow M}$. Then we proved Goal number 1
        which says M believes M and AHE are sharing key $K$.
\end{itemize}
Now lets work towards goal number 2.
\begin{itemize}
    \item Message $M_2$ is constructed by meter to be sent to the AHE.
    \item The message constitutes of this: $H(0, K_{M \rightarrow AHE})$
        and $T_{AHE}$ and $ID_{AHE}$.
    \item Because of the shared public key between them AHE sees A and $y_m$
        and $T_{AHE}$. Therefore AHE sees the key $K_{M \rightarrow AHE}$.
    \item AHE shares secret key $y_m$ with M. and this key was used to encrypt
        the message $M_2$. Therefore AHE believes M said $M_2$. If AHE sees
        the key $K_{M \rightarrow AHE}$ then AHE can believe that M once said 
        $K_{M \rightarrow AHE}$. If we can prove that this key is fresh
        we can prove the goal number 2.
    \item If we know that the inputs for key creation is fresh therefore the
        key is fresh. AHE believes that $T_{AHE}$ and $A$ and $y_m$ is fresh.
        Therefore we can say that AHE believes that the key is fresh.
    \item Via this step we can say that AHE believes that M believes that
        $K_{M \rightarrow AHE}$.
    \item This assumption is same as AHE believes that M and AHE share the key
        $K$.
\end{itemize}
For goal number 3 and 4 we have:
\begin{itemize}
    \item We know that M believes that AHE believes the $K_{M \rightarrow AHE}$
        and AHE believes the same key and AHE believes that they share the key.
        Therefore we can say that M believes that AHE believes that They share the
        key $K$. Note that the Key $K$ is constructed from $K_{M \rightarrow AHE}$
        and and $ID_M$ and $ID_{AHE}$. This is goal number 3.
    \item AHE believes that M believes $K_{M \rightarrow AHE}$. Note that these
        keys are the same because of the component $T_{AHE}$ and either side
        can have these by knowing $T_{AHE}$ and some more components that both
        parties already agree on. M also believes that $K_{M \rightarrow AHE}$.
        and M knows they share the key constructed by $K_{M \rightarrow AHE}$.
        Therefore AHE believes that M believes they share the same key $K$. This
        is goal number 4.
\end{itemize}
\bibliographystyle{plain}
\bibliography{ref.bib}
\end{document}
