\documentclass[a4paper, 11pt]{article}
\usepackage[top=2cm, bottom=3cm, left = 2cm, right = 2cm]{geometry} 
\geometry{a4paper} 
\usepackage{textcomp}
\usepackage{graphicx} 
\usepackage{amsmath,amssymb}  
\usepackage{bm}  
\usepackage{memhfixc} 
\usepackage{fancyhdr}
\usepackage{enumerate}
\usepackage{xepersian}
\settextfont[Scale=1]{HM FNazli}
\setlatintextfont[Scale=.9]{Noto Sans}
\pagestyle{fancy}
\setlength{\headheight}{14pt}
\addtolength{\topmargin}{-2pt}

\title{
    یک بررسی همه جانبه از شیوه‌های صنعت در سیستم‌های
    \lr{real-time}
}
\author{حسین افکار}
%\date{}

\begin{document}
\maketitle
% \tableofcontents

\section{مقدمه}
سیستم‌های
\lr{real-time}
شامل دامنه‌ی بزرگی از سیستم‌ها می‌شوند. این تنوع باعث می‌شود که توصیف کردن این مدل سیستم‌ها
و روش‌های طراحی آن‌ها سخت باشد.
روش‌های همه‌پرسی و مصاحبه برای مشخص کردن شیوه‌های صنعتی در مقالات فیلد‌های مختلف استفاده
شده‌است که این کار یک دید از ارتباط، منفعت‌ها، و اشکلات تکنولوژی‌ها و روش‌های مختلف فراهم می‌آورد
که در ادامه باعث می‌شود که روند‌ها و موقعیت‌های تحقیقات آتی فراهم شود.
در مورد سیستم‌های
\lr{real-time}
این خلأ وجود دارد که این مقاله سعی می‌کند که این خلأ را برطرف کرده و واگرایی بین روش‌های صنعتی
و نتایج تحقیقاتی را کم تر کند.
پنج هدف این مقاله عبارت‌اند از
\begin{itemize}
    \item فهمیدن این که قابل پیش‌بینی بودن زمان‌بندی برای صنعت مهم است یا خیر.
    \item شناسایی زمینه‌ای مشکلات مرتبط با صنعت.
    \item تعیین‌ این که چه ابزار یا روش‌هایی در صنعت برای دستیابی
    به قابلیت پیش‌بینی زمانی استفاده می‌شوند.
    \item تعیین این که چه روش‌ها یا ابزاری برای تأمین کردن نیازمندی‌های سیستم‌های
    \lr{real-time}
    استفاده می‌شود.
    \item تعین روند‌های برای پیش‌بینی آینده توسعه سیستم‌های
    \lr{real-time}.
\end{itemize}
پرسمان طراحی شده توسط این مقاله چهار هدف زیر را دنبال می‌کند
\begin{itemize}
    \item تعریف و بینش سیستم‌های
    \lr{real-time}
    بر پایه پاسخ ۱۲۰ کارشناس صنعتی از ارگان‌ها، کشور‌ها، و دامنه‌های کاربردی متفاوت
    \item شناخت تفاوت‌های قابل توجه بین سه صنعت بزرگ این حوزه
    \item عمومی‌سازی داده‌ها به سمت عموم توسط ابزار‌های آماری
    \item به‌دست آوردن شواهدی که ثابت می‌کند دست‌یافته‌های این مقاله جزئی از دانش عمومی
    نیستند.
\end{itemize}
\section{ایده‌ها و خلاصه مقاله}
چهار موردی که توسط
\lr{Wohlin et al.}
برای اعتبار مطرح شده‌است در این مقاله در جدول ۱ مورد بررسی قرار گرفته‌است.
همچنین در طراحی پرسمان هیچ اطلاعاتی از کاربران ذخیره نمی‌شود تا ناشناس بمانند. 
برای اینکه مشکلات سوالات چند گزینه‌ای رفع شود سوالات توسط یک تیم از متخصصات قبل از انجام
پرسمان نهایی بررسی شدند.
بعد از انجام پرسمان داده‌ها برای احقاق ۵ هدف این مقاله بررسی شده‌اند که به شرح زیر می‌باشد
\subsection{قابلیت پیش‌بینی زمانی}
نتیجه‌ای که از پرسمان به دست آمده است این است که قابلیت پیش‌بینی زمانی مهم است اما یک بخشی
از طراحی سیستم می‌باشد و طراحان درستی کاربردی را مهم‌تر ارزیابی می‌کنند. که در این کنار
بیشتر ۹۰ درصد ایمنی و امنیت سیستم را مهم تر ارزیابی کرده‌اند.

\subsection{شناسایی زمینه‌های مشکلات در صنعت}
نتایج پرسمان نشان می‌دهد که بیش از ۹۰ درصد سیستم‌ها چند‌ هسته‌ای هستند که ادامه این پرسمان نشان
می‌دهد که پلتفرم‌های سخت‌افزاری پیچیده و توزیع‌ شده هستند.
در ادامه نشان می‌دهد که بیشتر سیستم‌عامل مورد استفاده در این سیستم‌ها یا
\lr{RTOS}
است و بعد از آن استفاده از سخت‌افزار بدون سیستم‌عامل در نتایج خودش را بیشتر نشان‌ می‌دهد.
در ادامه گفته شده که ترکیب سیستم‌های
\lr{bare metal}
،
\lr{RTOS}
، و
\lr{linux}
ترکیب غالبی می‌باشد.
در ادامه این پرسمان نشان می‌دهد که ددلاین‌ها می‌توانند به طور کامل رعایت نشوند و تعریف
ساده‌تری از ددلاین در صنعت وجود دارد.
در آخر نشان داده شده است که قید‌های زمانی متفاوتی در یک سیستم می‌تواند وجود داشته باشد
\subsection{تعیین ابزار و روش‌های مورد استفاده برای دستیابی به پیش‌بینی زمانی}
پرسمان نشان می‌دهد که ابزار‌های اندازه‌گیری زمانی بیشتر از ابزار‌های آنالیز زمانی استاتیک استفاده می‌شوند
که بیشتر از نیمی از شرکت‌کنندگان ا ابزار‌های سنجشی آنالیز زمانی استفاده می‌کنند.
در ادامه نشان داده شده است که بیشتر از ۷۰ درصد شرکت‌کنندگان از یک روش استاتیک برای افزایش
قابلیت پیش‌بینی زمانی سیستم مانند خاموش کردن هسته‌ها و یا قفل کردن کش استفاده کرده‌اند.
در ادامه برای کم کردن اثرات تخلف زمانی حدود نصف شرکت‌کنندگان رفتن به مد ایمن را انجام داده‌اند
و یا سیستم را ریستارت کرده‌اند.
همچنین طبق این پرسمان سیستم‌های
\lr{real-time}
همه سیستم‌ها از روش‌های پریودیک فعال سازی تسک‌ها استفاده نمی‌کنند و این مقدار حدود
سی درصد مقدار موجود می‌باشد.
\subsection{تعیین روش‌های تأمین نیازمندی‌های سیستم‌های 
\lr{real-time}
}
طبق این پرسمان در یک سیستم ممکن است انواع روش‌های زمان‌بندی‌ای موجود باشد که لزوما
\lr{real-time}
نباشند و بیش از ۵۰ درصد شرکت‌کنندگان اظهار کرده‌اند که از دو یا چند روش زمان‌بندی در یک سیستم
استفاده می‌کنند.
همچنین نشان داده شده‌است که روش‌های بازپس‌گیری پردازنده از تسک‌ها ممکن است سیاست‌های
متفاوتی را دنبال کند و چند سیاست در یک سیستم استفاده شوند.
در ادامه بعضی از سیستم‌ها به تسک‌ها اجازه می‌دهند که بین هسته‌های متفاوت جابه‌جا شوند ولی در کل
سی درصد این سیستم‌ها این اجازه را برای برخی از تسک‌ها صادر می‌کنند.
در آخر روش معمول برای تست کردن تجاوز از زمان انجام تست‌ و چک کردن برای اتفاق این گونه خطا
است.
\subsection{روند‌ها برای آینده سیستم‌های
\lr{real-time}
}
این پرسمان نشان داده است که نظرات نسبت به استفاده از سیستم‌های چند هسته‌ای در آینده مثبت است
و در دورنمای سال ۲۰۲۴ بیشتر شرکت‌کنندگان پیش‌بینی می‌کنند که از سیستم‌های چندهسته‌ای با شتاب‌دهنده‌
استفاده کنند.
ولی در آخر این پرسمان نشان می‌دهد که سیستم‌های تک هسته‌ای در سیستم‌ها تا بعد از سال ۲۰۲۹
باقی‌ خواهند ماند. \\

بخش آخر این مقاله به این موضوع اختصاص داده‌شده است که آیا یافته‌های این پرسمان جزئی از دانش
مشترک است یا خیر که مقاله ۴ دیدگاه از این موضوع برداشت کرده است
\begin{enumerate}
    \item محققان نسبت به پاسخ‌دادن تصادفی بدتر تر عمل کرده‌اند
    ولی کارشناسان صنعتی کمی از پاسخ‌دادن تصادفی بهتر عمل کرده‌اند
    در این کنار محققان پاسخ میانی که نزدیک‌ترین پاسخ به پاسخ صحیح است را بیشتر از کارشناسان
    صنعتی انتخاب‌ کرده‌اند.
    \item با انتخاب کردن مسیر دانش جمعی انتخاب جواب پرطرفدار
    نمره ۲ از ۱۳ را در پی داشت که جواب پرطرفدار محققان ۵ از ۱۳ بار دورترین جواب مسأله بود
    و کم‌طرفدارترین جواب ۷ بار از ۱۳ بار درست بود
    که در این کنار جواب پرطرفدار کارشناسان صنعتی ۴ بار از ۱۳ بار درست بود که پرطرفدارترین
    جواب ۳ بار از ۱۳ بار دورترین جواب سوال بود.
    \item هیچ سوالی توسط بیش از ۵۰ درصد محققان پاسخ صحیح داده نشده‌ بود ولی
    بیش از ۵۰ درصد کارشناسان صنعتی به ۳ سوال جواب صحیح داده‌ بودند.
    \item سوالات ۶، ۸، و ۹ از بین سوالات توسط کمتر ۲۰ درصد
    شرکت‌کنندگان پاسخ صحیح داده شده بودند که جواب این سوالات در زمره غافلگیر‌کننده دسته‌بندی شده
    است. 
\end{enumerate}
در این آزمون فاصله زیادی بین نمره کامل و نتایج محققان و کارشناسان صنعتی وجود داشت.
برای همین این مقاله در ادامه دلایل زیر را مطرح می‌کند
\begin{itemize}
    \item ممکن است دید شرکت‌کنندگان از موضوع قدیمی باشد
    مانند متخصصان سلامت دید شیوه‌های صنعتی در زمان شکل می‌گیرد که ممکن است به کندی این
    دید به روز شود.
    \item دید محققان در مورد مقالات به صورت تخصصی است و معمولا مقالات به صورت زیرمسأله 
    بررسی می‌شود که ممکن است بخش کوچکی از یک مشکل پیچیده در صنعت سیستم‌های بی‌درنگ
    باشد و با این دید تخصصی و زیرمسأله‌ای مسائل پیچیدگی کمتری خواهند داشت ولی در این کنار
    کارشناسان صنعتی دید گسترده‌تری نسبت به مشکلات موجود در بخش خود
    و قسمت‌های دیگر سیستم دارند.
    \item ممکن است این پرسمان مبنای معتبری برای انجام این تحقیق نباشد
    البته طبق مواردی که در بخش ۳ مقاله مطرح شده است تلاش برای کم کردن مشکلات این پرسمان
    انجام شده‌است
    \item ممکن است که آزمون به اندازه کافی جدی گرفته نشده باشد
\end{itemize}
\section{نتیجه گیری و بحث}
عدم وجود یک بررسی سیستماتیک در روش‌های صنعتی این مشکل را به وجود می‌آورد که ممکن است
تحقیقات آکادمیک از صنعت فاصله بگیرند. این مقاله نشان می‌دهد که صنعت اهمیت
مشکلات زمانی را درک می‌کند اما در کنار آن به موضوعات دیگری مانند درستی کاربردی و امنیت
توجه ویژه‌ای دارد. \\
بیشتر سیستم‌های
\lr{real-time}
امروزی سیستم‌های توزیع شده‌ای هستند که از چند هسته استفاده می‌کنند و سلسله مراتب
حافظه‌ای پیچیده‌ای دارند.
این مورد که بیشتر شرکت‌کنندگان این پرسمان در نظر گرفته‌اند که شرکت‌ها در آینده نزدیک از سیستم‌های
چند هسته‌ای و پیچیده‌تر استفاده می‌کنند اهمیت مشکلات زمانی را بیشتر می‌کند.
آوردن موضوعاتی مانند مالتی‌کور‌ها و کش‌های چند سطحی باعث می‌شود که انومالی‌های
زمانی در این‌گونه سیستم‌ها بیشتر از حالت عادی شده و به طور کلی
\lr{determinism}
در سیستم از دست برود که ممکن است این موضوعات در آینده اهمیت مشکلات زمانی را در سیستم‌های
\lr{real-time}
برای کارشناسان صنعتی بیشتر کند. \\
نتایج این پرسمان نشان می‌دهد که سیستم‌های تک هسته‌ای در ۱۰ سال آینده نیز استفاده خواهند شد.
این موضوع با اینکه مشکلات زمانی در سیستم‌های تک هسته‌ای کم‌تر است و این سیستم‌ها
قابل پیش‌بینی‌تر هستند ارتباط نزدیکی دارد. \\
موضوع دیگری که این پرسمان مطرح شد سست‌تر بودن تعریف ددلاین برای کارشناسان
صنعتی است که نشان می‌دهد ددلاین‌ها به صورتی که به عنوان دیوار غیر‌قابل‌ گذر در قبل در نظر گرفته
شده‌اند نیستند و اهل صنعت بیشتر به اینکه در صورت رخداد تخطی از ددلاین چه واکنشی نشان بدهند
روی‌آوری کرده‌اند. \\
مورد جالب دیگری که در مورد سیستم‌های
\lr{real-time}
در این پرسمان مطرح شد اهمیت سیستم‌های
\lr{soft real-time}
در برابر سیستم‌های
\lr{hard real-time}
است که نشان می‌دهد تمامی کاربرد‌ها در سیستم‌های هارد خلاصه نمی‌شود و قید‌های سافت
بیش‌از ۶۰ درصد سیستم‌ها حضور دارند. \\
به طور کلی تخصصی‌سازی که در زمینه سیستم‌های
\lr{real-time}
انجام شده است مانع این می‌شود که از نتایج کار به راحتی در صنعت استفاده شود که این موضوع با
ماهیت سیستم‌های
\lr{real-time}
سازگاری دارد و به‌دست آوردن نتایج عمومی برای این‌گونه سیستم‌ها سخت می‌باشد
این مقاله مسیر بسیار مناسبی برای
تلفیق نتایج آکادمیک و موارد صنعتی ارائه می‌دهد و موضوعاتی که در این مقاله در مورد آن بحث
شده‌است نشان می‌دهد که صنعت به این نیاز دارد که روی مواردی کار شود که در قبل تمرکز اصلی
آکادمی نبوده است که طی کردن این مسیر با توجه به ماهیت تخصصی سیستم‌های
\lr{real-time}
سخت می‌باشد.

\bibliographystyle{abbrv}
% \bibliography{references}  % need to put bibtex references in references.bib 
\end{document}
