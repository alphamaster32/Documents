% LaTeX Template for short student reports.
% Citations should be in bibtex format and go in references.bib
\documentclass[a4paper, 11pt]{article}
\usepackage[top=2cm, bottom=3cm, left = 2cm, right = 2cm]{geometry} 
\geometry{a4paper} 
\usepackage[utf8]{inputenc}
\usepackage{textcomp}
\usepackage{graphicx} 
\usepackage{amsmath,amssymb}  
\usepackage{bm}  
\usepackage{memhfixc} 
\usepackage{fancyhdr}
\usepackage{xepersian}
\settextfont[Scale=1]{HM FNazli}
\setlatintextfont[Scale=.9]{Noto Sans}
\pagestyle{fancy}

\title{امنیت و آسیب پذیری سیستم های به‌شدت خودکار}
\author{حسین افکار}
%\date{}

\begin{document}
\maketitle
% \tableofcontents

\section{مقدمه}
یکی از مزایای خودکارسازی حداکثری، سرعت حرکت تولید کننده‌ها، بخش سلامت، و هواپیما سازی‌ها به 
سمت دیجیتالی شدن است.
که باعث ارتباط و تعدد تکرار مدل های صنعتی و به اشتراک‌گذاری سرویس‌ها با یکدیگر می شود
دیجیتال‌سازی باعث به‌وجود آمدن خطر حمله‌های سایبری به وسایلی می‌شود که زندگی ما را به صورت
فیزیکی تحت تاثیر قرار می‌دهند. اینترنت اشیإ پزشکی رشد زیادی را در سالیان اخیر تجربه کرده است
به‌گونه‌ای که سه از پنج شرکت خدمات درمانی از این گونه وسایل استفاده می‌کنند که هفت شرکت
از هشت شرکت استفاده از این ابزار را در نقشه فعالیت های آتی خود پیش‌بینی می‌کنند.
بیشتر از نصف استفاده از این تکنولوژی به مانیتورینگ بیمار اختصاص داده‌‌شده است که به این دستگاه‌ها
اجازه می‌دهد که این داده‌های تولید شده را جمع آوری و آنالیز بکنند و به این شکل از بیمار مراقبت
بهتری نمایند. \\
تازه بودن این تکنولوژی در صنعت به این منعنی است که رگولاتوری‌ها و استاندارد‌ها هنوز نتوانسته‌اند به 
مرحله‌ای از پختگی برسند. سازمان های مانند
\lr{FDA}
هنوز در تلاش هستند که موضع خود را در این صنایع مشخص کنند و بتوانند به تولید‌کننده‌ها و بیماران
کمک بهتری بکنند.
همانند صنعت خدمات درمانی، صنعت هوایی هم با چالش هایی از این قبیل مانند بهبود تجربه مسافر و
افزایش کارآیی و امنیت پرواز مواجه هستند.
\lr{Aviation 4.0}
یا
\lr{IoA}
معنای جدیدی است که می‌تواند خصوصیات جدیدی برای مسافر‌ها و فرستندگان کالا و سهام‌داران ارائه کند.
برخلاف صنعت خدمات درمانی ناظران هوایی استفاده از این سیستم‌ها مانند
\lr{AHMS}
را به شرکت‌های هوایی پیشنهاد می‌کنند.
این سیستم‌ها می‌توانند به خلبانان کمک کنند تا داده های بسیار زیاد هواپیما را راحت‌تر تحلیل کنند تا
از لحاظ فیزیکی و ذهنی بتوانند به شرایط موجود واکنش نشان دهند. \\
به خاطر افزایش درخواست برای سیستم های
\lr{IoA}
و
\lr{IoMT}
و همچنین دسترسی این سیستم‌ها به اینترنت پرسرعت باعث شده‌است که آسیب‌پذیری این سیستم‌ها و تبعات
فاجعه‌بار این مشکلات دارای اهمیت ویژه‌ای باشد.

\section{ایده‌ها و خلاصه مقاله}
هوشمندی در دستگاه‌ها به معنی داشتن کامپیوتر یا کنترلر درون آن‌هاست وقتی که دو دستگاه می‌توانند با
هم ارتباط برقرار کنند به 
\lr{endpoint}
تبدیل می‌شوند که هکر‌ها می‌توانند مانند کامپیوتر‌های شخصی به آن‌ها حمله کنند.
دستگاه‌های درمانی باید در‌مقابل این‌چنین حملاتی ایمن باشند و الزامات امنیتی در آن‌ها رعایت شوند تا از 
حمله های سایبری گسترده مانند
\lr{wannacry}
جلوگیری کنند.
برای سازمان‌دهی به این موضوعات این چنینی
\lr{FDA}
در سال ۲۰۱۸ از راهنمای امنیت سایبری خود رونمایی کرده است.
با این وجود سوالات متعددی باقی‌‌مانده است که به شرح زیر در مقاله مطرح شده‌اند.
\begin{itemize}
    \item استاندارد معقول برای یک دستگاه \lr{IoMT} امن چیست.
    \item چه‌چیزی میان یک مشکل طراحی یا عدم اطلاع کاربر از روش استفاده صحیح تفاوت ایجاد می‌کند.
    \item تا چه مدت نیاز به آپدیت و مانیتورینگ امنیتی نرم‌افزار هست.
    \item درصورتی که کاربر موفق به دانلود آپدیت نشود آیا به عنوان علت خرابی مشخص می‌شود
    و مسئولیت را از دوش نرم‌افزار برمی‌دارد؟
    \item آیا این ایرادات امنیتی باعث می‌شود که سهامداران شرکت واکنش نشان بدهند؟
\end{itemize}
بیمارستان‌ها و صنعت درمان به سرعت در حال حرکت به سوی سیستم های دیجیتالی هستند
که ممکن است خطرهایی حتی برای بیماران به وجود آورد. اگر استاندارد‌های امنیتی سخت‌گیرانه اعمال نشوند
می‌توانند باعث دزدیده شدن اطلاعات بیمار و یا خطر های جانی شوند.
همچنین بیشتر بیمارستان‌ها از دستگاه‌های قدیمی به دلیل هزینه استفاده می‌کنند که بسیاری از آن‌ها از
نرم‌افزار‌های قدیمی به دلیل کاهش هزینه‌ها استفاده می‌کنند که قابل آپدیت نیستند.
به عنوان مثال 
\lr{FDA}
۴۶۵۰۰۰ دستگاه را برای داشتن یک ایراد سایبرفیزیکی مرجوع کردند که باعث می‌شود هکر کنترل یک دستگاه
کنترل‌کننده قلب را در دست گیرد.
برای اینکه از آسیب حملات به شکل مؤثر جلوگیری کنیم باید پرسنل بیمارستان‌ها و مراکز خدمات درمانی
را آموزش امنیت‌سایبری دهیم و علاوه بر این با استفاده از آپدیت‌های نرم‌افزاری و همچنین بررسی خطرات
احتمالی به صورت دائم از سیستم‌ها در برابر خطرات حمله‌های سایبری محافظت کنیم. \\
همانند مراکز درمانی، صنعت هوایی نیز در معرض مشکلات با دامنه حدودا یکسانی است.
استفاده از تکنولوژی‌های
\lr{IoA}
در صنعت هوایی می‌تواند مکانیزم‌هایی برای امن‌تر شدن و بهینه‌تر شدن کار فراهم آورد بنابر مقاله
قسمت‌هایی که لازم به بررسی هستند به شرح زیر می‌باشد.
\begin{itemize}
    \item پرواز اتوماتیک برپایه قوانین و شرایط از پیش تعیین شده.
    \item توسعه یک سیستم پیش‌بینی برای زمان‌بندی تعمیرات هواپیما.
    \item کمک کامپیوتر‌های بصری برای امنیت کابین.
    \item بروزرسانی پیش‌بینی آب‌وهوایی بی‌درنگ.
    \item بهبود سیستم های جستجو و نجات در مناطق دورافتاده و دریایی.
    \item بررسی عملکرد انسانی و هشداردهی بدون مداخله روحی بی‌درنگ.
    \item همکاری در شبکه‌های هوانوردی جهانی.
    \item صحت‌سنجی نرم‌افزار و چگونگی امن‌سازی نقطه‌ به نقطه و فهم روند سایبرفیزیکی سیستم.
    \item بهبود شبکه سنسوری و آنالیز و جابه‌جایی داده‌ها در سیستم.
    \item رابط اتوماسیون و انسان برای اینکه انسان در کنترل باشد.
\end{itemize}
به‌خاطر اینکه هرگونه اختلال در شبکه هوایی اقتصاد دنیا را تحت تاثیر قرار می‌دهد دور از ذهن نیست که
امنیت در این زمینه از اهمیت ویژه‌ای برخوردار باشد.
شرکت بوئینگ یک روند تکرار شونده برای جلوگیری از ایرادات امنیتی دارد که با وجود اینکه این شرکت تجربه
زیادی در از بین بردن فاکتور های نفوذ در امنیت دارد، مشکلاتی از قبیل روش‌هایی که ما برای ساخت
سیستم‌ها استفاده می‌کنیم وجود دارد که به گفته
\lr{lanport}
در این مقاله نرم‌افزار باید همانند خانه ساخته شود به این‌گونه که از قطعات امن آماده و تست شده
استفاده شود.
\section{نتیجه گیری و بحث}
امنیت در سیستم‌های اتوماسیون شدید در بحث‌هایی مانند
\lr{IoA}
و
\lr{IoMT}
دارای اهمیت ویژه‌ای هستند. امنیت در این سیستم‌ها به دلیل اینکه با محیط فیزیکی حساسی تعامل
می‌کنند باید جدی گرفته شود. در صنعت خدمات درمانی به دلیل اینکه محیط متفاوتی نسبت به
بقیه صنایع وجود دارد، برنامه‌ریزی امنیت هنوز در این سیستم‌ها به اندازه کافی جدی گرفته نشده است و
همچنین به دلیل وجود رگولاتوری‌های محافظه کار مانند
\lr{FDA}
جای چندانی برای استاندارد‌سازی نذاشته‌اند. امنیت در سیستم به دلیل پیشرفت تکنولوژی یک موضوع پویا
است که این موضوع یکی از نکات این مقاله که استفاده از اجزای آماده امن در ساخت سیستم‌ها
را دچار مشکل می‌کند برای همین وجود تیم تحقیقاتی امنیتی در شرکت‌هایی که از اتوماسیون شدید استفاده
می‌کنند لازم است و این که سهام‌دارن سنتی این موضوع را درک کنند نیز ضروری است که مقاله به آن
پرداخته است.
برخلاف صنعت خدمات درمانی صنعت هوایی به اهمیت امنیت در سیستم‌های خود به دلیل آسیب شدید
هر گونه مشکل امنیتی در اقتصاد جهانی پی‌برده است به همین دلیل برخلاف صنعت خدمات درمانی
از روش های مدیریتی استفاده می‌کند که مراحل صحت امنیتی در آن دیده شده است.
در این مقاله در مورد انواع خطر‌های امنیتی در سیستم های هوایی و درمانی بحث شد و این مقاله
سعی کرده است که آگاهی را در زمینه امنیت بالا ببرد تا اینکه رگولاتوری‌ها و سهام‌داران شرکت‌ها
نسبت به خطرات حملات سایبری آگاه شده و اقدامات لازم را برای جلوگیری از خطرات این سیستم‌ها
انجام دهند.

\bibliographystyle{abbrv}
% \bibliography{references}  % need to put bibtex references in references.bib 
\end{document}
