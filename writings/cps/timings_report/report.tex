% LaTeX Template for short student reports.
% Citations should be in bibtex format and go in references.bib
\documentclass[a4paper, 11pt]{article}
\usepackage[top=3cm, bottom=3cm, left = 2cm, right = 2cm]{geometry} 
\geometry{a4paper} 
\usepackage[utf8]{inputenc}
\usepackage{textcomp}
\usepackage{graphicx} 
\usepackage{amsmath,amssymb}  
\usepackage{bm}  
\usepackage[pdftex,bookmarks,colorlinks,breaklinks]{hyperref}  
\hypersetup{linkcolor=black,citecolor=black,filecolor=black,urlcolor=black} % black links, for printed output
\usepackage{memhfixc} 
\usepackage{pdfsync}  
\usepackage{fancyhdr}
\pagestyle{fancy}

\title{Timing Anomalies in Dynamically Scheduled Microprocessors}
\author{Hossein Afkar}
%\date{}

\begin{document}
\maketitle
% \tableofcontents

\section{Introduction}
Timing analysis is an important part of modeling in real-time systems.
One of the important metrics in timing analysis is WCET or worst-case execution
time. Before the introduction of this paper, the way to estimate WCET was to
use the longest instruction latency and assume the cache will always miss.
This paper proves that in processors using instruction-level parallelism or
ILP, estimated WCET may be wrong and anomalies occur in several parts related
to pipelines and caches.
This paper identifies the conditions for these anomalies to occur and it also
presents various ways of estimating WCET in a processor with dynamic
instruction scheduling.


\section{Methods and Materials}
This paper starts by defining, what a timing anomaly is. If the instruction
latency is increased by \textit i or decreased by \textit d, the timing anomaly
happens if latency passes the boundary of zero and \textit i and the same
settings apply for the \textit d in the decreased cycles as it should not be
decreased past this value or become an instruction that increases in the
latency.
It also defines a processor which is not dynamically scheduling
instructions are free of timing anomalies.
after the explanation of examples of how this anomalies can occur, ways of
eliminating these anomalies are defined and by using the \textit{PSIM} which is
a per cycle PowerPC ISA simulator on an overall of seven benchmark programs
is evaluated
These benchmark programs are compiled using the gnu compile collection
using \textit{-O0} which disables the optimizations incurred by the compiler.
Methods defined by this paper to eliminate the anomalies listed as
follow
\begin{description}
    \item[The pessimistic serial execution model] Assuming a serial
    the processor we can make add the instruction latencies but we should
    assume that in each functional unit there is a single instruction
    executing at a time
    \item[The program modification method] The latter method is way too
    pessimistic. One way to make a better estimate of the WCET is to modify
    the program and insert instructions to flush the pipeline, cache, and
    the page table like \textit{invlpg}, \textit{wbinvd}, and
    \textit{cpuid} a serializing instruction like \textit cpuid should be
    executed after any variable latency instruction to make the state of 
    the pipeline is known to the environment.
\end{description}
After defining these methods, using the \textit{PSIM} simulator WCET of the
The latter benchmark is computed using the program modification method described
above.

\section{Results}
The WCET of seven benchmarks such as \textit{matmult}, \textit{bsort}, 
\textit{fib}, and more are calculated using the modified \textit{PSIM} that is
attached to a timing simulator based on the previous works that the authors
have done on cycle level symbolic execution and the ISA used for this analysis
is based on PowerPC. The results are not based on the detailed simulation of 
the pipeline instead the serial execution time is divided by the number of 
issues the architecture has. Also when modifying the program this paper used
\textit{sync} instruction to flush the pipeline and is placed on the merging
point of the paths taken on the program. Also, the \textit{sync} latency is 
assumed to be 8 cycles when put before and after a variable latency 
instruction. The actual WCET was computed by running the program with random 
data if the worst-case scenario was too complex to determine. This paper has
also calculated WCET based on worst-case serial execution and the proposed
program modification. Also for comparison, the unsafe WCET is included to be 
compared against the latter WCET calculation methods.
According to the results, the serial execution model overestimates the WCET at
least by a factor of 2. The estimated WCET of a modified program, however, is 
shorter than serial execution in all benchmarks, also in \textit{fib} and
\textit{DES} this method produced no additional overhead at all because these
programs contained no variable length latencies so no sync was inserted a no 
merging of the program paths has been done. In \textit{matmult} and 
\textit{ifdctint} slow down is caused by variable length instructions but no 
merging was done in these programs. For the remaining programs, the slowdown is 
mostly the product of merging and variable latency instructions. These results
show that the program modification method works well for programs with 
variable latency instructions.

\section{Discussion}
Results show that the program modification method can be used to get a fairly 
accurate estimation of how the program behaves in dynamically scheduled 
processors, however, there must be support in the underlying architecture to 
support this method. For example, if we use a RISCV32IM without the A extension 
which enables the atomic instructions to support we may not have the ability 
to flush the pipeline buffer to support this method of program modification.
Also using this method we must statically account for the events that happen 
in the context of program execution like syscalls and interrupts. Also, 
the scheduling of the tasks by the operating system may not be handled like merge 
points in the lifetime of a program. Also syncing primitives for various 
architectures like \textit{x86\_64} may not empty the pipeline deep enough to 
make the state of the pipeline known to the environment. In addition emulation 
of some architectures might not be equally possible and materials explaining 
how the ISA schedules and dispatches instructions may be limited and which may 
make constructing a dependable and deterministic emulator impossible. also in
an environment where architectural issues are more than two the estimation of 
the serial time being a fraction of the number of issues is not accurate as the
several issues present in \textit{CISC} environments might become too tricky to
handle in this context.

\section{Conclusion}
This paper defines a set of definitions to make an understanding of the timing 
anomalies in the context of the multi-issue pipelined processors easier.
Showing how anomalies work in the context of a PowerPC ISA made clear 
that estimation of the WCET in a multi-issue machine
needs to be something more than calculating the WCET for each instruction and
sum them up. Also by defining the methods like program modification and serial
the execution made a framework for estimating WCET much closer to what happens
in reality.


\bibliographystyle{abbrv}
% \bibliography{references}  % need to put bibtex references in references.bib 
\end{document}
