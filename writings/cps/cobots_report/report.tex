% LaTeX Template for short student reports.
% Citations should be in bibtex format and go in references.bib
\documentclass[a4paper, 11pt]{article}
\usepackage[top=3cm, bottom=3cm, left = 2cm, right = 2cm]{geometry} 
\geometry{a4paper} 
\usepackage[utf8]{inputenc}
\usepackage{textcomp}
\usepackage{graphicx} 
\usepackage{amsmath,amssymb}  
\usepackage{bm}  
\usepackage[pdftex,bookmarks,colorlinks,breaklinks]{hyperref}  
\hypersetup{linkcolor=black,citecolor=black,filecolor=black,urlcolor=black} % black links, for printed output
\usepackage{memhfixc} 
\usepackage{pdfsync}  
\usepackage{fancyhdr}
\pagestyle{fancy}

\title{Collaborative Robots}
\author{Hossein Afkar}
%\date{}

\begin{document}
\maketitle
% \tableofcontents

\section{Introduction}
Robots are becoming an increasingly important part of our lives.
With all the capabilities that robots are offering it has become
a crucial issue that we address fears and matters related to our
interaction with robots. This paper's idea revolves around robots
specialization using industry 4.0 notion and how the robotization trend
moves forward. Before this paper is discussed, the authors described the role of
FabLabs (fabrication labs) in collaborative design using tools and techniques
available at FabLabs and how it differs from legacy mass production products as
it enables people to create specialized products rather than settling for mass
produced products.
Robot automation brings many social issues including job loss into the society
as worker's skills become obsolete, the matter of how society is equipped to
deal with this kind of problem becoming more important, also governments and
legislators are actively discussing the latter social issues.
This paper connects with previous works by discussing how robots built with
industry 4.0 will help us build specialized robots and how production methods
and concepts will be different. It will also discuss how the interaction
between robots and mankind will become an integral part of our lives.

\section{Results}
Adaptation of robots has seen steady improvements in recent years as reported
by the international federation of robots more than 3 million robots will be
in use around 2020. This has caused more demand for producing robots with a
greater range of useful capabilities. Old industrial robots cannot adapt and
repurpose at the same demanding speed, causing notions like smart factories to
emerge. Smart factories are a network of digitally linked machines that are a
growing notion of industry 4.0. Smart factories - using industry 4.0
notions - shortens product development time, repurposes quickly, and
reduce product defects. Central to the issue of smart factories is the
smart manufacturing units (SMU). Every SMU is a flexible system that
self-optimizes and self-adapts according to the surronding environment. It also
connects to the network to make information available whenever it is needed.
The key idea of this paper revolves around the idea that how the robotization
trend moves forward according to the notions developed by industry 4.0 and
the demand for specialized robots with a mix of capabilities. \\
To address the ongoing trend of the world this paper defines the term cobots
or collaboration bots which are specialized robots with ideas that make them
superior to the old industry robots. As noted above Industrial robots are basic
and expensive and are not able to adapt to their environment at the same
demanding speed. because of the improvements in technology, which enabled us
to make smaller and cheaper robots, the term cobots were introduced. Cobots are 
equipped with sensors that allow them to work alongside humans and
identify and navigate around objects. \\
Cobots are defined by the ISO 10218 standard which defines power and force
limiting, safety monitored stop, speed, and position monitoring, hand guiding,
and risk assessment abilities. \\
This paper defines the trend as moving forward from cobots to microbots.
The main challenge with cobots is the software that is both accessible to
nonexperts and smart enough to collaborate with other humans and cobots.
Despite the recent improvements in vision technology like light radars, it is
still a challenge for the cobots to recognize people's movements,
object movements, and surrounding objects. according to the DARPA smart cobots
should be of moderate weight and capable of instantaneously interpreting what
happens around them. \\
To support the development of multifunctional micro-to-milli robotic
platforms, DARPA announced the SHRIMP program (short-ranged independent
micro-robotic platforms). The main challenges involving microbots are their
size, weight, and power. after addressing those challenges providing microbots 
with smart algorithms to enable their work and collaboration toward 
achieving a common goal is essential. \\
AI and machine learning also made their way into industrial robots and
microbots. robot teamwork can also be achieved by implementing technologies
like blockchain to make robot operations more secure, flexible, and autonomous.

\section{Conclusion}
This paper addresses many issues regarding excessive automation using
robots. Although robots bring forth many social issues like jobs being
lost to them, they bring features and innovations that are not easy to
overlook in the future. These features and innovations will carry themselves
over to the trends surrounding this technology. Future with robots includes
high collaboration with people and other robots through improvements in
technology like vision and artificial intelligence. This improvement
paves the path for cobots that enable collaboration and microbots
that use smart algorithms that address size, weight, and power which
is the future to work towards robot teamwork using blockchain and AI.

\bibliographystyle{abbrv}
% \bibliography{references}  % need to put bibtex references in references.bib 
\end{document}
