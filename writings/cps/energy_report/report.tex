\documentclass[a4paper, 11pt]{article}
\usepackage[top=2cm, bottom=3cm, left = 2cm, right = 2cm]{geometry} 
\geometry{a4paper} 
\usepackage{textcomp}
\usepackage{graphicx} 
\usepackage{amsmath,amssymb}  
\usepackage{bm}  
\usepackage{memhfixc} 
\usepackage{fancyhdr}
\usepackage{enumerate}
\usepackage{xepersian}
\settextfont[Scale=1]{HM FNazli}
\setlatintextfont[Scale=.9]{Noto Sans}
\pagestyle{fancy}
\setlength{\headheight}{14pt}
\addtolength{\topmargin}{-2pt}

\title{زمان‌بندی مقاوم برای یک سیستم
\lr{Weakly-Hard Real-Time}
با انرژی متغیر
}
\author{حسین افکار}
%\date{}

\begin{document}
\maketitle
% \tableofcontents

\section{مقدمه}
سیستم‌های سایبر-فیزیکی نیاز جدیدی به منابع انرژی با طول عمر بالا را ایجاد کرده‌اند.
به دلیل وجود نیاز برای استفاده از منابع تجدید‌پذیر مانند خورشید و باد این نیاز ایجاد شده که یک
سیستم برای استفاده از این منابع انرژی گسسته ایجاد بشود.
این مقاله مفهوم
\lr{Energy resiliency}
مطرح می‌شود که تعریف این خاصیت به زمان‌بند کمک می‌کند تا با کنترل کارآیی سیستم نسبت به
تغییرات انرژی در سیستم مقاوم شود.

\section{ایده‌ها و خلاصه مقاله}
نیاز به انرژی‌های پاک مانند خورشیدی، بادی، و ... باعث‌‌شده است که روش‌های جمع‌آوری و مصرف
مؤثر انرژی مورد توجه قرار گیرد.
در این بین برخلاف انرژی‌های معمول که همیشه در دسترس هستند انرژی‌های پاک به صورت تصادفی
عمل کرده و همیشه در دسترس نیستند که این تصادفی بودن سیستم را مجبور می‌کند که این تغییرات
را پیش‌بینی کرده و در برابر آن‌ها خود را مقاوم سازد.
در این مقاله یک سیستم
\lr{WHRT}
با سطوح کارآیی و انرژی متفاوت بررسی می‌شود
\subsection{\lr{WHRT}}
یک سیستم
\lr{WHRT}
سیستمی است که می‌تواند تخطی تعدادی از جاب‌های یک تسک از ددلاین خود را تحمل کند.
برای مشخص کردن این قید از نوتیشن
\lr{(m,k)-firm}
استفاده می‌شود که نشان می‌شدهد در پنجره
lr{k}
از یک تعدادی جاب از یک تسک
\lr{m}
عدد باید ددلاین خود را ببینند که دو
حالت
\lr{R}
و
\lr{E}
در آن معرفی می‌شود.
حالت
\lr{E}
وقتی است که جاب‌هایی که باید ددلاین خود را ببینند در فاصله یکسانی از یک‌دیگر قرار گیرند که
طبق فرمول ۱ مقاله می‌باشد.
حالت
\lr{R}
وقتی است که جاب‌هایی که باید ددلاین خود را ببینند در اول مجموعه قرار دارند که طبق فرمول ۲
مقاله می‌باشد. \\
در یک سیستم با پترن
lr{E}
این رابطه که با بزرگتر بودن
$\frac{m}{k}$
کارآیی بیشتر است برقرار می‌باشد ولی در یک سیستم با پترن
\lr{R}
ممکن است همیشه این موضوع صادق نباشد به این دلیل که پارتیشن آخر جاب‌ها شامل فرمول نمی شود.

\subsection{مدل مصرف‌کننده انرژی}
یک تسک ست به فرم
$\tau_i: (e_i,\pi_i,pow_i,M_i)$
برای سیستم تعریف شده‌اند که 
$e_i$
بدترین زمان اجرا و
$\pi_i$
پریود تسک با ددلاین ضمنی را مشخص می‌کنند. در این کنار
$pow_i$
بدترین توان مصرفی تسک را مشخص می‌کند و به ازای هر تسک یک مجموعه
$M_i$
داریم که به تعداد
\lr{L}
که تعداد سطوح کارآیی ما است، عضو دارد.
برداشت انرژی در این سیستم در هایپرپریود ثابت است و فقط در مرز‌های هایپرپریود عوض می‌شود.
همچنین داریم که برای تمامی سطوح کارآیی
$\frac{\Pi}{\pi_i}\;mod\;k_il = 0$
که نشان می‌دهد در پنجره‌های
\lr{k}
تغییر سطح کارآیی را نداریم.
همچنین در نظر می‌گیریم که در سیستم تسک‌ها از یک مدل
\lr{R}
یا
\lr{E}
تبعیت می‌کنند.
\subsection{مدل تأمین‌کننده سیستم}
تامین‌کننده سیستم از دو بخش تولید‌کننده و ذخیر‌ه‌کننده سیستم تشکیل شده‌است.
که تامین‌کننده به طور پیوسته در حال شارژ کردن سیستم است که این مقدار در بین هایپرپریود می‌تواند
از انتگرال مرتبه اول توان بر روی زمان به دست آید.
در مورد ذخیره‌کننده سیستم این مقاله در نظر گرفته است که نشت انرژی در آن وجود ندارد
و راندمان شارژر صد درصد است و تلفات ندارد.

\subsection{مقاومت در مقابل تغییر انرژی}
مقاومت سیستم در مقابل تغییرات انرژی در این مقاله نسبت انتگرال روی زمان کارآیی واقعی بر روی
کارآیی حداکثر که تارگت است بدست می‌آید که در یک پریود یکسان اندازه‌گیری می‌شوند.
ولی این تعریف برای سیستم یه مشکل اساسی دارد که
اول سیستم با سطح کارآیی صفر دوام نخواهد‌ آورد باوجود اینکه فرمول ۵ عددی بیشتر از صفر را گزارش
می‌دهد. مورد دوم این است که سیستم ممکن است هیچ مسیری به کارآیی بهینه خودش نداشته باشد
باوجود اینکه نرخ شارژ کم نشده و مقدار فرمول ۵ بیشتر از صفر است.

\subsection{کارآیی}
در قسمت‌های قبلی گفته شد که هر تسک به تعداد سطوح کارآیی سیستم باید قید
\lr{(m,k)}
داشته باشد که برای نشان دادن رابطه بین این قید‌ها مقاله قضیه شماره یک را مطرح می‌کند که شامل ۵
بخش می‌باشد که برای این قیود یک
\lr{Partial Order}
ارائه می‌دهد. \\
همچنین برای حل کردن مشکل مقایسه بعضی از قیود سطح کارآیی صفر نیز ارائه شده است
که قابل مقایسه با تمامی سطوح کارآیی می‌باشد.
در سطح کارآیی صفر زمان‌بند هیچ تضمینی در مورد سطح کارآیی مورد نظر کاربر نخواهد داد.

\subsection{اندازه‌گیری مقاومت}
برای اندازه‌گیری مقاومت یک سیستم کارآیی را از فرمول ۷ و نرخ انرژی را بدست می‌آوریم که
کارآیی کمینه کمترین کارآیی در هایپرپریود می‌باشد که از نتایج
$min(Perf_h,1)$
برای اندازه‌گیری بقای سیستم استفاده می‌شود.
یک قسمت دیگر مقاومت سیستم این است که انتظار داریم که با بهتر شدن کارآیی کاهش نرخ شارژ را
نداشته باشیم. که در تمامی هایپرپریود‌ها مجموع تغییر نرخ را بدست می‌آوریم اگر مثبت بود نتیجه
می‌گیریم که گزاره شماره هشت صحیح است.
در ادامه با استفاده از این مقادیر بولین که از فرمول‌های ۸ و ۹ به دست می‌آید می‌شود
ضریب
$TTR$
را برای هایپرپریود‌ها به دست آورد که طبق فرمول شماره ده به فرم زیر می‌باشد. \\
\begin{equation}
    TTR\_CO = \max(\min(\sum_{h=1}^{H-1}(Rate_h-Rate_{h-1})^+,1),
min(Perf_{h+1}-Perf_h+1)^+,1)
\end{equation}
این مقدار برای موقعی که نرخ شارژ افزایشی هست و کارآیی کاهشی، صفر می‌باشد.
که طبق این اظهارات مقاله مقاومت سیستم به صورت زیر به‌دست می‌آید. \\
\begin{equation}
    R(T) = \min(Perf_min,1) \times \min\limits_{1 \le h \le \frac{T}{\Pi}}
(TTR_CO(h)) \times \frac{\sum_{h=1}^{\frac{T}{\Pi}}Perf_i}
{L \times \frac{T}{\Pi}}
\end{equation}

\subsection{توصیف مشکلات}
سوال اولی که این مقاله مطرح کرده‌است این است که چگونه مقاومت را بیشینه کنیم.
طبق گفته این مقاله برای جواب دادن به این سوال باید ابتدا مشخص کنیم که تست امکان زمان‌بندی
سیستم به چه صورت است. روش‌هایی برای تست زمان‌بندی برای زمان‌بند‌های سیستم‌های با برداشت
انرژی ارائه شده‌است ولی برای حل مشکل اول این مقاله باید تغییر‌ یابند. \\
برای حل کردن مشکل اول یک مشکل دیگر نیز به سیستم اضافه می‌شود که با عوض کردن سطوح
کارآیی سیستم توسط زمان‌بند ممکن است سیستم دچار آنومالی شده و ایجاد این حالت گزار آنالیز را
پیچیده‌تر کند.
\subsection{\lr{Safety}}
تعریف اولی که مقاله برای این موضوع مطرح کرده است بحث ایمنی کارآیی می‌باشد که
طبق این تعریف وقتی سیستم از یک قید تسک به قید تسک دیگر تغییر حالت می‌دهد قید تسک 
بعد از آن باید از لحاظ
\lr{Ordering}
بین دو قید اولیه و ثانویه قرار گیرد. \\
تعریف بعدی در مورد ایمنی انرژی است که بیان می‌کند اگر سیستم به مود کارآیی دیگری تغییر حالت دهد
برای ایمن بودن انرژی باید مصرف انرژی تسک‌ست کاهش یابد.
سپس مقاله این دو تعریف را با دو پترن
\lr{R}
و
\lr{E}
بررسی می‌کند و ثابت می‌کند که پترن
\lr{E}
از لحاظ انرژی و کارآیی ایمن است ولی پترن
\lr{R}
نه از لحاظ کارآیی و نه از لحاظ انرژی ایمن نیست.

\subsection{تست زمان‌بندی}
این مقاله برای تست زمان‌بندی روش آنلاین را در پی می‌گیرد
روش آفلاین طبق گفته این مقاله به دو دلیل برای سیستم‌های مقاوم به انرژی مناسب نیست.
دلیل اول این‌ است که نسبت به تعداد تسک‌ها و هایپرپریود پیچیدگی نمایی دارد و دلیل دوم این است
که انرژی ذخیره‌شده اولیه در سیستم به همراه قید‌های سیستم‌های
\lr{WHRT}
را نظر نمی‌گیرد.
در این مقاله روش آنلاین به شرح زیر نمایش داده شده‌است.

\begin{equation}
    \forall 1 \le i \le n \sum_{j=1}^{i}\lceil \lceil \frac{\pi_i}{\pi_j} \rceil
\frac{m_{jl}}{k_{jl}} \rceil \; e_j \le \pi_i
\end{equation}

\begin{equation}
    \max\limits_{1 \le i \le n}\Bigg( \sum_{j=1}^{i}\lceil \lceil
\frac{\pi_i}{\pi_j} \rceil \frac{m_{jl}}{k_{jl}} \rceil \times (pow_j - Rate)^+
+  (\frac{\Pi}{\pi_i} - 1)(\lceil \lceil
\frac{\pi_i}{\pi_j} \rceil \frac{m_{jl}}{k_{jl}} \rceil \times e_j \times 
pow_j -Rate \times \pi_i)^+ \Bigg)
\end{equation}

که اثبات آن در مقاله موجود می‌باشد.

\subsection{الگوریتم مقاوم به انرژی}
الگوریتم پیشنهاد داده‌شده در این مقاله در ابتدای هایپرپریود سطح کارآیی سیستم را مشخص می‌سازد
به این گونه که در ابتدای هایپرپریود یک مساله کنترلی اپتیمال حل شده و خروجی آن
برای هایپرپریود فعلی مورد استفاده قرار می‌گیرد.
برای محاسبه این سطوح دو مدل‌سازی در این مقاله انجام شده است که مدل اول بر پایه
\lr{Integer Linear Programming}
می‌باشد که می‌تواند جواب بهینه را پیدا کند. اما این مدل نمی تواند در روش آنلاین زمان‌بندی استفاده
شود به این دلیل که سربار محاسباتی آن زیاد است. به همین دلیل در مدل دوم ورودی کنترلی سیستم
را ساده‌سازی می‌کند تا سربار زمانی سیستم کاهش یابد. در آخر اثبات می‌شود که ضریب تقریب
روش دوم نسبت به مدل اول از ۲ کمتر خواهد بود.
این موارد در معادلات ۱۷ و ۱۸ مقاله شرح داده شده‌اند.

\subsection{نتایج عملی}
برای ساختن الگوی استفاده تسک‌ها از الگوریتم
\lr{UUnifast}
استفاده شده‌است و همچنین پریود‌ها با یک تکنیک محدود کننده هایپرپریود مشخص شده‌اند و انرژی
مصرفی تسک‌ها بین ۱ تا ۱۰ واحد انرژی در نظر گرفته شده‌است.
برای ساختن قید‌های
\lr{(m,k)}
از فرمول
$(\max(l - L + k_{il},1), k_{il})$
استفاده شده‌است و در سطح کارآیی صفر این مقدار
$(0,k_{il})$
در نظر گرفته شده‌است. \\
برای اینکه خطای تست آنلاین را با توجه به سطح کارآیی حساب کنیم.
برای این کار یک جدول تهیه می‌شود که ردیف‌های آن با نرخ شارژ و ستون‌های آن با مقدار شارژ
موجود پر می‌شود. به عنوان مثال درایه
$(i,j)$
بیشترین مقدار کارآیی ممکن برای تسک‌ ست را در آن وضعیت انرژی نشان می‌دهد.
مقدار هر درایه توسط الگوریتم‌های آفلاین و آنلاین حساب خواهد شد. یک جدول دیگر نیز برای
الگوریتم
$PFP_{ASAP}$
ساخته می‌شود که تسک‌ست در بیشترین لول کارآیی شروع به کار می‌کند و در صورت میس شدن ددلاین
یک درجه کارآیی را کم کرده و ادامه می‌دهیم تا قابل زمان‌بندی شود.
با داشتن این جدول‌ها خطا با فرمول زیر به دست می‌آید.
\begin{equation}
    Error = \frac{MSPT_{PFP\_ASAP}(RATE,E(0) - MSPT_x(Rate, E(0))}
{MSPT_{PFP\_ASAP(Pr, E(0))}}
\end{equation}
مقادیر بدست آمده توسط این مقاله نشان می‌دهد که این خطا قابل چشم‌پوشی است و همچنین
زمان صرف شده برای الگوریتم آنلاین در اوردر میلی‌ثانیه می‌باشد ولی در اگوریتم آفلاین در اوردر دقیقه
است.
برای اینکه مؤثر بودن زمان‌بندی مقاوم به انرژی نشان داده شود، مدل کنترلی سیستم توسط توباکس
\lr{YALMIP}
استفاده شده‌است در ابتدا هایپرپریود کنترلر مقادیر نرخ شارژ، 
\lr{TTR}،
و انرژی ذخیره‌شده در سیستم را دریافت می‌کند. برای انجام این آزمایش از نتایج
آزمایشگاه
\lr{SRL}
در مورد پنل‌های خورشیدی استفاده شده‌است.
نتایج نشان می‌دهند که میانگین و واریانس ضریب تقریب به ترتیب ۱.۲۱ و ۰.۱۱ می‌باشد.

\section{نتیجه گیری و بحث}
در این مقاله مقاومت به تغییرات انرژی در یک سیستم
\lr{WHRT}
بررسی شد که در آن از منابع انرژی استفاده می‌شود که نسبت به تغییرات محیطی حساس هستند
و نرخ خروجی آن‌ها متغیر است. در این مقاله در نظر گرفته شد که این تغییرات انرژی فقط در مرز
هایپرپریود انجام می‌شوند. \\
همچنین در این مقاله یک
\lr{Partial Ordering}
برای قید
$(m,k)$
که در سیستم‌های
\lr{WHRT}
استفاده می‌شود پیشنهاد شد و با استفاده از این قید و تعریفات انجام شده روی آن
این مقاله ثابت می‌کند که پترن
\lr{E}
در این مدل از سیستم‌ها ایمنی کارآیی و انرژی را تضمین می‌کند. \\
همچنین یک تست امکان زمان‌بندی آنلاین برای الگوریتم
$PFP_{ASAP}$
ارائه شده است.
موضوع ریلکس کردن مقادیر کنترلی که در بخش ششم مقاله معرفی شده بود
اهمیت زمان پاسخ در الگوریتم‌های زمان‌بندی را مشخص می‌کند به عنوان مثال در صورتی که موضوع
فیزیکی دیگر مانند سیالات موضوع مقاله بود موضوع زمان می‌توانست تاثیر زیادی بر روی مسائل مربوط
به زمان‌بندی بگذارد. زمان‌بند‌های سیستم‌هایی که از معادلات
\lr{PDE}
در خود استفاده می‌کنند یا از معادلات سیستم‌های سیالاتی که حل‌ناپذیر هستند و نیاز به روش‌های
تکرار دارند استفاده می‌کنند، می‌توانند از نتایج و ایده‌های مطرح شده در این مقاله بهره‌گیری کنند که
این مقاله با ریلکس کردن مقادیر کنترلی توانسته بود مشکل خود در رابطه با رابط‌های انرژی را حل بکند.
مورد بعدی که ممکن است جالب باشد ترکیب یافته‌های این مقاله با
\lr{Thermal Aware Scheduling}
است که می‌تواند در یک سری از موارد
\lr{Embedded}
در تولید انرژی به راندمان انرژی تولیدی در سیستم کمک کند و همچنین ترکیب این مورد
با متریال‌های
\lr{Thermoelectric Generator}
می‌تواند موارد جدیدی برای بهینه کردن شار عبوری از این متریال‌ها همراه با مقاومت در برابر تغییرات
سطوح انرژی تولیدی این مواد باشد.
همچنین سربار حل معادله کنترلی در ابتدای این الگوریتم ممکن است برای یک سری از سیستم‌هایی
که بر پایه معادلات
\lr{PDE}
کار می‌کنند و مانند معادلات سیالاتی از روش‌های تکرار حل می‌شوند ممکن است سیستم را تحت تأثیر
قرار دهد و ما را وادار کند که این تاخیرات را نیز در تست‌های خود در نظر بگیریم که از حیطه این مقاله
خارج است.

\bibliographystyle{abbrv}
% \bibliography{references}  % need to put bibtex references in references.bib 
\end{document}
