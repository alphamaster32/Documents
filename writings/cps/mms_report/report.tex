\documentclass[a4paper, 11pt]{article}
\usepackage[top=2cm, bottom=3cm, left = 2cm, right = 2cm]{geometry} 
\geometry{a4paper} 
\usepackage{textcomp}
\usepackage{graphicx} 
\usepackage{amsmath,amssymb}  
\usepackage{bm}  
\usepackage{memhfixc} 
\usepackage{fancyhdr}
\usepackage{enumerate}
\usepackage{xepersian}
\settextfont[Scale=1]{HM FNazli}
\setlatintextfont[Scale=.9]{Noto Sans}
\pagestyle{fancy}
\setlength{\headheight}{14pt}
\addtolength{\topmargin}{-2pt}

\title{سیستم‌های مالتی مدل - یک اسم دیگر برای سیستم‌های \lr{MCS}}
\author{حسین افکار}
%\date{}

\begin{document}
\maketitle
% \tableofcontents

\section{مقدمه}
از زمانی که مدل وستال معرفی شده، حدود بیشتر از ۵۰۰ مقاله در مورد ایده
\lr{MCS}
منتشر شده‌است که تعدادی از این مقالات به نقد مدل وستال پرداخته‌اند. بیشتر این نقد در مورد مفهوم
\lr{Criticality}
است که این مقاله سعی می‌کند که بین مدل انتزاعی تسک‌ها و کاربرد مدل وستال تفاوت قائل شود. \\
ویژگی اصلی این مدل این است که برای متغیر‌های تعریف کننده هر تسک، چند مقدار
را متصور شده‌است. به دلیل این‌که این آنالیز در کاربرد‌های دیگر نیز استفاده می‌شود در این مقاله اسم
\lr{Multi-Mode Systems}
برای این مدل پیشنهاد شده است و همچنین کاربرد‌هایی که از نتایج این مدل بهره برده‌اند نیز معرفی
می‌شوند.

\section{ایده‌ها و خلاصه مقاله}
\subsection{مجموعه تسک مالتی مدل}
در مجموعه تسک استاندارد سه مؤلفه پریود، ددلاین، و
\lr{WCET}
را داشتیم. در مدل وستال گفته شد که مود‌های متفاوت می‌توانند اعداد متفاوتی به بعضی مؤلفه‌ها
نسبت دهند که در این مدل تمرکز بر روی
\lr{WCET}
بود. مدل‌هایی که تحت تأثیر
\lr{MCS}
نتایج خود را ارائه دادند نیازی به مفهوم
\lr{Criticality}
نداشتند بلکه می‌خواستند ایده تخصیص چند مقدار به مشخصات زمانی را بررسی کنند. \\
با استفاده از اسم
\lr{MMS}
تمرکز اصلی بر روی مشخصات مدل سیستم‌های بی‌درنگ می‌گذاریم که این کار باعث می‌شود از یک
کاربرد خاص این مدل فاصله بگیریم و این اجازه را به کاربرد‌های دیگر بدهیم که از نتایج یک مدل
خاص استفاده نکنند. \\
در مدل
\lr{MMS}
علاوه بر پارامتر‌های معمول یک سیستم، یک مجموعه
\lr{meta-parameter}
نیز داریم که می‌توانند مد‌‌های کاربردی تسک‌ها را مدل کنند.
\subsection{کاربرد‌های \lr{MMS}}
\subsubsection{\lr{Hard Real-Time Systems}}
هدف اصلی وستال یکی از کاربرد‌های
\lr{MMS}
می‌باشد که در آن
\lr{Criticality}
به عنوان یک
\lr{meta-parameter}
نشان داده می‌شود که معمولا به دو مقدار
\lr{HI}
و
\lr{LO}
محدود می‌شود که قید
$C(LO) \le C(HI)$
در آن برقرار می‌باشد. در نبود خطا
\lr{C(LO)}
یک مقدار قابل قبول است اما ممکن است برای مسأله ایمنی استاندارد‌ها ما را مجبور کنند که یک مقدار
دیگر را نیز در نظر بگیریم. \\
مدل‌هایی که برای
\lr{MMS}
و در اصل
\lr{MCS}
پیشنهاد شده‌اند، تأثیر تسک‌های
\lr{LO}
را بر روی تسک‌های
\lr{HI}
در نظر می‌گیرند و با این کار الزامات زمانی سیستم را برآورده می‌کنند ولی با این کار لود پردازشگر را پایین
می آورند که برای پارتیشن‌بندی سیستم ممکن است مطلوب باشد.
\subsubsection{\lr{Soft Real-Time Systems}}
در سیستم‌هایی با
\lr{Criticality}
پایین‌تر، موازنه‌ای بین استفاده از پردازشگر و صحت زمانی سیستم وجود دارد. در این شرایط
\lr{C(LO)}
حد بالای مناسبی است ولی ممکن است شرایطی به‌وجود آید که از این حد تخطی کند هر چند که نباید از
\lr{C(HI)}
تخطی کند. \\
\lr{Quinton et al}
از کلمه
\lr{Typical}
برای توصیف استفاده می‌کند تا این رفتار‌های آسیب‌شناختی در نظر گرفته نشوند.
روش‌های دیگر فقط زمانی به ددلاین توجه می‌کنند که رفتار تسک‌ها از میانگین معمول بیشتر نشود.
همچنین این روش‌ها در نظر می‌گیرند که این زمان از
\lr{C(HI)}
بیشتر نباشد. \\
بسیاری از مقالات طرح‌های مقاومی پیشنهاد داده‌اند که به سیستم اجازه می‌دهد که تخطی از
\lr{C(LO)}
را با استفاده از
\lr{Graceful Degradation}
تحمل کنند که در این روش از ایده
\lr{Importance}
برای تحقق
\lr{Graceful Degradation}
در سیستم استفاده می‌شود.
\subsubsection{\lr{Budget-Based}}
به جای تمرکز بر روی حدس زدن زمان اجرا می‌توان از نتایج زمان‌بندی بر روی
\lr{MMS}
برای تخصیص بودجه به پارتیشن‌های سیستم استفاده کرد. \\
تعدادی سیستم تخصیص بودجه و هماهنگی با این بودجه تخصیص داده شده پیشنهاد شده‌است
مثلا اگر یک تسک بودجه کمی داشته باشد از یک الگوریتم سریع‌تر و کم اثر‌تر برای آن تسک استفاده
می‌شود.
\subsubsection{\lr{Fault Tolerance}}
خطا در یک سیستم رویدادی است که کم رخ می‌دهد. یک سیستم مقاوم در صورت وجود خطا مجموعه‌ای
از دستورات دارد که تأثیرات خطا را کم می‌کنند. در این صورت
\lr{C(Not\_Robust)}
همان
\lr{WCET}
بدون این مجموعه دستورات و
\lr{C(Robust)}
با این مجموعه دستورات است که این دو مقدار باعث می‌شوند که نتایج سیستم‌های
\lr{MMS}
برای این حالت نیز قابل اجرا باشند.
\subsubsection{\lr{Graceful Degradation}}
در زمانی که خطا پیش بیاید ممکن است نیاز داشته باشیم تا بار روی پردازشگر را کاهش دهیم
که در زمان اجرا باید بررسی شود. ایده
\lr{Importance}
معمولا به ما کمک می‌کند تا معین کنیم چه تسکی باید خاموش یا محدود شود.
\lr{Importance}
و
\lr{Criticality}
به هم مربوط هستند ولی یکی نیستند.
\lr{Criticality}
برای صحت سنجی استفاده می‌شود در حالی که
\lr{Importance}
برای مدیریت و افزایش پایداری سیستم استفاده می‌شود. ممکن است برای سیستم نیاز داشته باشیم
که ددلاین را عوض کنیم که در این صورت ددلاین چند مقداری خواهیم داشت.
\subsubsection{\lr{Adaptivity}}
یکی از مثال‌هایی که با مفهوم
\lr{Criticality}
درگیر نمی‌شود توسط
\lr{Baruah}
معرفی شده‌است. در این موضوع هر لینک از شبکه با دو پارامتر تأخیر حد بالا و معمول شناخته می‌شود
حد بالا برای این استفاده می‌شود که مطمئن باشیم مسیری با ددلاین مشخص پیدا می‌شود و حد معمول
برای این است که مسیر با حداقل تأخیر را پیدا کنیم که با این پارامتر‌ها در زمان اجرا می‌توانیم سیستم را
\lr{Adapt}
کنیم تا بهترین لینک برای ادامه را پیدا کنیم.
\subsubsection{\lr{Modeling Mode Changes}}
قبل از کار وستال سیستم‌های با مود در تحقیقات خودشان را نشان می‌دادند. ایده این روش این است که
قید‌های زمانی یک تسک تغییر کند. این که تمامی تسک‌ها مود خود را در یک زمان تغییر می‌دهند
نشان می‌دهد که این ایده از سیستم‌های
\lr{MMS}
مقید تر است.
\subsubsection{\lr{Value Added Computation}}
در سیستم‌های دینامیک زمان‌هایی پیدا می‌شود که می‌تواند توسط تسک‌هایی که نیاز به زمان بیشتر
برای افزایش ارزش خود دارند استفاده شود. ایده
\lr{Value}
برای این کار استفاده می‌شود که شبیه به
\lr{Importance}
است با این تفاوت که
\lr{Value}
در زمان
\lr{underload}
تصمیم می‌گیرد و
\lr{Importance}
تصمیم گیری را در زمان
\lr{overload}
انجام می‌دهد.
\subsection{یک مدل برای سیستم}
در این بخش خصوصیات یک سیستم مالتی مدل بررسی می‌شود. در مدل تک مدله سیستم تعدادی پارامتر
اصلی خواهد داشت. پارامتر‌های متا، پارامتر‌هایی هستند که مشخصات سیستم مالتی مدل را برای ما
مشخص می‌کنند. یک سیستم ممکن است بیشتر از یک پارامتر متا داشته باشد که هر کدام ممکن است
روی یک پارامتر اصلی یا همه پارامتر‌ها فعالیت کنند. \\
پارامتر‌های متا ممکن است بر روی تصمیم‌گیری زمان‌بندی اثر بگذارند. به عنوان مثال در مدل وستال
مانیتورینگ ران‌تایم وجود نداشت و همینطور زمان‌بندی پویا را شامل نمی‌شد.
در اینجا
\lr{Criticality}
فقط یک پارامتر متا است. در افزونه‌های مدل وستال ممکن است از
\lr{Criticality}
یا
\lr{Importance}
برای کنترل رفتار ران‌تایم استفاده شود. \\
یک پارامتر متا را
\lr{Ordered}
می‌نامند در صورتی که بر روی پارامتر اصلی به صورت
\lr{Consistent}
اثر بگذارد. برای مثال پارامتر اصلی
\lr{P}
با پارامتر متا
\lr{M}
با مجموعه مقادیر
\lr{m1}
و
\lr{m2}
در صورتی برای یک تسک
$\tau_i: P(m1) \le P(m2)$
آنگاه برای همه تسک‌ها این نامساوی باید برقرار باشد. درغیر این صورت پارامتر
\lr{Monotonic}
است که این شرط را لزوما رعایت نمی‌کند. \\
وجود چند حالت برای سیستم دارای تناقض است که تغییر مود‌ها با سریال کردن مود‌ها این مشکل را حال
می‌کنند. بدین صورت که هر مدل در یک زمان معتبر است البته پروتکل تغییر بین مود‌ها پیچیده می‌شود
و این نیازمند است که به مدل‌ها توجه ویژه‌ای شود. \\
ددلاین باید در همه مدل‌ها رعایت شود و یا یک حد آخری برای آن درنظر گرفته شود که از آن هیچ‌وقت
تخطی نمی‌شود.
پارامتر‌های مشتق هم در سیستم موجود هستند که از پارامتر‌های اصلی و متا نشأت می‌گیرند مانند
\lr{allocation}
و اولویت.
\section{نتیجه گیری و بحث}
در این مقاله در مورد این بحث شد که مدلی که برای سیستم‌های
\lr{MCS}
معرفی شده مدلی است که برای سیستم‌های دیگر نیز می‌تواند مورد استفاده قرار گیرد و نیازی
به محدود کردن آن فقط به پارامتر
\lr{Criticality}
نیست. برای همین
\lr{MMS}
معرفی شد تا بتواند کاربرد‌های دیگری را نیز دربر بگیرد.
نکته بعدی که در این مقاله مطرح شده جدا کردن متغیر‌ها و دسته‌بندی آن‌ها با عناوینی مانند متا و مشتق
است. به‌طور کلی هدف این مقاله این است که این مدل‌سازی را به کاربرد‌های دیگری
به جز سیستم‌های
\lr{MCS}
بسط دهد و همینطور مقالات مطرح شده این موضوع را به هم متصل کرده و برای آن‌ها یک مدل
واحد به نام
\lr{MMS}
را شرح دهد.


\bibliographystyle{abbrv}
% \bibliography{references}  % need to put bibtex references in references.bib 
\end{document}
