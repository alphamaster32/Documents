\documentclass[a4paper, 11pt]{article}
\usepackage[top=2cm, bottom=3cm, left = 2cm, right = 2cm]{geometry} 
\geometry{a4paper} 
\usepackage{textcomp}
\usepackage{graphicx} 
\usepackage{amsmath,amssymb}  
\usepackage{bm}  
\usepackage{memhfixc} 
\usepackage{fancyhdr}
\usepackage{enumerate}
\usepackage{tikz}
\usepackage{float}
\usepackage{booktabs}
\usepackage[framed]{ntheorem}
\pagestyle{fancy}
\setlength{\headheight}{14pt}
\addtolength{\topmargin}{-2pt}
\theoremstyle{nonumberplain}
\newtheorem{definition}{Definition}

\title{Reliable and Efficient Concurrent Synchronization for
Embedded Real-Time Software}
\author{Hossein Afkar}
%\date{}

\begin{document}
\maketitle
% \tableofcontents

\section{Preface}
When using the most recent hardware, the most important feature that comes to
mind is the multi-threading capability of the said hardware. In order to use
the maximum capacity of the hardware, we need to make use of the features that
is available in hardware so the matter of multi-threaded programming is of
utmost importance to us as the system designers. In order to synchronize
multiple threads that are running in a system, we need to design and use
synchronization primitives that are available to us in the context of the
hardware and software. When using the traditional synchronizations mechanisms
like semaphores, we need to keep in mind that these kinds of synchronizations
are blocking and may affect the process prority assigned by the system
designer. One kind of problem that may occur using the traditional
synchronization primitives it the issue of priority inversion. \\
In 1997 the mars pathfinder suffered from a similiar issue when a low priority
task that was collecting meterlogical data blocked the bus that a high priority
task needed using a mutex. Priority inheritance is the method that is used
when handling mutual exclusive access to resources present on the system.
But using a mutex is not always the answer when designing systems. Using 
methods that utilize lock free data structures and algorithms is generally
prefered in the context of embedded systems as it will satisty the need to
use priority inheritance and thus lowering the system overhead. \\
In this project we analyze two lock free algorithm for a vector shared by a
number of threads. We also describe the methods that is available in the
hardware to make such algorithms possible and how these methods get used in a
general purpose systems programming language like C++.

\section{Introduction}

In this section we present an overview of the algoritms presented by D. Dechev
et al. \cite{cas}. As stated in the paper data structures like linked lists,

\subsection{Lock-free Dynamically Resizable Arrays}
As stated in the paper data structures like linked lists, double-ended queues,
stacks, hash tables, and binary search trees are studied in previous works but
despite the widespred use of the STL vector which is dynamically resizable
array, the problem of design and implementation of a dynamic lock-free array
has not yet been discussed. \\
The vector random access, data locality, and dynamic memory management poses
serious challenges for its non-blocking implementation.

\pagebreak

Therefore the paper proposed a set of design principles to guide the
implementation.
\begin{itemize}
    \item thread-safety
    \item lock-freedom
    \item portablility
    \item easy-to-use interfaces
    \item high level of parallelism
    \item minimal overhead
    \item simplicity
\end{itemize}
Algorithms proposed in this section are heavily dependant on the atomics
capability of the compiler and hardware therefore portability might be
compromised in architectures like RISCV architectures without the atomic
extentions but in spite of that multi-threading capability is not really
possible without the help of hardware and thus any hardware capable of
multi-threading must support the atomic instructions and instructions
controlling the memory ordering.

\subsection{Software Transactional Memory}

\section{Hardware Capabilites and Memory Ordering}
Hardware plays an important role in implemeting the software algorithms
presented in this project. In order to understand how hardware can help us
implement these algorithms we need to have a clear view of how hardware
manages memory and how it helps us achieve memory ordering.

\subsection{Memory Ordering}
In order to write correct and effiecient programs that use synchronised
primitives we need to have a clear understanding of memory ordering. This view
translates to how writes and reads are handled between different processes.
This model presents a way to eliminate the gap between the programmer expected
results an how the hardware process the result. \\
There is a formal definition of the sequential consistency provided by
Lamport\cite{ordering}:

\begin{definition}[Sequantial Consistency]
    [A multiprocessor system is sequentially consistent if] the result of any
    execution is the same as if the operations of all the processors were
    executed in some sequential order, and the operations of each individual
    processor appear in this sequence in the order specified by its program.
\end{definition}

This definition states that in order to achieve sequential consistency some
constrains should be placed on the hardware and compilers because reordering
the writes and reads in the program order can mess up the seqential consistent
view of the memory. This constrains state that
\begin{enumerate}
    \item A processor must ensure that memory operations are addressed in the
        program order.
    \item In a cache-based system all writes in the same location must be
        serialized and the value of write is not returned by a read until all
        cache line invalidates or updates generated by this write are
        acknowledged.
\end{enumerate}


% \bibliographystyle{abbrv}
% \bibliography{references}  % need to put bibtex references in references.bib 
\begin{thebibliography}{10}
    \bibitem{cas}
        D. Dechev,P. Pirkelbauer, and B. Stroustrup.
        Lock-free Dynamically Resizable Arrays.
        In A. A Shvartsman, editro, OPODIS,
        volume 4305 of Lecture Notes in Computer Science,
        pages 142-156.
        Springer 2006.
    \bibitem{fraser}
        K. Fraser and T. Harris. Concurrent programming without locks.
        ACM Trans. Comput. Syst., 25(2):5 2007.
    \bibitem{ordering}
        Adve, Sarita V., and Kourosh Gharachorloo.
        "Shared memory consistency models: A tutorial."
        computer 29.12 (1996): 66-76.
\end{thebibliography}

\end{document}
