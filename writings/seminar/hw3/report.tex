\documentclass[a4paper, 11pt]{article}
\usepackage[top=2cm, bottom=3cm, left = 2cm, right = 2cm]{geometry} 
\geometry{a4paper} 
\usepackage{textcomp}
\usepackage{graphicx} 
\usepackage{amsmath,amssymb}  
\usepackage{bm}  
\usepackage{memhfixc} 
\usepackage{fancyhdr}
\usepackage{enumerate}
\usepackage{xcolor}
\usepackage{xepersian}
\settextfont[Scale=1]{HM FNazli}
\setlatintextfont[Scale=.9]{Noto Sans}
\pagestyle{fancy}
\setlength{\headheight}{14pt}
\addtolength{\topmargin}{-2pt}

\title{
    تمرین شماره سه
}
\author{حسین افکار}
%\date{}

\begin{document}
\maketitle
% \tableofcontents

در این تمرین به بررسی بخش مقدمه می‌پردازیم \\
\begin{latin}
Computing systems are increasingly moving toward multicore
platforms and their memory subsystem represents a
crucial shared resource. As applications become more memory
intensive and processors include more cores that share the
same memory system, the performance of main memory
becomes more critical for overall system performance. \\
In a multi-core system, the processing time of a memory
request is highly variable as it depends on the location of the
access and the state of DRAM chips and the DRAM controller.
There is inter-core dependency as the memory accesses from
one core could also be influenced by requests from other
cores; the DRAM controller commonly employs scheduling
algorithms to re-order requests in order to maximize overall
DRAM throughput [16]. All these factors affect the temporal
predictability of memory intensive real-time applications due
to the high variance of their memory access time. Therefore,
there is an increasing need for memory bandwidth management solutions that
provide Quality of Service (QoS).
\end{latin}
در این دو پاراگراف نویسنده سعی کرده است در مورد اهمیت و محوریت موضوع مطالبی را مطرح کند
همچنین پاراگراف یک و دو در هم تنیده هستند و بخشی از صحبت‌های پاراگراف دو در بخش بعدی که
مرور متون است نیز کاربرد دارد.
\begin{latin}
This problem has already been recognized by many researchers,
and recent work has focused on designing more
predictable memory controllers. For example, a reservation
based approach has been applied to design a DRAM controller
that supports real-time features [4]. Resource reservation and
reclaiming techniques [17], [1] have been widely studied by
the real-time community to solve the problem of assigning
different fractions of a shared resource in a guaranteed manner
to contending applications. Proposed techniques have been
successfully applied to CPU management [10], [7], [6] and
more recently to GPU management [14], [15].
\end{latin}
در این بخش سعی شده است که متون قبلی مرور شوند که در این پاراگراف مولف محور است
\begin{latin}
In our previous work, we investigated memory bandwidth
reservation at the operating system level [21]. \textcolor{red}{However}, our
existing solution has significant limitations. First of all, it
assumes a constant available memory bandwidth, which is
not true in DRAM-based systems. Second, it can not adapt
to dynamic changes in memory resource usage. Third, while
the work in [21] provides safe performance guarantees for
hard real-time tasks, it makes no effort to optimize memory
throughput for soft real-time tasks, possibly resulting
in severely reduced system performance. To address these
limitations and challenges, we propose a new, efficient and
fine-grained memory bandwidth management system, which
we call MemGuard.
\end{latin}
در  این بخش سعی شده است که شکاف کار‌های موجود نشان داده شود که در آخر
مولف قید می‌کند که برای برطرف کردن این مسائل ابزار
\lr{MemGuard}
را معرفی می‌کنیم.
\begin{latin}
    Unlike CPU bandwidth reservation, under MemGuard the
    available memory bandwidth can be described as having
    two components: guaranteed and best effort. The guaranteed
    bandwidth represents the minimum service rate the DRAM
    system can provide, while the additionally available bandwidth
    is best effort and can not be guaranteed by the system. Memory
    bandwidth reservation is based on the guaranteed part in order
    to achieve temporal isolation. However, to efficiently utilize all
    the guaranteed memory bandwidth, a reclaiming mechanism is
    proposed leveraging each core’s usage prediction. The system
    throughput is further improved by exploiting the best effort
    bandwidth after the guaranteed bandwidth of each core is
    satisfied.
    Since our reclaiming algorithm is prediction based, misprediction can
    lead to a situation where guaranteed bandwidth is
    not delivered to the core. Therefore, MemGuard is intended to
    support mainly soft real-time systems. However, hard real-time
    tasks can be accommodated within this resource management
    framework by selectively disabling the reclaiming feature.
    We evaluate the performance of MemGuard under different
    configurations and we present detailed results in the evaluation
    section.
\end{latin}
در این بخش هدف از مقاله را داریم که این هدف به ویژگی‌های روش معرفی شده در این مقاله
می‌پردازد.
تعریف هدف در این بخش کمی مشکل است ولی با تخفیف می‌توان این ویژگی‌های این روش
را به هدف مقاله مربوط کرد.
\begin{latin}
\textcolor{red}{In summary}, the contributions of this work are: (1)
decomposing overall memory bandwidth into a guaranteed and a
best effort components. Then, we experimentally identify the
boundary so we can apply the proposed reservation technique;
(2) designing and implementing (in Linux kernel) an efficient
memory bandwidth reservation system, named MemGuard;
(3) evaluating MemGuard with an extensive set of realistic
SPEC2006 benchmarks [11] and showing its effectiveness on
a real multi-core hardware platform.
The remaining sections are \textcolor{red}{organized as follows}: Section
II describes the challenge of predictability in modern multicore systems.
Section III describes the details of the proposed
MemGuard approach. Section IV describes the evaluation
platform and the software implementation. Section V presents
the evaluation results. Section VI discusses related work. We
conclude in Section VII.
\end{latin}
در آخر ساختار مقاله مطرح شده است که نشان می‌دهد این مقاله راه‌حل جدیدی را ارائه می‌کند.
\bibliographystyle{abbrv}
% \bibliography{references}  % need to put bibtex references in references.bib 
\end{document}
