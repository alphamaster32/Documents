\documentclass[a4paper, 11pt]{article}
\usepackage[top=2cm, bottom=3cm, left = 2cm, right = 2cm]{geometry} 
\geometry{a4paper} 
\usepackage{textcomp}
\usepackage{graphicx} 
\usepackage{amsmath,amssymb}  
\usepackage{bm}  
\usepackage{memhfixc} 
\usepackage{fancyhdr}
\usepackage{enumerate}
\usepackage{xepersian}
\settextfont[Scale=1]{HM FNazli}
\setlatintextfont[Scale=.9]{Noto Sans}
\pagestyle{fancy}
\setlength{\headheight}{14pt}
\addtolength{\topmargin}{-2pt}

\title{
    تمرین شماره دو
}
\author{حسین افکار}
%\date{}

\begin{document}
\maketitle
% \tableofcontents

\section{چکیده}
در این مقاله بخش‌های
\lr{Background}
،
\lr{Purpose}
،
\lr{Results}
، و
\lr{Conclusion}
وجود دارد

\lr{
    \begin{itemize}
        \item Background: Memory bandwidth in modern multi-core platforms
        is highly variable for many reasons and is a big challenge
        in designing real-time systems as applications are increasingly
        becoming more memory intensive.
        \item Purpose: In this work, we proposed,
        designed, and implemented an efficient memory bandwidth reservation system, that we call MemGuard. MemGuard distinguishes
        memory bandwidth as two parts: guaranteed and best effort.
        \item Results:  It
        provides bandwidth reservation for the guaranteed bandwidth
        for temporal isolation, with efficient reclaiming to maximally
        utilize the reserved bandwidth. It further improves performance
        by exploiting the best effort bandwidth after satisfying each core’s
        reserved bandwidth.
        \item Conclusion: MemGuard is evaluated with SPEC2006
        benchmarks on a real hardware platform, and the results
        demonstrate that it is able to provide memory performance
        isolation with minimal impact on overall throughput.
    \end{itemize}
}

در چکیده این مقاله
\lr{Methodology}
وجود ندارد و همچنین بهتر بود دلایل بهتری در بخش
\lr{Purpose}
برای مقاله آورده می‌شد که کمک می‌کرد تا هدف این مقاله علاوه بر ساخت ابزار
به شکل بهتری مشخص شود.

\section{کلمات کلیدی}
\lr{
    \begin{itemize}
        \item Bandwidth
        \item Random access memory
        \item Multicore processing
        \item Real-time systems
        \item Regulators
        \item Benchmark testing
        \item Throughput
    \end{itemize}
}
\bibliographystyle{abbrv}
% \bibliography{references}  % need to put bibtex references in references.bib 
\end{document}
