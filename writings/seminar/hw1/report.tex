\documentclass[a4paper, 11pt]{article}
\usepackage[top=2cm, bottom=3cm, left = 2cm, right = 2cm]{geometry} 
\geometry{a4paper} 
\usepackage{textcomp}
\usepackage{graphicx} 
\usepackage{amsmath,amssymb}  
\usepackage{bm}  
\usepackage{memhfixc} 
\usepackage{fancyhdr}
\usepackage{enumerate}
\usepackage{xepersian}
\settextfont[Scale=1]{HM FNazli}
\setlatintextfont[Scale=.9]{Noto Sans}
\pagestyle{fancy}
\setlength{\headheight}{14pt}
\addtolength{\topmargin}{-2pt}

\title{
    تمرین شماره یک
}
\author{حسین افکار}
%\date{}

\begin{document}
\maketitle
% \tableofcontents

عنوان مقاله انتخاب شده
\lr{
    MemGuard: Memory Bandwidth Reservation 
    System for Efficient Performance Isolation in
    Multi-core Platforms
}
است

دلایل انتخاب این مقاله
\begin{itemize}
    \item این مقاله در ژورنال
    \lr{RTAS}
    چاپ شده است که طبق گزارش
    \lr{SciMago}
    دارای
    \lr{H-Index}
    ۳۶ می‌باشد.
    \item ضریب تأثیر این جورنال در سال ۲۰۲۰ برابر ۳.۱۴۷ می‌باشد.
    \item تعداد سایتیشن برای این مقاله ۳۶۱ می‌باشد.
    \item نویسنده اول این مقاله دارای
    \lr{H-Index}
    ۲۵ است.
    \item نویسنده دوم این مقاله دارای 
    \lr{H-Index}
    ۴۰ می‌باشد و تعداد سایتیشن‌های ایشان عدد ۵۰۵۵ است.
\end{itemize}

\bibliographystyle{abbrv}
% \bibliography{references}  % need to put bibtex references in references.bib 
\end{document}
