\documentclass[a4paper, 11pt]{article}
\usepackage[top=2cm, bottom=3cm, left = 2cm, right = 2cm]{geometry} 
\geometry{a4paper} 
\usepackage{textcomp}
\usepackage{graphicx} 
\usepackage{amsmath,amssymb}  
\usepackage{bm}  
\usepackage{memhfixc} 
\usepackage{fancyhdr}
\usepackage{enumerate}
\usepackage{xcolor}
\usepackage{xepersian}
\settextfont[Scale=1]{HM FNazli}
\setlatintextfont[Scale=.9]{Noto Sans}
\pagestyle{fancy}
\setlength{\headheight}{14pt}
\addtolength{\topmargin}{-2pt}

\title{
   \lr{The Culture Map In Sciencentific Communication}
}
\author{حسین افکار}


%\date{}

\begin{document}
\maketitle
محور اصلی این ارائه تأثیر فرهنگ در ارتباطات میان محققان بنا نهاده شده است.
هدف اصلی این ارائه این بود که بتوانیم با شناسایی فرهنگ با دیگران ارتباط مؤثر داشته باشیم و بتوانیم
انتظارات آن‌ها را برآورده کنیم. \\
در ابتدا مثالی از تفاوت فرهنگی چینی‌ها و آمریکایی‌ها آورده شده بود. فرهنگ در چین حکم می‌کند
که تا از آن‌ها نظری خواسته نشود صحبتی نکنند ولی آمریکایی‌ها این الزام را ندارند.
این مشاهده باعث می‌شود که متوجه شویم که این تفاوت‌ها به ویژگی‌های شخصی وابسته نیستند.
برای همین کتابی که محتویات این ارائه بر اساس آن تنظیم شده است ۸ زینه برای سنجش
این تفاوت‌‌های فرهنگی ارائه می‌دهد که به شرح زیر می‌باشند.
\begin{itemize}
    \item ارتباطات: نیازمند پیش‌زمینه زیاد یا کم.
    \item سنجش و بازخورد: مستقیم یا غیر مستقیم
    \item قانع‌کردن: اول قواعد یا اول کاربردی
    \item رهبری: بر اساس مساوات یا بر اساس سلسله مراتب
    \item تصمیم‌گیری: بر اساس توافق یا از بالا به پایین
    \item اطمینان: بر اساس کار یا بر اساس رابطه
    \item مخالفت: به صورت درگیری یا غیر مستقیم
    \item زمان‌بندی: به صورت خطی و یا منعطف
\end{itemize}
در مورد تمامی این موارد توضیحاتی در کل ارائه داده شد که بر اساس کتاب
\lr{the Culture Map}
است.
سپس راه‌حل‌هایی برای جلوگیری از مشکلات احتمالی مطرح شد.
برای مثال پیشنهاد داده شد که تیم‌های که
\lr{Multi Cultural}
هستند توافقات خود را مکتوب کنند و ارتباط را از پیش زمینه‌دار
\lr{Low Context}
به
\lr{High Context}
تبدیل کنند. البته این مدل می‌تواند باعث این شود که طرف مقابل فکر کند که ما به آن‌ها اطمینان نداریم. \\
در مورد معقوله بازخورد اوضاع پیچیده‌تر می‌شود.
تفاوت‌ها در فرهنگ‌ها باعث می‌شود که بازخورد‌ها به صورت مستقیم انجام نشوند و پنهان در کلمات باشند.
به عنوان مثال مقایسه‌ای بین دو فرهنگ بریتانیایی و هلندی انجام شد که اولی به صورت
غیر مستقیم با کلمات برای بازخورد‌ها بازی می‌کنند ولی هلندی‌ها خیر.
در آخر با این ارائه متوجه شدیم که تفاوت‌های فرهنگی در کنار تفاوت‌های شخصیتی
می‌توانند روی ارتباطات تأثیر می‌گذارند. تیم‌های چندفرهنگه می‌تواند تیم‌های
قوی‌ای باشند ولی اهمیت شفاف بودن در آن‌‌ها زیاد می‌شود.
در آخر منابعی معرفی شد مانند تست‌های شخصیتی
\lr{MBTI}
و کتاب
\lr{Thinking Fast and Slow}
که کتاب جالبی در مورد روند فکر کردن در انسان‌ها می‌باشد و دسته‌بندی‌هایی
بر روی تفکرات و دانش‌ جمعی انسان‌ها به زبان ساده برای خواننده‌ای که پیش‌زمینه‌ای از علم
روان‌شناسی ندارند انجام می‌دهد.
% \bibliographystyle{abbrv}
% \bibliography{references}  % need to put bibtex references in references.bib 
\end{document}
