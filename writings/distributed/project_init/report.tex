\documentclass[a4paper, 11pt]{article}
\usepackage[top=2cm, bottom=3cm, left = 2cm, right = 2cm]{geometry} 
\geometry{a4paper} 
\usepackage{textcomp}
\usepackage{graphicx} 
\usepackage{amsmath,amssymb}  
\usepackage{bm}  
\usepackage{memhfixc} 
\usepackage{fancyhdr}
\usepackage{enumerate}
\usepackage{tikz}
\usepackage{float}
\usepackage{booktabs}
\usepackage{listings}
\usepackage[framed]{ntheorem}
\pagestyle{fancy}
\setlength{\headheight}{14pt}
\addtolength{\topmargin}{-2pt}
\theoremstyle{nonumberplain}


\title{Real-Time Containers: Communication and Mixed Criticality Concepts}
\author{Hossein Afkar, Sina MirSattarian}
%\date{}

\begin{document}
\maketitle
% \tableofcontents

\section{Introduction}
Containerization is a very important section in cloud computing and distributed
systems. It provides a flexible way to isolate and deploy many software
components. As Industry 4.0 becomes widespread, methods for deploying real-time
containers in a distributed manner has become of importance in real-time
communities. It could also be of interest to see how mixed criticality
systems can be brought into the container world.

\section{Challenges}
Challenges regarding Real-Time containers includes but not limited to:
\begin{itemize}
    \item Communication between real-time containers.
    \item I/O and networking behaviour may harm determinism.
    \item Proper latency test and performance analysis of containers.
    \item Process in different containers not aware of the shared resources
        shared between them.
    \item How mixed criticality concepts may enter the containers world
\end{itemize}

\section{Solutions}
Solutions related to the challenges declared in this project may be presented
as follow:
\begin{itemize}
    \item Use operating system concepts to create a performant IPC.
    \item Use asynchronous syscalls like \textit{io\_uring} to handle
        non-deterministic I/O.
    \item Propse Tools to be created and benchmarking methods for containers.
    \item Propose monitor Creation that is compliant with mixed-criticality
        systems standards to ensure safe execution
\end{itemize}
Note that these solutions and challenges are done by some shallow literature
review in the context of the main project we may try to propose different
solutions or deem some challenges beyond the ability of this group.
% \bibliographystyle{abbrv}
% \bibliography{references}  % need to put bibtex references in references.bib 

\begin{thebibliography}{3}
    \bibitem{one} Struhár, Václav, et al.
        "Real-time containers:
        A survey." 2nd Workshop on Fog Computing and the IoT (Fog-IoT 2020).
        Schloss Dagstuhl-Leibniz-Zentrum für Informatik, 2020.
    \bibitem{two}
        Cinque, Marcello, et al.
        "Virtualizing mixed-criticality systems:
        A survey on industrial trends and issues."
        Future Generation Computer Systems 129 (2022): 315-330.Z
    \bibitem{three}
        Reichenbach, Kelvin Andres.
        "System-call offloading via Linux’io\_uring on the
        Jailhouse partitioning hypervisor."
\end{thebibliography}
\end{document}
