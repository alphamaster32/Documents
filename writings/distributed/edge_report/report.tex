\documentclass[a4paper, 11pt]{article}
\usepackage[top=2cm, bottom=3cm, left = 2cm, right = 2cm]{geometry} 
\geometry{a4paper} 
\usepackage{textcomp}
\usepackage{graphicx} 
\usepackage{amsmath,amssymb}  
\usepackage{bm}  
\usepackage{memhfixc} 
\usepackage{fancyhdr}
\usepackage{enumerate}
\usepackage{tikz}
\usepackage{float}
\usepackage{booktabs}
\usepackage{listings}
\usepackage[framed]{ntheorem}
\pagestyle{fancy}
\setlength{\headheight}{14pt}
\addtolength{\topmargin}{-2pt}
\theoremstyle{nonumberplain}


\title{Workload Prediction Using AI On Edge Compute Sites}
\author{Hossein Afkar}
%\date{}

\begin{document}
\maketitle
% \tableofcontents

\section{Introduction}
Rapid development in the edge computing has brough upon the need to
accurately determine the required resources for the limited power that is
available on edge computing nodes. With the recent increasing adoptation
of edge computing, it is critical to efficiently and accurately predict
workloads. \\
In this report we try to see where the challenges are in edge workload
prediction and also try to see if there are any
deterministic ways of implementing workload prediction
that can be used instead of AI in edge computing space.

\section{Ideas and Methods}
More services are adopting edge computing because of low latency and high
bandwidth advantages. Limited resources in the edge is one of the most
important characteristic of these systems. Predicting workload is an important
part of the resource management for edge service providers.
According to the paper \cite{paper} there are two approaches to the workload
forecasting: edge-only and cloud-only. \\
Edge only tries to implement the forecasting model on the edge site and
it models the intra-site correlations accurately. This model fails to
address the inter-site correalations with enough precision. \\
Cloud-only workload forecasting refers to a centralized location for all edge
sites to transmit their data into. This model adds additional time overhead
to both the training and inference phase. To address the mentioned challenges
the paper \cite{paper} introduces the ELASTIC tool that tries to solve the
challenges. ELASTIC is based on two stages:
\begin{itemize}
    \item Global Stage: In this stage corase-grained spatial-temporal
        forcasting at a centralized cloud is performed to capture
        inter-site correlations.
    \item Local Stage: This stage performs fine-grained spatial-temporal
        forcasting in each edge site and the goal is to capture
        the intra-site correlations between different workloads.
\end{itemize}
New research is geared towrads the AI in workload prediction as a simple search
in \textit{google scholar} shows. In order to understand the path the research
has taken it is important to know why deterministic algorithms fail at
workload prediction.
According to the phd thesis \cite{phd} there are five categorization of
workloads in the edge computing space.
\begin{enumerate}
    \item Static Workload: This workload is from unchanged behaviour of the
        users. It could be useful in determinisitc real-time communities.
        Resource utilization in this space is constantand can be allocated
        easily.
    \item Periodic Workloads: Periodic workloads are characterized by periodic
        increase and decrease in workload intensity over time. This can also
        be utilized in real-time Industry 4.0 workloads and for ensuring
        stability cloud providers must ensure the avalible resources during
        peak workload input patterns.
    \item Unpredictable Workloads: This workload defines increasing and
        decreasing workload intensity without any pattern or hard to detect
        patterns. Cloud providers often face unexpected and unplanned
        resource change from these workloads
    \item Continuously Changing Workloads: These workloads gradually increase
        or decrease in response to traffic arrival rate. Such workloads are
        easy to determine and their changes are often linear with regards to the
        traffic.
    \item Once-in-a-lifetime Workloads: These workloads are stable in a long
        time interval but at a certain point they will experience a strong
        change. The peak is very rare but cloud providers should take caution
        in providing huge resources for a short amount of time.
\end{enumerate}

\section{Conclusion}
Workload prediction is a very important area for cloud providers and edge
service providers. In this area because of the unpredictability present
in the user-level applications, it is indeed hard to propose a deterministic
algorithm for workload predictions unless we constrain our model in an
exterme way which is not suitable for general workloads. AI shines greatly in
this area because of the excellent prediction capability that AI provides us
with we can predict workloads if we have a well trained model. But the
challenge still remains in AI model resource utilization and network
bandwidth consumption.


% \bibliographystyle{abbrv}
% \bibliography{references}  % need to put bibtex references in references.bib 

\begin{thebibliography}{1}
    \bibitem{paper} Li, Yanan, et al.
        "ELASTIC: Edge Workload Forecasting based on
        Collaborative Cloud-Edge Deep Learning." (2023).
    \bibitem{phd} Lu, Yao. Workload Prediction and Resource Management
        for Energy Efficiency in Cloud Data Centres. Diss.
        University of Leicester, 2022.
\end{thebibliography}
\end{document}
