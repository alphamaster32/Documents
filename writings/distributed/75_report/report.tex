\documentclass[a4paper, 10pt]{article}
\usepackage[top=2cm, bottom=3cm, left = 2cm, right = 2cm]{geometry} 
\geometry{a4paper} 
\usepackage{textcomp}
\usepackage{graphicx} 
\usepackage{amsmath,amssymb}  
\usepackage{bm}  
\usepackage{memhfixc} 
\usepackage{fancyhdr}
\usepackage{enumerate}
\usepackage{tikz}
\usepackage{float}
\usepackage{booktabs}
\usepackage{listings}
\usepackage[framed]{ntheorem}
\pagestyle{fancy}
\setlength{\headheight}{14pt}
\addtolength{\topmargin}{-2pt}
\theoremstyle{nonumberplain}


\title{
    Progress Report:
    Real-time Communications using a Shared Memory Concept
}
\author{Hossein Afkar \and Sina MirSattarian}
%\date{}

\begin{document}
\maketitle
% \tableofcontents
In this project we proposed an architecture for a soft real-time memory
bandwidth reservation system for communication in real-time containers.
Communication in real-time containers is a very important topic according
to the survey\cite{survey}. Local real-time communication between real-time
containers is also an important notion in Industry 4.0 therefore we focused on
this issue. \\
Existing solutions did not take into account the real-time properties of the
software and the importance of bandwidth reservation in local communication. \\
What we achieved so far and solved challenges:
\begin{itemize}
    \item Overview of container communication methods.
    \item What we need for real-time capabilities.
    \item Architecture of an real-time local communication that can be mapped
        to the network using RDMA.
\end{itemize}

What we need to achieve and remaining challenges:
\begin{itemize}
    \item Practical implementation inside linux.
    \item Gathering data and comparison between GRPC and our method in local
        environment.
\end{itemize}

Practical implementation may be the bottleneck for our project.
Also polishing the proposed architecture is an important part of our porject.
Estimated time to finish may be in first half of the month of tir.


% \bibliographystyle{abbrv}
% \bibliography{references}  % need to put bibtex references in references.bib 

\begin{thebibliography}{3}
    \bibitem{memguard} Yun, Heechul, et al.
        "Memguard: Memory bandwidth reservation system for efficient
        performance isolation in multi-core platforms." 2013 IEEE 19th
        Real-Time and Embedded Technology and Applications Symposium (RTAS).
        IEEE, 2013.
    \bibitem{shimmy} Abranches, Marcelo, and Sepideh Goodarzy.
        "Shimmy: Shared memory channels for high performance inter-container
        communication." USENIX Workshop on Hot Topics in Edge Computing
        (HotEdge). 2019.
    \bibitem{survey}
        Struhár, Václav, et al. "Real-time containers: A survey."
        2nd Workshop on Fog Computing and the IoT (Fog-IoT 2020).
        Schloss Dagstuhl-Leibniz-Zentrum für Informatik, 2020.
\end{thebibliography}
\end{document}
