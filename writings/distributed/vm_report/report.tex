\documentclass[a4paper, 11pt]{article}
\usepackage[top=2cm, bottom=3cm, left = 2cm, right = 2cm]{geometry} 
\geometry{a4paper} 
\usepackage{textcomp}
\usepackage{graphicx} 
\usepackage{amsmath,amssymb}  
\usepackage{bm}  
\usepackage{memhfixc} 
\usepackage{fancyhdr}
\usepackage{enumerate}
\usepackage{tikz}
\usepackage{float}
\usepackage{booktabs}
\usepackage{listings}
\usepackage[framed]{ntheorem}
\pagestyle{fancy}
\setlength{\headheight}{14pt}
\addtolength{\topmargin}{-2pt}
\theoremstyle{nonumberplain}
\newtheorem{definition}{Definition}

\lstset{basicstyle=\ttfamily,
  showstringspaces=false,
  commentstyle=\color{red},
  keywordstyle=\color{blue}
}

\title{Virtual Machine Environment}
\author{Hossein Afkar}
%\date{}

\begin{document}
\maketitle
% \tableofcontents

\section{Environment}
In this assignment we used the lxd virtual machine environment to deploy
machines. LXC is the linux container environment that was built in 2008 and
used cgroups for containerization of applications. LXD was built on top of
this abstraction in order to manage this library. LXD also offers to use KVM
for full virtualization of the operation system and underlying components. \\
First we should initialize and init our lxd machines by using the following
commands.
\begin{lstlisting}[language=bash]
#!/bin/bash
lxd init
lxd launch ubuntu:22.04 server --vm
lxd launch ubuntu:22.04 client --vm
\end{lstlisting}

For archlinux host machines that do not have the secureboot shim in the OVMF
package we should supply following command to disable the secureboot.

\begin{lstlisting}[language=bash]
#!/bin/bash
lxc profile set default security.secureboot=false
\end{lstlisting}

Then we are presented by the VMs \textit{client} and \textit{server} in the
image list which we can get a shell from with the following command.

\begin{lstlisting}[language=bash]
#!/bin/bash
lxc exec server -- bash
root@server:~#
\end{lstlisting}

After this we will get a environment to run our sample code on.

\section{Client and Server}
For this use case we chose the golang programming language because golang is
the defacto programming language of the cloud and it provides nice tooling and
very fast compile and runtimes. \\
Methods of communications between client and server will be a http socket
that utilizes the http standard library of the golang programming language. \\

\pagebreak
Server Code is presented as follows.

\begin{lstlisting}[language=go]
func serve(port int) error {
	port_str := ":" + strconv.Itoa(port)
	tcpAddr, err := net.ResolveTCPAddr("tcp4", port_str)
	if err != nil {
		return err
	}

	listener, err := net.ListenTCP("tcp", tcpAddr)
	if err != nil {
		return err
	}

	for {
		conn, err := listener.Accept()
		if err != nil {
			continue
		}
		conn.Write([]byte("Hello World!"))
		conn.Close()
	}

	return nil
}
\end{lstlisting}

Client Code is also presented below.

\begin{lstlisting}[language=go]
func get(port int) error {
	port_str := ":" + strconv.Itoa(port)
	tcpAddr, err := net.ResolveTCPAddr("tcp4", port_str)
	if err != nil {
		return err
	}

	conn, err := net.DialTCP("tcp", nil, tcpAddr)
	if err != nil {
		return err
	}

	result, err := ioutil.ReadAll(conn)
	if err != nil {
		return err
	}

	fmt.Println(string(result))

	return nil
}

\end{lstlisting}

To push the files to the vm we can either use scp or lxc.
\begin{lstlisting}[language=bash]
#!/bin/bash
lxc file push lightship.tar.gz server/root/
lxc file push lightship.tar.gz client/root/
\end{lstlisting}

Then we will run the application in client or server mode.

\begin{lstlisting}[language=bash]
#!/bin/bash
lighship server -p 10808
lighship client -p 10808
\end{lstlisting}

Output of the client program with the running server is listed as follow.

\begin{lstlisting}[language=bash]
root@client:~# ./lightship/release/lightship client -p 10808 -a 10.128.152.62
10.128.152.62
Hello World!
root@client:~#
\end{lstlisting}

\section{Conculsion}
In this project we created two virtual machines and connected them using a
bridge. By using the golang programming language which is the defacto language
of cloud and the platform for docker and kubernetes, we was able to create a
tcp socket and send a \textit{Hello World!} from the server to the client
program.



% \bibliographystyle{abbrv}
% \bibliography{references}  % need to put bibtex references in references.bib 
\end{document}
