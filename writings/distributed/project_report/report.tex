\documentclass[a4paper, 10pt]{article}
\usepackage[top=2cm, bottom=3cm, left = 2cm, right = 2cm]{geometry} 
\geometry{a4paper} 
\usepackage{textcomp}
\usepackage{graphicx} 
\usepackage{amsmath,amssymb}  
\usepackage{bm}  
\usepackage{memhfixc} 
\usepackage{fancyhdr}
\usepackage{enumerate}
\usepackage{tikz}
\usepackage{float}
\usepackage{booktabs}
\usepackage{listings}
\usepackage[framed]{ntheorem}
\pagestyle{fancy}
\setlength{\headheight}{14pt}
\addtolength{\topmargin}{-2pt}
\theoremstyle{nonumberplain}


\title{Real-time Containers: Communication and Mixed-Criticality Concepts}
\author{Hossein Afkar \and Sina MirSattarian}
%\date{}

\begin{document}
\maketitle
% \tableofcontents

\section{Preface}
This project places its focus on a real-time container survey which states
some of the future works prospective of this area of interest. \\
The list of challenges of the real-time container based virtualization is
stated as follow:
\begin{itemize}
    \item Lack of Tools for real-time container management.
    \item Communication between real-time contianers.
\end{itemize}
The survey stated various challenges regarding the real-time containers but we
limited our area of research to mixed-critical containers and real-time
containers communications.

\section{Shimmy: Shared Memory Channels for High Performance Inter-Container
Communication}
With the adoption of the cloud technologies in the IoT, the need of
high performance communication between a fleet of containers has surfaced. \\
Application developers are rethinking their designs and migrating
towards the microservice architectures, in which the communications plays
in important role in. \\
By studying this paper we tend to familiarize ourselves with a sample
architecture surrounding shared memory communications in containers. In order 
to make this communication channel real-time, we need to apply more control
mechanisms on this communication channels. \\
Architecture that was proposed by this paper was based on allocating memory
using shmget() systemcall. By allocating this memory and passing the shm\_id
to the container, we can use this shared memory and initialize it for message
passing mechanisms using shared memory channels. This paper states that an API
is needed to fully utilize this technique. Also docker makes this easier by 
the ipc flag which makes the container share the same ipc namespace as the main
host thus making the communications easier. After reading this papers
architectures we must enable this architectures to become real-time which is
due for further stages of this project.

\section{Achieving Isolation in Mixed-Criticality Industrial Edge Systems
with Real-time Containers}
The emergence of edge computing and the Industrial Internet of Things (IIoT) has
led to the deployment of complex, distributed systems in industrial settings.
These systems often integrate multiple applications with different levels of
criticality, which poses challenges for ensuring safety, security, and performance.
In this context, achieving isolation between applications is a key requirement to
prevent faults and failures from propagating across the system. This technical
report summarizes the paper "Achieving Isolation in Mixed-Criticality Industrial
Edge Systems with Real-Time Containers" by Haddad et al., which proposes a
solution based on real-time containers to address the isolation challenges in
mixed-criticality industrial edge systems. \\
The paper defines a system model for mixed-criticality industrial edge systems,
which includes three types of components: sensors, actuators, and controllers.
Each component is associated with one or more applications with different levels
of criticality, ranging from safety-critical to non-critical. The paper identifies two
main challenges for achieving isolation in this context: (1) the need to prevent
faults and failures from propagating across applications and (2) the need to
ensure timely and predictable execution of critical applications. \\
To address these challenges, the paper proposes the use of real-time containers,
which provide isolation between applications at the operating system level. Real-
time containers are based on the concept of virtualization, which allows multiple
operating systems to run on a single physical machine. Real-time containers use
hardware virtualization to create isolated environments for each application,
which prevents faults and failures from propagating across applications.
The paper proposes a specific implementation of real-time containers, based on
the Xen hypervisor and the Xtratum real-time operating system. This
implementation provides fine-grained control over the allocation of resources,
such as CPU time, memory, and I/O bandwidth, to each application. This ensures
that critical applications receive the resources they need to meet their timing
requirements, while non-critical applications are prevented from interfering with
the execution of critical applications. \\
The paper evaluates the proposed solution through experiments on a hardware-
in-the-loop simulation platform. The experiments compare the performance of
the real-time container-based system with a traditional approach based on a
monolithic operating system. The results show that the real-time container-based
system provides better isolation between applications and better predictability of
timing behavior, even under heavy load conditions. \\
The paper presents a solution based on real-time containers to address the
isolation challenges in mixed-criticality industrial edge systems. The proposed
solution provides fine-grained control over the allocation of resources to each
application, which ensures that critical applications receive the resources they
need to meet their timing requirements. The evaluation results demonstrate the
effectiveness of the proposed solution in providing better isolation and
predictability compared to traditional approaches. Overall, the proposed solution
provides a promising approach for addressing the isolation challenges in mixed-
criticality industrial edge systems.

\section{Project Progress}
As stated in the project proposal, there are two main challenges regarding
the real-time container concepts. These challenges are real-time 
container communication and mixed-criticality concepts.
In this report, we read and analyzed two technical papers and one survey.
In the first paper, which was a survey it was made clear that the real-time
containers need more research in communication and mixed-criticality concepts.
However as we went forward in our literature review, we found out that
the mixed-criticality concepts are a very well-researched topic and may not
be relevant in the context of the survey and future works and our ideas and
challenges stated in the survey was already done by a the team that made the
survey on real-time containers.
By Studying the architecture of an example of communication between real-time
containers it is possible to move forward in this direction and combine
real-time concepts in communications between containers.
Several papers have been published on how to make bandwidth allocation between
processes deterministic and how to schedule the memory bandwidth needs of the
process that are isolated in a virtual machine environment. By utilizing
Those concepts we can try to propose an architecture for real-time container
communications and try to find out about ways to profile and trace such
properties. We can also try to implement these methods at the kernel level
and change the kernel scheduler to include the changes required to make cgroups
processes real-time. In the next report we will try to read papers on
bandwidth control and scheduling in order to be prepared to contribute an
architecture for real-time containers communications.



% \bibliographystyle{abbrv}
% \bibliography{references}  % need to put bibtex references in references.bib 

\begin{thebibliography}{2}
    \bibitem{shimmy} Abranches, Marcelo, and Sepideh Goodarzy.
        "Shimmy: Shared memory channels for high performance inter-container
        communication." USENIX Workshop on Hot Topics in Edge Computing
        (HotEdge). 2019.
    \bibitem{mcs}
        Barletta, Marco, et al. "Achieving isolation in mixed-criticality
        industrial edge systems with real-time containers." 34th Euromicro
        Conference on Real-Time Systems (ECRTS 2022). Schloss
        Dagstuhl-Leibniz-Zentrum für Informatik, 2022.
    \bibitem{survey}
        Struhár, Václav, et al. "Real-time containers: A survey."
        2nd Workshop on Fog Computing and the IoT (Fog-IoT 2020).
        Schloss Dagstuhl-Leibniz-Zentrum für Informatik, 2020.
\end{thebibliography}
\end{document}
