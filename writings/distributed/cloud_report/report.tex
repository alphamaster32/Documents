\documentclass[a4paper, 11pt]{article}
\usepackage[top=2cm, bottom=3cm, left = 2cm, right = 2cm]{geometry} 
\geometry{a4paper} 
\usepackage{textcomp}
\usepackage{graphicx} 
\usepackage{amsmath,amssymb}  
\usepackage{bm}  
\usepackage{memhfixc} 
\usepackage{fancyhdr}
\usepackage{enumerate}
\usepackage{tikz}
\usepackage{float}
\usepackage{booktabs}
\usepackage{listings}
\usepackage[framed]{ntheorem}
\pagestyle{fancy}
\setlength{\headheight}{14pt}
\addtolength{\topmargin}{-2pt}
\theoremstyle{nonumberplain}


\title{Contention And Resource Management In Cloud Infrastructure}
\author{Hossein Afkar}
%\date{}

\begin{document}
\maketitle
% \tableofcontents

\section{Introduction}
In container technology, we are promised with low-overhead and ease
of deployment. Containers rely on kernel-provided tools like
\textit{namespaces} and \textit{cgroups} to provide isolation and manage
resources. Docker container management platform exploits such features
to provide container solutions. However, services that are run in
data centers require methods like load balancing, resource allocation, and 
contention management to work efficiently and correctly. \\
Docker swarm is the name of the technology that orchestrates containers
across different nodes. In the paper, \cite{paper} questions are asked
about interferences and contention of resources when running
distributed applications in containers and how to implement them.

\section{Ideas and Methods}
In the afforementioned paper there are three questions.
The first question concerns itself with the interference and asks whether
these interferences affect performance in any way. \\
In order to synthesise a test case for this question this paper used instances
in \textit{AWS} and \textit{CloudFormation} to form a docker swarm cluster. \\
After crafting tests it can be made clear that interfernce exists and is
present and docker swarm does nothing to prevent or migitate this issue.
The second question tries to find value for implementation of a contention
manager in container orchestration software. This question is answered with
a prototype in mind. This prototype shows that being aware of the dynamic
resource utilization in the nodes and using a better placement strategy
with a naive algorithm can lead to 1.8\% and 1.1\% on YCSB1 and YSCB2
benchmarks. \\
The final question is about how to implement a resource-aware container engine.
This paper proposes two machine learning based algorithms to find good
placement for contaniner scheduling and placement into the cluster node.
In all of the tests proposed algorithms outperformed random and binpack
placement algorithms. 

\section{Conclusion}
Contention and resource management are important issues in cloud
infrastructures. To get the full performance for the applications packaged
into containers we need to place them in nodes concerning the IO and CPU
utilization that the node is experiencing at the moment. \\
Container orchestration platforms like k8s and Docker Swarm must have insight
into the types of applications that are packaged and if it is resource
intensive in IO or CPU so that they can make a decision based on the dynamic
resource available at the moment. \\
Also looking at this problem statically may not be the best idea.



% \bibliographystyle{abbrv}
% \bibliography{references}  % need to put bibtex references in references.bib 

\begin{thebibliography}{1}
    \bibitem{paper} Chiang, Ron C.
        "Contention-aware container placement strategy for docker swarm with
        machine learning based clustering algorithms."
        Cluster Computing (2020): 1-11.
\end{thebibliography}
\end{document}
