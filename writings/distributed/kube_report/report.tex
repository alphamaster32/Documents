\documentclass[a4paper, 11pt]{article}
\usepackage[top=2cm, bottom=3cm, left = 2cm, right = 2cm]{geometry} 
\geometry{a4paper} 
\usepackage{textcomp}
\usepackage{graphicx} 
\usepackage{amsmath,amssymb}  
\usepackage{bm}  
\usepackage{memhfixc} 
\usepackage{fancyhdr}
\usepackage{enumerate}
\usepackage{tikz}
\usepackage{float}
\usepackage{booktabs}
\usepackage{listings}
\usepackage[framed]{ntheorem}
\pagestyle{fancy}
\setlength{\headheight}{14pt}
\addtolength{\topmargin}{-2pt}
\theoremstyle{nonumberplain}
\newtheorem{definition}{Definition}

\lstset{
  basicstyle=\ttfamily,
  columns=fullflexible,
  frame=single,
  breaklines=true,
  postbreak=\mbox{\textcolor{red}{$\hookrightarrow$}\space},
}

\title{Kubernetes and FaaS platforms}
\author{Hossein Afkar}
%\date{}

\begin{document}
\maketitle
% \tableofcontents

\section{Environment}
In this section we will describe the environment that this computer assignment
was deployed on.
Kubernetes distribution of choice for this assignment was the \textit{k3s} by
rancher. Reasoning for this distribution consist of its ease of use and simple
one node deployment methods. By making helm accessible to us it also makes
deploying out FaaS of choice easier. \\
Our local machine is using \textit{arch linux} as the linux distribution of
choice; therefore we can use \textit{AUR} to install k3s and its dependecies.
After installing k3s we can start it using \textit{systemd}.
Then we have to confiugure the \textit{kubectl} to use the k3s certificates.
this is achieved by editing the \textit{yaml} file located at the home folder.
By using the variables defined in the \textit{/etc/rancher/k3s.yaml} we can
configure the \textit{kubectl} to authenticate with our local deployment of the
kubernetes cluster. \\
The next step is to choose a FaaS platform. We chose the \textit{OpenWhisk}
as stated by the computer assignment description. The best way to deploy
\textit{OpenWhisk} is to use \textit{Helm} to deploy it onto our single node
cluster. OpenWhisk organiztion has provided helm charts in the orgs
repositories for us to use but before that we should define a \textit{yaml}
file that describes the environment and pass it the the helm for deployment.
Luckily the repository was informative enough about the deploy and
configuration process. \\
We can deploy the \textit{OpenWhisk} using the following command command and
check progress using \textit{kubectl}.
\begin{lstlisting}[language=bash]
helm install owdev openwhisk/openwhisk -n openwhisk --create-namespace -f kube/openwhisk.yml
kubectl get pods -n openwhisk --watch
\end{lstlisting}
After that we are ready to deploy functions.
In order to deploy function we should use the \textit{OpenWhisk} client that is
called \textit{wsk}. Configuring \textit{wsk} is done through a file named
\textit{~/.wskprops}. It should be suppiled with the api keys that are set as
default by the helm chart. After this we supply our golang based functions to
the openwhisk go action runner.
We can add and run an action to the \textit{Openwhisk
environment} using the following command.
\begin{lstlisting}[language=bash]
wsk -i action create ACTION_NAME ACTION
wsk -i action invoke ACTION_NAME --result
\end{lstlisting}
This will print the json results provided by the invoked function.
Our functions are pretty simple One functions invokes another with an
api call supplying a rest post request to the invoked function with the message
and the timestamp. The invoked funtion then will supplies the callee with
another json message with timestamp and exits gracefully. The callee the
returns the results and ends the deployment as intended. \\
An example of the client function is stated as follow:
\begin{lstlisting}[language=go]
const (
	Url   = "https://172.16.157.210:31001/api/v1/namespaces/_/actions/server?blocking=true&result=true"
	Token = "Basic MjNiYzQ2YjEtNzFmNi00ZWQ1LThjNTQtODE2YWE0ZjhjNTAyOjEyM3pPM3haQ0xyTU42djJCS0sxZFhZRnBYbFBrY2NPRnFtMTJDZEFzTWdSVTRWck5aOWx5R1ZDR3VNREdJd1A="
)

type Request struct {
	Msg       string `json:"message"`
	Timestamp string `json:"timestamp"`
}

// Main is the function implementing the action
func Main(obj map[string]interface{}) map[string]interface{} {
	msg := make(map[string]interface{})

	http.DefaultTransport.(*http.Transport).TLSClientConfig = &tls.Config{InsecureSkipVerify: true}

	req := Request{
		Msg:       "Hello World! From Client.",
		Timestamp: time.Now().Format(time.RFC3339Nano),
	}

	b, err := json.Marshal(req)
	if err != nil {
		msg["err"] = err
		return msg
	}

	r, err := http.NewRequest("POST", Url, bytes.NewBuffer(b))
	if err != nil {
		msg["err"] = err
		return msg
	}
	r.Header.Add("Content-Type", "application/json")
	r.Header.Add("Authorization", Token)
	client := &http.Client{}
	resp, err := client.Do(r)
	if err != nil {
		msg["err"] = err
		return msg
	}

	defer resp.Body.Close()

	var server_resp Request
	//json.Unmarshal(b, &server_resp)
	json.NewDecoder(resp.Body).Decode(&server_resp)

	// Json returning logic
	msg["resp"] = server_resp
	msg["sent"] = req.Timestamp
	// encode the result back in json
	return msg
}

\end{lstlisting}
The server function behaves similarly and supplies a json object of similar
semantics to the client.

\section{Conculsion}
In this computer assignment we deployed a local, single node cluster onto our
local machine, Familiaring ourselves with the kubernetes environment and using
\textit{kubectl} to manage a local kubernetes cluster. \\
After that by using a helm chart to deploy our FaaS platform of choice we were
introduced to the ease of using helm charts.
Using OpenWhisk was by far different from our usual programming and debugging
environment and taugh us many things. \\
In the end this assignment follows suit from the last one that was made to
illustrate the container communication platfroms. In this assignment we used
the rest api to make connections between functions. Invoking functions without
warmup induces some overhead to our overall setup but makes the overall\
deployment easier. If our function is rarely called and
does not communicate a great amount of data, we can consider using a FaaS
platform for the ease of deployment. It can also help us by bringing functions
closer to the users in different geographical environments.
platforma



% \bibliographystyle{abbrv}
% \bibliography{references}  % need to put bibtex references in references.bib 
\end{document}
